\chapter{Introduction and overview}

\section{Notes}

``Instead of looking at quantum systems purely as phenomena to be explained..., they looked at them as systems that can be \textit{designed}...No longer is the quantum world taken merely as presented, but instead it can be \textit{created}." 

\subsection{Bloch Spheres}

For a complex number $z = x + iy$, where $x$ and $y$ are real, we can write 
$$\abs{z}^{2} = z^{*} z = (x - iy)(x + iy) = x^{2} + y^{2}$$

Using the polar representation gives us $z = r(\cos \theta + i \sin \theta)$. Substituting Euler's identity ($e^{i \theta} = \cos \theta + i \sin \theta$) results in 
$$ z = re^{i \theta}$$

We know qubits have complex amplitudes, so we can write
$$\ket*{\psi} = r_{\A} e^{i \theta_{\A}} \ket*{0} + r_{\B} e^{i \theta_{\B}} \ket*{1}$$
where all four variables are real. 

It turns out the term $e^{i \gamma}$ doesn't affect the probabilities $\abs{\A}^{2}$ or $\abs{\B}^{2}$:
$$ \abs{e^{i \gamma} \A}^{2} = (e^{i \gamma} \A)^{*} (e^{i \gamma} \A) = (e^{-i \gamma} \A^{*})(e^{i \gamma} \A) = \A^{*} \A = \abs{\A}^{2}$$
so we can multiply by $e^{-i \theta_{\A}}$ to get 
$$\ket*{\psi} = r_{\A} \ket*{0} + r_{\B} e^{i \theta} \ket*{1}$$
with three real parameters: $r_{\A}$, $r{\B}$, and $\theta = \theta_{\B} - \theta_{\A}$. 

Switching back to Cartesian coordinates and recalling the normalization condition, we have 
$$
\begin{aligned}
\abs{r_{\A}}^{2} + \abs{x + iy}^{2} &= r_{\A}^{2} + (x - iy)(x + iy) \\
&= r_{\A}^{2} + x^{2} + y^{2} = 1
\end{aligned}
$$
which is the equation for a sphere. Renaming $r_{\A}$ as $z$, we can use the identities
$$
\begin{aligned}
x &= r \sin \theta \cos \phi \\
y &= r \sin \theta \sin \phi \\
z &= r \cos \theta
\end{aligned}
$$
and after subbing in $r = 1$, we can write 
$$
\begin{aligned}
\ket*{\psi} &= z \ket*{0} + (x + iy) \ket*{1} \\
&= \cos \theta \ket*{0} + \sin \theta (\cos \phi + i \sin \phi) \ket*{1} \\
&= \cos \theta \ket*{0} + e^{i \phi} \sin \theta \ket*{1}
\end{aligned}
$$

Notice that $\theta =0 \rightarrow \ket*{\psi} = \ket*{0}$ and $\theta = \frac{\pi}{2} \rightarrow \ket*{\psi} = e^{i \phi} \ket*{1}$, which suggests that $0 \leq \theta \leq \frac{\pi}{2}$ generates all superpositions. In fact, one can easily check that plugging in $\theta' = \pi - \theta$ and $\phi' = \phi + \pi$ (the opposite point on the sphere) results in $-\ket*{\psi}$. Since the lower hemisphere of the sphere differs only by a phase factor of -1, we can choose to only consider the upper hemisphere. 

To map points in the upper hemisphere onto a sphere, we can write 
$$\ket*{\psi} = \cos \frac{\theta}{2} \ket*{0} + e^{i \phi} \sin \frac{\theta}{2} \ket*{1}$$
where $0 \leq \theta \leq \pi$ and $0 \leq \phi \leq 2\pi$. 

{
\centering \includegraphics[width=.3\linewidth]{ch1/bloch.png}\par }

The poles represent classical states. When a qubit is measured, it has higher probability of collapsing to the pole it's closer to. This representation makes it clear that the Pauli Z gate results in only a \textit{phase change} because it does not affect the state the qubit will collapse to.\footnote{Image from \href{http://akyrillidis.github.io/notes/quant_post_7}{Anastasios Kyrillidis's notes}.} Now that we've derived Bloch spheres, there are two key properties we must understand:

\begin{enumerate}
	\item \textit{Orthogonality of opposite points:} Let $\ket*{\psi} = \cos \frac{\theta}{2} \ket*{0} + e^{i \phi} \sin \frac{\theta}{2} \ket*{1}$, and let $\ket*{\varphi}$ the opposite point on the Bloch sphere,
$$\ket*{\varphi} = \cos \Big( \frac{\pi - \theta}{2} \Big) \ket*{0} + e^{i(\phi + \pi)} \sin \Big( \frac{\pi - \theta}{2} \Big) \ket*{1} = \cos \Big( \frac{\pi - \theta}{2} \Big) \ket*{0} - e^{i \phi} \sin \Big( \frac{\pi - \theta}{2} \Big) \ket*{1}$$
Using the identity $\cos(a + b) = \cos a \cos b - \sin a \sin b$, we get
$$(\varphi, \psi) = \cos \Big( \frac{\theta}{2} \Big) \cos \Big( \frac{\pi - \theta}{2} \Big) - \sin \Big( \frac{\theta}{2} \Big) \sin \Big( \frac{\pi - \theta}{2} \Big) = \cos \Big( \frac{\pi}{2} \Big) = 0$$

	\item \textit{Rotations:} The Pauli X, Y, and Z gates are so-called because when exponentiated they yield rotation operators, which rotate the Bloch vector ($\sin \theta \cos \phi$, $\sin \theta \sin \phi$, $\cos \theta$) about the $x$, $y$, and $z$ axes. For example, the exponentiated Pauli X gate is
	$$R_{X} (\theta) = e^{-i\theta X / 2}$$
The exponentiated Pauli Y and Z gates are similarly defined. 

To understand this, note that in the special case where $A^{2} = I$ (which holds for all the Pauli matrices), 
$$
\begin{aligned}
e^{i \theta A} &= I + i\theta A - \frac{\theta^{2} I}{2!} - i \frac{\theta^{3} A}{3!} + \frac{\theta^{4} I}{4!} + i \frac{\theta^{5} A}{5!} + \cdots \\
&= \Big( 1 - \frac{\theta^{2}}{2!} + \frac{\theta^{4}}{4!} + \cdots \Big) I + i \Big( \theta - \frac{\theta^{3}}{3!} + \frac{\theta^{5}}{5!} + \cdots \Big) A \\
&= \cos (\theta) I + i \sin (\theta) A
\end{aligned}
$$

Now we can exponentiate the Pauli X gate as 
$$R_{X} (\theta) = e^{-i \theta X / 2} = \cos \frac{\theta}{2} I - i \sin \frac{\theta}{2} X = \begin{bmatrix}
\cos \frac{\theta}{2} & -i \sin \frac{\theta}{2} \\
-i \sin \frac{\theta}{2} & \cos \frac{\theta}{2}
\end{bmatrix}$$

Let's consider $R_{X}( \pi)$, 
$$R_{X} (\pi) = \begin{bmatrix}
\cos \frac{\pi}{2} & -i \sin \frac{\pi}{2} \\
-i \sin \frac{\pi}{2} & \cos \frac{\pi}{2}
\end{bmatrix} = \begin{bmatrix}
0 & -i \\
-i & 0
\end{bmatrix} = -iX$$
This is a 180 degree rotation across the x-axis since we swap amplitudes and the phase changes.\footnote{Most of this section is from \href{http://www.vcpc.univie.ac.at/~ian/hotlist/qc/talks/bloch-sphere.pdf}{Ian Glendinning's talk}.} \textbf{Todo:} Need an actual reason for why. 
\end{enumerate}

\subsection{Why must quantum gates be unitary?}

For a quantum state $\ket{\psi} = \A \ket{0} + \B \ket{1}$, we know $\abs{\A}^{2} + \abs{\B}^{2} = 1$ must be true. Another way of writing this is $(\ket{\psi}, \ket{\psi}) = 1$. This must also be true after the application of a quantum gate $U$. Thus, we have 
$$
\begin{aligned}
1 &= (U \ket{\psi}, U \ket{\psi}) = (\ket{\psi}, \ket{\psi}) \\
&= (U^{\dagger} U \ket{\psi}, \ket{\psi}) = (\ket{\psi}, \ket{\psi})
\end{aligned}
$$
which means $U^{\dagger}U = I$ is a fundamental constraint on quantum gates. 

\subsection{Decomposing single qubit operations}

Move to section 4.2 according to Box 1.1 on p20. 

\subsection{What does the matrix representation of a quantum gate mean?}

$$U_{CN} = \begin{bmatrix}
1 & 0 & 0 & 0 \\
0 & 1 & 0 & 0 \\
0 & 0 & 0 & 1 \\
0 & 0 & 1 & 0
\end{bmatrix}$$

This $U_{CN}$ gate acts on two qubits, so it is a transformation on basis vectors $\ket*{0} \otimes \ket*{0}$, $\ket*{0} \otimes \ket*{1}$, $\ket*{1} \otimes \ket*{0}$, and $\ket*{1} \otimes \ket*{1}$. The columns tell us that an input of the third basis vector, $\ket*{1} \otimes \ket*{0}$, will output the fourth basis vector, $\ket*{1} \otimes \ket*{1}$. 

\textbf{Todo:} What does this say about superpositions?

\subsection{Why must quantum gates be reversible?}
Since quantum gates are unitary, we know $U^{\dagger} U = I$, which means quantum gates are invertible, and the inverse is also a quantum gate because it is unitary. 

One implication of this is that classical gates like \textsc{XOR} and \textsc{NAND} have no quantum cousins because these gates are irreversible since they result in a loss of information. 

Well, technically, the Toffoli gate can be used to simulate \textsc{NAND} gates (and therefore all classical gates), but it can only do so because it explicitly preserves all input bits. 

\section{Solutions}

\exercise

\exercise 