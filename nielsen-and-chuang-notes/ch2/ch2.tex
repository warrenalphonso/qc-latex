\chapter{Introduction to quantum mechanics}

\textbf{Exercise 2.1: (Linear dependence: example)} 

$$\begin{bmatrix}
1 \\
-1
\end{bmatrix} + \begin{bmatrix}
1 \\
2
\end{bmatrix} - \begin{bmatrix}
2 \\ 
1
\end{bmatrix} = 0$$


\textbf{Exercise 2.2: (Matrix representations: example)} 

$A$ is the NOT gate, 
$$A = \begin{bmatrix}
0 & 1 \\
1 & 0
\end{bmatrix}$$


\textbf{Exercise 2.3: (Matrix representation for operator products)} 

Figure 2.12 implies that we can write any linear transformation as a matrix whose columns are the transformed basis vectors. So
$$A = \begin{bmatrix}
A \ket{v_{1}} & \cdots & A \ket{v_{n}}
\end{bmatrix}$$
 so 
$$BA = \begin{bmatrix}
BA \ket{v_{1}} & \cdots & BA \ket{v_{n}}
\end{bmatrix}$$
I think this is a sufficient answer, though I'm not completely sure what the exercise is asking us to demonstrate. 


\textbf{Exercise 2.4: (Matrix representation for identity)} 

By definition of the identity operator, $$I \ket{v_{i}} = \ket{v_{i}}$$
Using Figure 2.12, $$I \ket{v_{i}} = \sum_{i} I_{ii} \ket{v_{i}}$$ which means $I_{ii} = 1$. 

\textbf{Exercise 2.5: }

1. $$(\ket{v}, \lambda \ket{w}) = v_{1}^{*} \lambda w_{1} + \cdots + v_{n}^{*} \lambda w_{n} =  \lambda (\ket{v}, \ket{w})$$

2. $$(\ket{v}, \ket{w}) = $$

\textbf{Exercise 2.53: What are the eigenvalues and eigenvectors of $H$?}

Solving $det(H - \lambda I) = 0$, 
$$det\Bigg(\frac{1}{\sqrt{2}} \begin{bmatrix}
1 & 1 \\
1 & -1
\end{bmatrix} - \begin{bmatrix}
\lambda & 0 \\
0 & \lambda
\end{bmatrix} \Bigg) = det\Bigg( \begin{bmatrix}
\frac{1}{\sqrt{2}} - \lambda & \frac{1}{\sqrt{2}} \\
\frac{1}{\sqrt{2}} & -\frac{1}{\sqrt{2}} - \lambda
\end{bmatrix}\Bigg) = \lambda^{2} - 1 = 0$$
So $\lambda^{2} = 1$.

For $\lambda = 1$, the eigenvector is $\frac{1}{-1 + \sqrt{2}}$.