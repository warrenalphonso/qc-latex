\chapter{Structure of Operators}

\section{Schur Representation} 

\begin{theorem}
Any operator $A: X \rightarrow X$ has an orthonormal basis $u_{1}, \cdots, u_{n}$ in $X$ such that $A$ in this basis is upper triangular. In other words, any $n \times n$ matrix $A$ can be written as 
$$A = UTU^{*}$$
where $U$ is unitary and $T$ is an upper triangular matrix. 
\end{theorem}

\begin{proof}
We proceed by induction on the dimension of $X$. 

\textbf{Base case:} $\Dim X = 1$

Then $A$ is simply a $1 \times 1$ matrix which is trivially upper triangular. 

\textbf{Inductive hypothesis:} Assume for $X$ with $\Dim X = n$, $A$ can be written as upper triangular in an orthonormal basis. 

\textbf{Inductive step:} We prove the statement for $\Dim X = n + 1$. 

Let $\lambda_{1}$ be an eigenvalue of $A$ and let $u_{1}$ be the corresponding normalized eigenvector of $A$, $Au_{1} = \lambda_{1} u_{1}$. Define $E := u_{1}^{\perp}$ and suppose $v_{2}, \cdots, v_{n + 1}$ is an orthonormal basis of $E$. Note $\Dim E = \Dim X - 1 = n$, so $u_{1}, v_{2}, \cdots, v_{n + 1}$ is an orthonormal basis of $X$. In this basis, $A$ has the form 
$$\begin{bmatrix}[c|c]
\lambda_{1} & *** \\
\hline 
\quad 0 \quad &  \\
\vdots & \qquad A_{1} \qquad \\
0 & 
\end{bmatrix}
$$
The $***$ denote values we don't care about and $A_{1}$ is an $n \times n$ block. Since $A_{1}$ denotes an operator in $E$, we can use our hypothesis to conclude there is an orthonormal basis $u_{2}, \cdots, u_{n + 1}$ in which $A_{1}$ is upper triangular. So in basis $u_{1}, \cdots, u_{n + 1}$, matrix $A_{1}$ is upper triangular, so matrix $A$ is also upper triangular.
\end{proof}

One use of the Schur representation is to present a more intuitive proof that the determinant of a matrix is the product of its eigenvalues and that the trace of a matrix is the sum of its eigenvalues. 

\section{Spectral Theorem for Self-Adjoint and Normal Operators}

\begin{definition}
An operator $A$ is \textbf{self-adjoint} if $A = A^{*}$. The matrix of a self-adjoint operator is called a \textbf{Hermitian matrix}. There is not a huge distinction between these two terms: self-adjoint refers to a general transformation while Hermitian refers to its matrix representations.
\end{definition}

\begin{theorem}
Let $A$ be a self-adjoint operator in \textbf{real or complex} inner product space $X$. Then all eigenvalues of $A$ are real and there exists an orthonormal basis of eigenvectors of $A$ in $X$. In matrix form, 
$$A = UDU^{*}$$
where $U$ is unitary and $D$ is diagonal with real entries. Additionally, if $A$ is real, $U$ can be chosen to be real. 
\end{theorem}

\begin{proof}
We apply our previous theorem to generate a Schur representation of $A$ in some orthonormal basis. What upper triangular matrices are self-adjoint? Only diagonal matrices with real entries meet this condition. This is a sufficient proof of the entire theorem. 

For an independent proof that the eigenvalues of self-adjoint operators are real, let $A = A^{*}$ and $Ax = \lambda x, x \neq 0$. Then 
$$(Ax, x) = (\lambda x, x) = \lambda (x, x) = \lambda \norm{x}^{2}$$
Additionally, 
$$(Ax, x) = (x, A^{*}x) = (x, Ax) = (x, \lambda x) = \overline{\lambda} (x, x) = \overline{\lambda} \norm{x}^{2}$$

This means $\lambda \norm{x}^{2} = \overline{\lambda} \norm{x}^{2}$, which means $\lambda = \overline{\lambda}$ since $x \neq 0$. 

We can also independently prove that eigenvectors of self-adjoint operators are orthogonal. Let $A = A^{*}$ be a self-adjoint operator and let $u, v$ be distinct eigenvectors such that $Au = \lambda u$ and $Av = \mu v$. Then 
$$(Au, v) = \lambda (u, v)$$ 
On the other hand, 
$$(Au, v) = (u, A^{*} v) = (u, Av) = (u, \mu v) = \overline{\mu} (u, v) = \mu (u, v)$$
Thus, we have $\lambda (u, v) = \mu (u, v)$. Since $\lambda \neq \mu$, $(u, v) = 0$. 
\end{proof}

What matrices are unitarily equivalent to a diagonal matrix? We know that $D^{*} D = DD^{*}$ for any diagonal matrix $D$. Therefore, $A^{*} A = AA^{*}$ if matrix $A$ is diagonal in some orthonormal basis. 

\begin{definition}
An operator $N$ is \textbf{normal} if $N^{*}N = NN^{*}$. Clearly, any self-adjoint operator ($A = A^{*}$) and any unitary operator ($U^{*}U = UU^{*} = I$) are normal. Note, however, that a normal operator must act in one space, so any unitary operator acting from one space to another is not normal. 
\end{definition}

\begin{theorem}
Any normal operator $N$ in a complex vector space has an orthogonal basis of eigenvectors. In other words, 
$$N = UDU^{*}$$
where $U$ is unitary and $D$ is diagonal. Note that even if $N$ is real, we do not claim that $U$ or $D$ are real. Also note that if $D$ is real, then by the previous theorem, $N$ is self-adjoint. 
\end{theorem}

\begin{proof}
We apply Schur representation to get an orthonormal basis such that $N$ in this basis is upper triangular. Now all we need to show is that an upper triangular normal matrix must be diagonal. 

We proceed by induction on dimension of $N$. 

\textbf{Base case: } $dim N = 1$

A $1 \times 1$ matrix is trivially upper triangular. 

\textbf{Inductive hypothesis: } Assume that any $n \times n$ upper triangular normal matrix is diagonal. 

\textbf{Inductive step: } Prove an $(n+1) \times (n+1)$ upper triangular normal matrix $N$ is also diagonal. 

$$N = \begin{bmatrix}[c|c]
a_{11} & a_{12} \cdots a_{1n} \\
\hline
0 & \\
\vdots & N_{1} \\
0 & 
\end{bmatrix}$$
where $N_{1}$ is an upper triangular $n \times n$ matrix. (This is the strongest assumption Schur's representation allows us to make.)

We know $N$ is normal and that for normal matrices $N^{*}N = NN^{*}$. Direct computation yields 
$$(N^{*} N)_{11} = \overline{a_{11}} a_{11} = \abs{a_{11}}^{2}$$
and 
$$(NN^{*})_{11} = \abs{a_{11}}^{2} + \cdots + \abs{a_{1n}}^{2}$$
so $(N^{*}N)_{11} = (NN^{*})_{11}$ if $a_{12}, \cdots, a_{1n}$ are all 0. Therefore, 
$$N = \begin{bmatrix}[c|c]
a_{11} & 0 \quad \cdots \quad 0 \\
\hline
0 & \\
\vdots & N_{1} \\
0 & 
\end{bmatrix}$$

Since $N$ is normal, 

$$N^{*}N = \begin{bmatrix}[c|c]
\abs{a_{11}}^{2} & 0 \quad \cdots \quad 0 \\
\hline
0 & \\
\vdots & N_{1}^{*} N_{1} \\
0 & 
\end{bmatrix} \qquad \text{ and } \qquad NN^{*} = \begin{bmatrix}[c|c]
\abs{a_{11}}^{2} & 0 \quad \cdots \quad 0 \\
\hline
0 & \\
\vdots & N_{1} N_{1}^{*} \\
0 & 
\end{bmatrix}$$

so $N_{1}^{*} N_{1} = N_{1} N_{1}^{*}$, which means $N_{1}$ is also normal, and by the inductive hypothesis, it is diagonal. Thus, matrix $N$ is diagonal. 
\end{proof}

The following lemma demonstrates a useful characteristic of normal operators. 

\begin{lemma}
An operator $N: X \rightarrow X$ is normal if and only if 
$$\norm{Nx} = \norm{N^{*} x} \qquad \forall x \in X$$
\end{lemma}

\begin{lproof}
Only if: Let $N$ be normal so $N^{*}N = NN^{*}$. Then 
$$\norm{Nx}^{2} = (Nx, Nx) = (N^{*}Nx, x) = (NN^{*} x, x) = (N^{*} x, N^{*} x) = \norm{N^{*} x}^{2} $$
so $\norm{Nx} = \norm{N^{*} x}$.

If: Suppose $\norm{Nx} = \norm{N^{*} x}$ and we want to show $N^{*}N = NN^{*}$. The Polarization Identities imply that $ \forall x, y \in X$, 

$$
\begin{aligned} 
(N^{*}N x, y) = (Nx, Ny) &= \frac{1}{4} \sum_{\alpha = \pm 1, \pm i} \alpha \norm{Nx + \alpha Ny}^{2} \\
&= \frac{1}{4} \sum_{\alpha = \pm 1, \pm i} \alpha \norm{N( x + \alpha y)}^{2} \\
&= \frac{1}{4} \sum_{\alpha = \pm 1, \pm i} \alpha \norm{N^{*} ( x + \alpha y)}^{2} \\
&= \frac{1}{4} \sum_{\alpha = \pm 1, \pm i} \alpha \norm{N^{*} x + \alpha N^{*} y}^{2} \\
&= (N^{*} x, N^{*} y) = (NN^{*} x, y)
\end{aligned} 
$$
and therefore, $N^{*}N = NN^{*}$. 
\end{lproof}

\section{Singular Value and Polar Decompositions}

\begin{definition}
A self-adjoint operator $A: X \rightarrow X$ is called \textbf{positive definite} if 
$$(Ax, x) > 0 \qquad \forall x \neq 0$$
and is called \textbf{positive semidefininte} or a \textbf{positive operator} if 
$$(Ax, x) \geq 0 \qquad \forall x \in X$$

We will use $A > 0$ for positive definite operators and $A \geq 0$ for positive semidefinite operators. 
\end{definition}

\begin{theorem}
Let $A = A^{*}$. Then 
\begin{enumerate}
	\item $A > 0$ if and only if all eigenvalues of $A$ are positive. 
	\item $A \geq 0$ if and only if all eigenvalues of $A$ are non-negative. 
\end{enumerate}
\end{theorem}

\begin{proof}
We can rewrite a self-adjoint matrix in some orthonormal basis as a diagonal matrix. Note that a diagonal matrix is positive definite or semidefinite if and only if all its diagonal entries (which are also eigenvalues) are positive or non-negative, respectively.  
\end{proof}

\begin{lemma}
Let $A$ be self-adjoint and positive semidefinite. There exists a unique positive semidefinite operator $B$ such that $B^{2} = A$. $B$ is called the positive square root of $A$, denoted $\sqrt{A}$.
\end{lemma}

\begin{lproof}
First, we'll prove the existence of $\sqrt{A}$. Let $v_{1}, \cdots, v_{n}$ be an orthonormal basis of eigenvectors of $A$ with corresponding eigenvalues $\lambda_{1}, \cdots, \lambda_{n}$. Since $A \geq 0$, all the eigenvalues are non-negative. In this basis, $A$ is a diagonal matrix with diagonal entries $\lambda_{1}, \cdots, \lambda_{n}$. We can define $B =\sqrt{A}$ as the matrix with diagonal entries $\sqrt{\lambda_{1}}, \cdots, \sqrt{\lambda_{n}}$. Clearly, $B^{2} = A$ and $B = B^{*} \geq 0$. 

To prove uniqueness, suppose there exists $C = C^{*} \geq 0$ such that $C^{2} = A$. Suppose $C$ is a diagonal matrix in orthonormal basis of eigenvectors $u_{1}, \cdots, u_{n}$ with diagonal entries $\mu_{1}, \cdots, \mu_{n}$. Since $A = C^{2}$, $A$ must be in the same basis as $C$, so $C$'s eigenvalues are of the form $\sqrt{\lambda_{1}}, \cdots, \sqrt{\lambda_{n}}$, which means $B = C$. 
\end{lproof}

\begin{definition}
For an operator $A: X \rightarrow Y$, its \textbf{Hermitian square $A^{*} A$} is a positive semidefinite operator in $X$. It is self-adjoint:
$$(A^{*} A)^{*} = A^{*} A^{**} = A^{*} A$$
and positive semidefinite:
$$(A^{*} Ax, x) = (Ax, Ax) = \norm{Ax}^{2} \geq 0 \qquad \forall x \in X$$

Therefore, by the above lemma there exists a unique positive semidefinite square root $R = \sqrt{A^{*} A}$. $R$ is called the \textbf{modulus} of $A$, denoted $\abs{A}$. The modulus denotes how ``big" $A$ is. 
\end{definition}

\begin{theorem}
For linear operator $A: X \rightarrow Y$, 
$$\norm{\: \abs{A} \: x} = \norm{Ax} \qquad \forall x \in X$$ 
\end{theorem}

\begin{proof}
For any $x \in X$, 
$$\norm{\: \abs{A} \: x}^{2} = (\abs{A} x, \abs{A} x) = (\abs{A}^{*} \abs{A} x, x) = (\abs{A}^{2} x, x) = (A^{*} A x, x) = (Ax, Ax) = \norm{Ax}^{2}$$
where $\abs{A}^{*} \abs{A} = \abs{A}^{2}$ because $\abs{A}$ is self-adjoint by the previous lemma. 
\end{proof}

\begin{theorem}
$$\Ker A = \Ker \abs{A} = (\Range \abs{A})^{\perp}$$
\end{theorem}

\begin{proof}
The first equality follows from the previous lemma. The second equality follows from the identity $\Ker T = (\Range T^{*})^{\perp}$ which is true because $\abs{A}$ is self-adjoint so $A = A^{*}$. 
\end{proof}

\subsection{Schmidt Decomposition}

\begin{definition}
Eigenvalues of $\abs{A}$ are called \textbf{singular values} of $A$. In other words, if $\lambda_{1}, \cdots, \lambda_{n}$ are eigenvalues of $A^{*}A$, then $\sqrt{\lambda_{1}}, \cdots, \sqrt{\lambda_{n}}$ are singular values of $A$. 
\end{definition}

Consider an operator $A : X \rightarrow Y$, and let $\sigma_{1}, \cdots, \sigma_{n}$ be the singular values of $A$. Let $\sigma_{1}, \cdots, \sigma_{r}$ be the \textit{non-zero} singular values of $A$, so that for $k > r$, $\sigma_{k} = 0$.

We know $\sigma_{1}^{2}, \cdots, \sigma_{n}^{2}$ are eigenvalues of $A^{*}A$. Let $v_{1}, \cdots, v_{n}$ be an orthonormal basis of eigenvectors of $A^{*}A$, such that $A^{*}Av_{k} = \sigma_{k}^{2} v_{k}$. This basis exists since $A^{*}A$ is self-adjoint. 

\begin{lemma}
The system 
$$w_{k} := \frac{1}{\sigma_{k}} Av_{k} \qquad \forall k = 1, \cdots, r$$
is an orthonormal system. 
\end{lemma}

\begin{lproof}
For any two eigenvectors $v_{j}, v_{k}$, 
$$\frac{1}{\sigma_{j} \sigma_{k}} (Av_{j}, Av_{k}) = \frac{1}{\sigma_{j} \sigma_{k}} (A^{*} Av_{j}, v_{k}) = \frac{1}{\sigma_{j} \sigma_{k}} (\sigma_{j}^{2} v_{j}, v_{k}) = \frac{\sigma_{j}^{2}}{\sigma_{j} \sigma_{k}} (v_{j}, v_{k})$$
which evaluates to 0 if $j \neq k$ or $1$ if $j = k$. 
\end{lproof}

\begin{definition}
We can use our formulation of $w_{k} = \frac{1}{\sigma_{k}} Av_{k}$ to rewrite 
$$A = \sum_{k = 1}^{r} \sigma_{k} w_{k} v_{k}^{*}$$
To prove this equality is true, if we multiply both sides by $x$, we can write 
$$Ax = \sum_{k = 1}^{r} \sigma_{k} (x, v_{k}) w_{k}$$
Since $v_{1}, \cdots, v_{n}$ is an orthonormal basis in $X$, we can substitute $x = v_{j}$ to get 
$$\sum_{k = 1}^{r} \sigma_{k} (v_{j}, v_{k}) w_{k} = \sigma_{j} (v_{j}, v_{j}) w_{j} = \sigma_{j} w_{j} = Av_{j} \qquad \forall j = 1, \cdots, r$$
and 
$$\sum_{k = 1}^{r} \sigma_{k} (v_{j}, v_{k}) w_{k} = 0 = Av_{j} \qquad \forall j > r$$
Since the equality holds for the orthonormal basis $v_{1}, \cdots, v_{n}$, it is true. This representation is called \textbf{Schmidt decomposition}.
\end{definition}

\begin{theorem}
Let
$$A = \sum_{k=1}^{r} \sigma_{k} w_{k} v_{k}^{*}$$
where $\sigma_{k} > 0$, and $v_{1}, \cdots, v_{r}$ and $w_{1}, \cdots, w_{r}$ are orthonormal systems. Then, this representation gives a Schmidt decomposition of $A$. 
\end{theorem}

\begin{proof}
We only need to show that $v_{1}, \cdots, v_{r}$ are eigenvectors of $A^{*}A$. 

Since $w_{1}, \cdots, w_{r}$ is an orthonormal system, 
$$w_{k}^{*} w_{j} = (w_{j}, w_{k})$$ 
which is 1 if $j = k$ and 0 otherwise. 

Therefore, we can write 
$$A^{*} A = \sum_{k=1}^{r} \sum_{j=1}^{r} \sigma_{k} \sigma_{j}    v_{k} w_{k}^{*} w_{j} v_{j}^{*} = \sum_{k=1}^{r} \sigma_{k}^{2} v_{k} v_{k}^{*}$$

If we multiply both sides by $v_{j}$, we get 
$$A^{*} A v_{j} = \sum_{k=1}^{r} \sigma_{k}^{2} v_{k} v_{k}^{*}v_{j} = \sum_{k=1}^{r} \sigma_{k}^{2} v_{k} (v_{j}, v_{k}) = \sigma_{j}^{2} v_{j}$$

which means $v_{1}, \cdots, v_{r}$ are eigenvectors of $A^{*} A$. 
\end{proof}

A corollary of this result is that if 
$$A = \sum_{k=1}^{r} \sigma_{k} w_{k} v_{k}^{*}$$
is a Schmidt decomposition of $A$, then 
$$A^{*} = \sum_{k=1}^{r} \sigma_{k} v_{k} w_{k}^{*}$$
is a Schmidt decomposition of $A^{*}$. 

\subsection{Singular Value Decomposition}

Let $A : \F^{n} \rightarrow \F^{m}$ and $\sigma_{1}, \cdots, \sigma_{r}$ be the non-zero singular values of $A$, and let 
$$A = \sum_{k=1}^{r} \sigma_{k} w_{k} v_{k}^{*}$$ 
be a Schmidt decomposition of $A$. We can rewrite this in matrix form as 
$$A = \widetilde{W} \, \widetilde{\Sigma} \, \widetilde{V}^{*}$$
where $\widetilde{\Sigma}$ is the diagonal matrix of $\sigma_{1}, \cdots, \sigma_{r}$, and $\widetilde{V}$ and $\widetilde{W}$ are matrices with columns $v_{1}, \cdots, v_{r}$ and $w_{1}, \cdots, w_{r}$, respectively. Note that $\widetilde{V}$ is an $n \times r$ matrix and $\widetilde{W}$ is an $m \times r$ matrix. 

\begin{theorem}
The number of nonzero singular values of $A$ equals the rank of $A$.
\end{theorem}

\begin{proof}
Recall that the singular values of $A$ are the eigenvalues of $A^{*}A$, and the number of nonzero eigenvalues of a square matrix correspond with the matrix's rank.  

Now we'll show $\Ker A = \Ker A^{*}A$. For $x \in \Ker A$, clearly we have $A^{*}Ax = 0$. For $x \in \Ker A^{*}A$, we have $A^{*}Ax = 0$. Left-multiplying by $x^{*}$ gives us $x^{*} A^{*} A x = (Ax)^{*} (Ax) = 0$, which means $Ax = 0$. 

Since $\Ker A = \Ker A^{*}A$, the rank-nullity theorem asserts that $A^{*}A$ and $A$ have the same rank. 
\end{proof}

If $A$ is invertible, then $m = n = r$, so $\widetilde{V}$ and $\widetilde{W}$ are unitary, and $\widetilde{\Sigma}$ is an invertible diagonal matrix. 

Furthermore, we can always write a similar representation with matrices $V$ and $W$, where $V$ is $n \times n$ and $W$ is $m \times m$. 

To find this representation, we need to complete the bases $v_{1}, \cdots, v_{r}$ and $w_{1}, \cdots, w_{r}$ to orthogonal bases in $\F^{n}$ and $\F^{m}$, respectively. To do this for $V$, we need to only find an orthogonal basis $v_{r+1}, \cdots, v_{n}$ in $\Ker V^{*}$, and we can always do this using Gram-Schmidt. 

Thus, we can write 
$$A = W \Sigma V^{*}$$
where $V$ is $n \times n$, $W$ is $m \times m$, and $\Sigma$ is a ``diagonal" $m \times n$ matrix of singular values. This representation is the \textbf{singular value decomposition} of $A$. The previous representation, $A = \widetilde{W} \, \widetilde{\Sigma} \, \widetilde{V}^{*}$, is the \textbf{reduced} or \textbf{compact} SVD of $A$.

Looking at the singular value decomposition, it is clear that the diagonal entries of $\Sigma$ are the singular values of $A$ (square root of eigenvalues of $A^{*}A$). To see this, we can write 
$$A^{*} A = V \Sigma W^{*} W \Sigma V^{*} = V \Sigma^{2} V^{*}$$
so that spectral decomposition tells us that $\Sigma^{2}$ is the diagonal matrix of eigenvalues and $v_{1}, \cdots, v_{n}$ are the corresponding eigenvectors of $A^{*} A$. Recall that if $\sigma_{k} \neq 0$ then $w_{k} = \frac{1}{\sigma_{k}} Av_{k}$. This means that $A = W \Sigma V^{*}$ can be obtained through the Schmidt decomposition as described earlier. 

Similarly, we can understand the reduced singular value decomposition as a \textit{matrix form} of the Schmidt decomposition for a \textit{non-invertible} matrix $A$. 

\begin{theorem}
If we have the SVD $A = W \Sigma V^{*}$, we can write the polar decomposition
$$A = (WV^{*}) (V \Sigma V^{*}) = U \abs{A}$$
where $\abs{A} = V \Sigma V^{*}$ and $U = WV^{*}$ is unitary. 
\end{theorem}

\begin{proof}
Notice that 
$$A^{*} A = V \Sigma W^{*} W \Sigma V^{*} = V \Sigma \Sigma V^{*} = V \Sigma V^{*} V \Sigma V^{*} = (V \Sigma V^{*})^{2}$$
This indicates that $\abs{A} = V \Sigma V^{*}$. 

Finally, since $WV^{*}$ is the product of unitary operators, it must also be unitary. Note that this only works if $A$ is square because $W$ and $V$ must also be square so that $WV^{*}$ is defined. 
\end{proof}

We'll skip an in-depth explanation of the polar decomposition for now. It will be revisited in the next chapter. 

\section{Applications of Singular Value Decomposition}

The SVD $A = W \Sigma V^{*}$ can be seen as the matrix $A$ in two different orthonormal bases: $v_{1}, \cdots, v_{n}$ and $w_{1}, \cdots, w_{n}$, so that $\Sigma = [A]_{WV}$. Because this is with respect to two \textit{different} bases, it doesn't tell us much about $A$'s spectral properties. However, singular values do tell us a lot about the \textit{metric properties} of a linear transformation. 

\subsection{Operator Norm}

Given $A: X \rightarrow Y$, we want to find the maximum of $\norm{Ax}$ on the closed unit ball $B = \{ x \in X : \norm{x} \leq 1 \}$. 

First, consider a diagonal matrix $A$ with non-negative entries. The maximum is exactly the maximal diagonal entry. If we have $r$ non-zero diagonal entries of $A$ with $s_{1}$ being the maximum, we know $\norm{Ax} \leq s_{1} \norm{x}$ because 
$$\norm{Ax}^{2} = \sum_{k=1}^{r} s_{k}^{2} \abs{x_{k}}^{2} \leq s_{1}^{2} \sum_{k=1}^{r} \abs{x_{k}}^{2} = s_{1}^{2} \norm{x}^{2}$$

For the general case, we can consider the SVD, $A = W \Sigma V^{*}$, where $W, V$ are unitary and $\Sigma$ is diagonal with non-negative entries. Since unitary transformations preserve the norm, we know the maximum of $\norm{Ax}$ on $B$ is the maximal diagonal entry of $\Sigma$, which is the maximal singular value of $A$. 

\begin{definition}
The maximum of $\norm{Ax}$ such that $\norm{x} \leq 1$ is called the \textbf{operator norm} of $A$, denoted $\norm{A}$. We can easily verify that it satisfies the 4 properties of the norm (Symmetry, Linearity, Non-negativity, Non-degeneracy). 

One of the main properties of the operator norm is 
$$\norm{Ax} \leq \norm{A} \cdot \norm{x}$$
which follows from realizing $\norm{Ax} - \norm{A} \leq 0$, so the inequality reduces to $0 \leq \norm{x}$. In fact, this often used to define the operator norm, since this property means the operator norm is the smallest number $C \geq 0$ such that 
$$\norm{Ax} \leq C \norm{x} \qquad \forall x \in X$$
\end{definition}

There is another common norm on the space of linear transformations. 

\begin{definition}
The \textbf{Hilbert-Schmidt inner product} on operators $M$ and $N$ is defined as 
$$ (M, N) = \Trace(N^{*}M)$$
so the induced norm is 
$$\norm{M}_{HS} = \sqrt{(M, M)} = \sqrt{\Trace(M^{*}M)}$$
\end{definition}

To compare these two norms, let $s_{1}, \cdots, s_{r}$ be non-zero singular values of $A$ with $s_{1}$ being the largest. We know $s_{1}^{2}, \cdots, s_{r}^{2}$ are the non-zero eigenvalues of $A^{*} A$. Since the trace is the sum of the eigenvalues, we know that 
$$\norm{A}_{HS}^{2} = \Trace(A^{*}A) = \sum_{k=1}^{r} s_{k}^{2}$$
We also know that the operator norm equals its largest singular value ($\norm{A} = s_{1}$), so we conclude that the operator norm of a matrix cannot be more than its Hilbert-Schmidt norm. 

\subsection{Moore-Penrose Pseudoinverse}

Recall that finding a least square solution amounts to solving the normal equation 
$$A^{*} A \hat{x} = A^{*} b$$
and that we can use this to write the matrix form of a projection as 
$$P_{\Range A} = A(A^{*} A)^{-1} A^{*}$$
If $A = \widetilde{W} \, \widetilde{\Sigma} \, \widetilde{V}^{*}$ is the reduced singular value decomposition of $A$, then we can define 
$$ x_{0} = \widetilde{V} \, \widetilde{\Sigma}^{-1} \, \widetilde{W}^{*} b$$

Indeed, $x_{0}$ is a least square solution of $Ax = b$, which means $Ax_{0} = P_{\Range A} b$. To see this, 
$$Ax_{0} = \widetilde{W} \, \widetilde{\Sigma} \, \widetilde{V}^{*} \widetilde{V} \, \widetilde{\Sigma}^{-1} \, \widetilde{W}^{*} b = \widetilde{W} \widetilde{W}^{*} b = P_{\Range \widetilde{W}} b = P_{\Range A} b $$

Our final step uses the fact that $\widetilde{W} \widetilde{W}^{*} = P_{\Range \widetilde{W}}$: 
$$P_{\Range \widetilde{W}} = \widetilde{W} (\widetilde{W}^{*} \widetilde{W})^{-1} \widetilde{W}^{*} = \widetilde{W} (I)^{-1} \widetilde{W}^{*} = \widetilde{W} \widetilde{W}^{*}$$
and that $P_{\Range \widetilde{W}} = P_{\Range A}$:
$$
\begin{aligned}
P_{\Range A} &= (\widetilde{W} \widetilde{\Sigma} \widetilde{V}^{*}) \big( (\widetilde{W} \widetilde{\Sigma} \widetilde{V}^{*})^{*} (\widetilde{W} \widetilde{\Sigma} \widetilde{V}^{*}) \big)^{-1} (\widetilde{W} \widetilde{\Sigma} \widetilde{V}^{*})^{*} \\
&= \widetilde{W} \widetilde{\Sigma} \widetilde{V}^{*} ( \widetilde{V} \widetilde{\Sigma}^{*} \widetilde{W}^{*} \widetilde{W} \widetilde{\Sigma} \widetilde{V}^{*} )^{-1} \widetilde{V} \widetilde{\Sigma}^{*} \widetilde{W}^{*} \\
&= \widetilde{W} \widetilde{\Sigma} \widetilde{V}^{*} \widetilde{V} \widetilde{\Sigma}^{-1} \widetilde{\Sigma}^{*^{-1}} \widetilde{V}^{*}  \widetilde{V} \widetilde{\Sigma}^{*} \widetilde{W}^{*} \\
&= \widetilde{W} \widetilde{W}^{*} = P_{\Range \widetilde{W}}
\end{aligned}
$$

\begin{definition}
The operator $A^{+} = \widetilde{V} \, \widetilde{\Sigma}^{-1} \, \widetilde{W}^{*} $ is called the \textbf{Moore-Penrose pseudoinverse} of $A$. In other words, the Moore-Pensrose pseudoinverse gives the least square solution to the equation $Ax = b$. 

The following properties of the Moore-Penrose pseudoinverse can be easily proved: 
\begin{enumerate}
	\item $AA^{+}A = A$
	\item $A^{+} A A^{+} = A^{+}$
	\item $(AA^{+})^{*} = AA^{+}$
	\item $(A^{+} A)^{*} = A^{+} A$
\end{enumerate}
\end{definition}

\begin{theorem}
The operator $A^{+}$ satisfying the above 4 properties is unique. 
\end{theorem}

\begin{proof}
Left and right multiplying property 1 by $A^{+}$ gives us: 
$$(A^{+} A)^{2} = A^{+} A$$ 
and
$$ (AA^{+})^{2} = AA^{+}$$
Properties 3 and 4 mean these two projections are also self-adjoint. We know these two properties mean both $A^{+}A$ and $AA^{+}$ are orthogonal projections.  

Clearly, $\Ker A \subset \Ker A^{+}A$. To prove $\Ker A^{+}A \subset \Ker A$, suppose we have some $u \in \Ker A^{+}A$. Right-multiplying property 1 by $u$ gives us $AA^{+}Au = Au$, which simplifies to $0 = Au$, so $u \in \Ker A$. Thus, $\Ker A = \Ker A^{+}A$, which means $A^{+}A$ is the orthogonal projection onto $(\Ker A)^{\perp} = \Range A^{*}$, 
$$A^{+}A = P_{\Range A^{*}}$$

Property 1 implies that $AA^{+} y = y$ for any $y \in \Range A$, so $\Range A \subset \Range AA^{+}$. Clearly, $\Range AA^{+} \subset \Range A$, so $\Range A = \Range AA^{+}$ which means $AA^{+}$ is the orthogonal projection onto $\Range A$, 
$$AA^{+} = P_{\Range A}$$

Now we can rewrite property 2 as 
$$P_{\Range A^{*}} A^{+} = A^{+} \qquad \text{ or } \qquad A^{+} P_{\Range A} = A^{+}$$

Combining these gives us 
$$P_{\Range A^{*}} A^{+} P_{\Range A} = A^{+}$$

This means that for any $b \in \Range A$, 
$$x_{0} := A^{+} b = P_{\Range A^{*}} A^{+} b \in \Range A^{*}$$
and that 
$$Ax_{0} = AA^{+} b = P_{\Range A} b$$
In other words, $x_{0}$ is a least square solution to $Ax = b$. However, since $x_{0} \in \Range A^{*} = (\Ker A)^{\perp}$, $x_{0}$ is the least square solution of minimal norm, which we know is given by $A^{+} = \widetilde{V} \, \widetilde{\Sigma}^{-1} \, \widetilde{W}^{*}$.
\end{proof}

\textbf{Generalize Moore-Penrose to the SVD: $A = W\Sigma V^{*}$. }

\section{Structure of Orthogonal Matrices}

An orthogonal matrix $U$ with $\Det(U) = 1$ is called a \textbf{rotation}. The following theorem illustrates why. 

\begin{theorem}
Let $U$ be an orthogonal operator in $\R^{n}$ and let $\Det(U) = 1$. Then there exists an orthonormal basis $v_{1}, \cdots, v_{n}$ such that the matrix of $U$ in this basis has the \textit{block diagonal form}:
$$\begin{bmatrix}
R_{\varphi 1} & & & & \makebox{\Huge 0} \\
 & R_{\varphi 2} & & & \\ 
 & & \ddots & & \\
 & & & R_{\varphi k} & \\
 \makebox{\Huge 0} & & & & I_{n - 2k}
\end{bmatrix}$$
where $R_{\varphi k}$ are 2-dimensional rotations, 
$$R_{\varphi k} = \begin{bmatrix}
\cos \ \varphi k & -\sin \  \varphi k \\
\sin \ \varphi k & \cos \ \varphi k
\end{bmatrix}$$
\end{theorem}

\begin{proof}
This is a long proof!

We know that if $p$ is a polynomial with real coefficients and $\lambda$ is its complex root ($p(\lambda) = 0$), then $\overline{\lambda}$ is also a root ($p(\overline{\lambda}) = 0$). This means all complex eigenvalues of a real matrix $A$ can be split into pairs $\lambda_{k}, \overline{\lambda_{k}}$. Since we also know eigenvalues of unitary matrices have absolute value 1, we can write any eigenvalue as $\lambda_{k} = \cos a_{k} + i \sin a_{k}$ and $\overline{\lambda_{k}} = \cos a_{k} -i \sin a_{k}$.

We fix a pair of eigenvalues $\lambda$ and $\overline{\lambda}$, and let $u$ be the eigenvector of $\lambda$, so $U u = \lambda u$. This means $U \overline{u} = \overline{\lambda} \overline{u}$. Now, we split $u$ into \textit{real} and \textit{imaginary} parts. Define  
$$x := Re(u) = \frac{u + \overline{u}}{2} \qquad y := Im(u) = \frac{u - \overline{u}}{2i}$$ 
so that $u = x + iy$. Then 
$$Ux = \frac{1}{2} U (u + \overline{u}) = \frac{1}{2} (\lambda u + \overline{\lambda} \overline{u}) = Re(\lambda u)$$ 
and 
$$Uy = \frac{1}{2i} U(u - \overline{u}) = \frac{1}{2i} (\lambda u - \overline{\lambda} \overline{u}) = Im(\lambda u)$$ 
Finally, because $\lambda = \cos a + i \sin a$, we can write 
$$\lambda u = (\cos a + i \sin a)(x + iy) = ((\cos a) x - (\sin a) y) + i((\cos a) y + (\sin a) x)$$
which means 
$$Ux = Re(\lambda u) = (\cos a) x - (\sin a) y \qquad Uy = Im(\lambda u) = (\cos a) y + (\sin a) x$$

In other words, $U$ preserves the subspace spanned by $x$ and $y$, denoted $E_{\lambda}$, and the matrix of the \textit{restriction of} $U$ onto this subspace is the rotation matrix 
$$R_{a} = \begin{bmatrix}
\cos a & \sin a \\
-\sin a & \cos a
\end{bmatrix}$$

Since $u$ and $\overline{u}$ are orthogonal eigenvectors of a unitary matrix, we can use the Pythagorean Theorem to write 
$$\norm{x} = \norm{y} = \frac{\sqrt{2}}{2} \norm{u}$$

It is clear from their definitions that $x$ and $y$ are orthogonal, so they form an orthogonal basis in $E_{\lambda}$. Since we can multiply each vector in a basis by a non-zero constant and not change the matrices of linear transformations, we can assume $\norm{x} = \norm{y} = 1$. 

Now we'll complete the orthonormal system $v_{1} = x, v_{2} = y$ to an orthonormal basis in $\R^{n}$. 

Since $U E_{\lambda} \subset E_{\lambda}$ (because $E_{\lambda}$ is an invariant subspace of $U$), the matrix of $U$ in this basis has the \textit{block triangular form} 
$$\begin{bmatrix}[c|c]
 \: R_{a} \:  & * \\
\hline 
0 & \:  U_{1} \:  \\
\end{bmatrix}$$
where 0 represents the $(n - 2) \times 2$ zero matrix. 

Since the rotation matrix $R_{a}$ is invertible, we can write $UE_{\lambda} E_{\lambda}$. Thus, 
$$U^{*} E_{\lambda} = U^{-1} E_{\lambda} = E_{\lambda}$$
so the matrix $U$ becomes 
$$\begin{bmatrix}[c|c]
 \: R_{a} \:  & 0 \\
\hline 
0 & \:  U_{1} \:  \\
\end{bmatrix}$$

Because we know $U$ is unitary, we can write 
$$I = U^{*}U = \begin{bmatrix}[c|c]
\: I \: & 0 \\
\hline
0 & \: U_{1}^{*} U_{1} \:
\end{bmatrix}$$
and because $U_{1}$ is square, it must also be unitary. 

If $U_{1}$ has complex eigenvalues, we can continue applying this procedure to decrease its size by 2 until we are left with a block with only real eigenvalues. Since real eigenvalues are either 1 or -1, in some orthonormal basis the matrix of $U$ has the form 
$$\begin{bmatrix}
R_{a_{1}} & & & & & \makebox{\Huge 0} \\
 & R_{a_{2}} & & & \\ 
 & & \ddots & & \\
 & & & R_{a_{d}} & \\
 & & & & -I_{r} & \\
 \makebox{\Huge 0} & & & & & I_{l}
\end{bmatrix}$$
where $I_{r}$ and $I_{l}$ are identity matrices of size $r$ and $l$, respectively. Sicne $\Det(U) = 1$, the multiplicity of eigenvalue -1 must be even. 

Finally, because a negative $2 \times 2$ identity matrix can be interpreted as the rotation through the angle $\pi$, we arrive at the form given in the theorem with $\varphi_{k} = \pi$. 
\end{proof}
