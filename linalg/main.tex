\documentclass[a4paper,10pt]{book}

%Use the following for font size 8 or 9, as 10 is the minimum provided normally
%\documentclass[8pt]{extbook}


%Compilers
\usepackage[utf8]{inputenc}
\usepackage[english]{babel}

%Graphics
\usepackage{xcolor}
\usepackage{graphicx}

%Math
\usepackage{amsmath}
\usepackage{amssymb}
\usepackage{caption}
\usepackage{physics}
\usepackage{amsthm}
\usepackage{mathtools}
\usepackage[xcolor]{mdframed}

%Math shortcuts 
\newcommand{\R}{\mathbb{R}}
\newcommand{\C}{\mathbb{C}}
\newcommand{\F}{\mathbb{F}}
\newcommand{\M}{\mathcal{M}}
\renewcommand{\P}{\mathcal{P}}
\renewcommand{\L}{\mathcal{L}}
\DeclareMathOperator{\Span}{span}
\DeclareMathOperator{\Dim}{dim}
\DeclareMathOperator{\Codim}{codim}
\DeclareMathOperator{\Range}{range}
\DeclareMathOperator{\Rank}{rank}
\DeclareMathOperator{\Null}{null}
\DeclareMathOperator{\Ker}{ker}
\let \Trace \relax %\Trace is already defined so resetting
\DeclareMathOperator{\Trace}{tr} 
\DeclareMathOperator{\Det}{det}


%Augmented matrices
\makeatletter
\renewcommand*\env@matrix[1][*\c@MaxMatrixCols c]{%
   \hskip -\arraycolsep
   \let\@ifnextchar\new@ifnextchar
   \array{#1}}
\makeatother


%Define abs
%\DeclarePairedDelimiter\abs{\lvert}{\rvert}

%Defining norm
%\DeclarePairedDelimiter{\norm}{\lVert}{\rVert}

%Defining inner product
\DeclarePairedDelimiterX{\inp}[2]{\langle}{\rangle}{#1, #2}


%Reducing margin
\usepackage[margin=2cm]{geometry}

%Header, footer
\usepackage{fancyhdr}
\pagestyle{fancy}
\fancyhf{}
\rhead{\leftmark}
\cfoot{\thepage} 
\renewcommand{\headrulewidth}{0pt}

%Index - not used
\usepackage{makeidx}
\makeindex



%Theorem instantiation
\definecolor{lightblue1}{RGB}{222, 243, 253}
\theoremstyle{plain}
\newtheorem{theorem}{Theorem}[section]
\surroundwithmdframed[
	hidealllines = true, 
	backgroundcolor = lightblue1,
	]{theorem}

%NEW theorem environment
%\definecolor{lightblue1}{RGB}{222, 243, 253}
%\newcounter{thmcounter}
%\newmdenv[
%	hidealllines = true, 
%	backgroundcolor = lightblue1,
%]{mytheorem}
%\newenvironment{theorem}[1][\unskip]{ %takes optional argument 1, \unskip removes the space below if no optional argument
%	\begin{mytheorem}
%	\refstepcounter{thmcounter} %increment counter 
%	\textbf{Theorem \thethmcounter} %\the__counter__ gets value of counter
%	\textsl{ #1}:
%}{
%	\end{mytheorem}
%}
%\numberwithin{thmcounter}{section}
	

%Proof instantiation 
\renewenvironment{proof}{\textsl{Proof.}}{\hfill$\blacksquare$}
\definecolor{lightblue2}{RGB}{232, 245, 252}
\surroundwithmdframed[
	hidealllines = true, 
	backgroundcolor = lightblue2,
	]{proof}
	
	
%NEW proof 
%\definecolor{lightblue2}{RGB}{232, 245, 252}
%\newmdenv[
%	hidealllines = true, 
%	backgroundcolor = lightblue2, 
%]{myproof}
%\renewenvironment{proof}{
%	\bigskip 
%	\noindent
%	\begin{myproof}
%	\textit{Proof.}
%}{
%	\end{myproof}
%	\par 
%	\bigskip
%}

	
%Lemma instantiation
\definecolor{lightgreen1}{RGB}{224, 255, 193}
\theoremstyle{plain}
\newtheorem{lemma}[theorem]{Lemma}
\surroundwithmdframed[
	hidealllines = true, 
	backgroundcolor = lightgreen1,
	]{lemma}
	
	
%NEW lemma 
%\definecolor{lightgreen1}{RGB}{224, 255, 193}
%\newmdenv[
%	hidealllines = true, 
%	backgroundcolor = lightgreen1,
%]{mylemma}
%\newenvironment{lemma}[1][\unskip]{ %takes optional argument 1, \unskip removes the space below if no optional argument
%	\bigskip 
%	\noindent
%	\begin{mylemma}
%	\refstepcounter{thmcounter} %increment counter 
%	\textbf{Lemma \thethmcounter} %\the__counter__ gets value of counter
%	\textsl{ #1}:
%}{
%	\end{mylemma}
%	\par 
%	\bigskip
%}
%\numberwithin{thmcounter}{section}

%Lemma Proof instantiation 
\newenvironment{lproof}{\textsl{Proof.}}{\hfill$\blacksquare$}
\definecolor{lightgreen2}{RGB}{232, 249, 214}
\surroundwithmdframed[
	hidealllines = true, 
	backgroundcolor = lightgreen2,
	]{lproof}
	

%NEW lproof 
%\definecolor{lightgreen2}{RGB}{232, 249, 214}
%\newmdenv[
%	hidealllines = true, 
%	backgroundcolor = lightgreen2,
%]{mylproof}
%\newenvironment{lproof}{
%	\bigskip 
%	\noindent
%	\begin{mylproof}
%	\textit{Proof.}
%}{
%	\end{mylproof}
%	\par 
%	\bigskip
%}


%Definition instantiation
\theoremstyle{definition}
\newtheorem{definition}{Definition}[section]
\definecolor{lightgreen}{RGB}{219, 255, 188}
\definecolor{subtlegray}{RGB}{248, 248, 248}
\surroundwithmdframed[
	hidealllines = true, 
	backgroundcolor = subtlegray,
	]{definition}
		
	
%Remove auto-indentation
\setlength{\parindent}{0cm}

%Preferred font and spacing
\linespread{1.3}
%\usepackage{lmodern}
\usepackage{kpfonts}

%Title color
\definecolor{lightyellow}{RGB}{255, 236, 148}

%Table of Contents 
\usepackage{hyperref}
\hypersetup{
    colorlinks,
    citecolor=black,
    filecolor=black,
    linkcolor=blue,
    urlcolor=black
}



\includeonly{outer_product_representation/outer_product_representation}

\begin{document}

\frontmatter 
{\let\cleardoublepage\clearpage 
%Fancy shmancy title page - code from https://en.wikibooks.org/wiki/LaTeX/Title_Creation

\begin{titlepage}
	\centering
	\vspace{1cm}
	{\scshape\LARGE University of California, Berkeley \par}
	\vspace{4cm}
	{\scshape\Large Literally everything I know about \par}
	\vspace{1.5cm}
	{\Huge\bfseries Linear Algebra\par}
	\vspace{1cm}
	\vspace{2.5cm}
	{\Large\itshape Warren Alphonso\par}
	\vfill
	{\large A very reductionist summary of \textsc{Linear Algebra and its Applications} by Lay, Lay, and McDonald, \textsc{Linear Algebra Done Wrong} by Treil, \textsc{Advanced Linear Algebra} by Surowski, and \textsc{Quantum Computation and Quantum Information} by Nielsen and Chuang. \par}
	\vfill
	{\large Last updated: \today \par}
\end{titlepage}

\tableofcontents
}

\mainmatter

\chapter{Basic Notations}
\section{Vector Spaces}
\begin{definition}
A \textbf{vector space} $V$ is a collection of vectors, along with vector addition and scalar multiplication defined such that for vectors $u$, $v$, and $w$: 

\begin{enumerate}
	\item Commutative: $v + w = w + v$
	\item Associative: $(u + v) + w = u + (v + w)$
	\item Zero vector: $ v + 0 = v$
	\item Additive inverse: $ v + (-v) = 0$
	\item Multiplicative identity: $ 1v = v$
	\item Multiplicative associative: $ (\alpha \beta)v = \alpha(\beta v)$
	\item Distribution of scalars: $ \alpha(u + v) = \alpha u + \alpha v$
	\item Distribution of vectors: $ (\alpha + \beta)u = \alpha u + \beta u$
\end{enumerate}
\end{definition}

These properties ensure that vector spaces are \textbf{abelian groups}.

\begin{definition}
An $m \times n$ \textbf{matrix} is an array with $m$ rows and $n$ columns. Elements of a matrix are called \textit{entries}. Given a matrix $A$, its \textbf{transpose} is defined as the matrix whose columns are $A$'s rows, so $A^{T}$ is an $n \times m$ matrix. 
\end{definition}

\section{Linear Combinations}
\begin{definition}
A \textbf{linear combination} of vectors $v_{1}, \cdots, v_{p} \in V$ is a sum of the form 
$$\alpha_{1} v_{1} + \cdots + \alpha_{p} v_{p} = \sum_{k = 1}^{p} \alpha_{k} v_{k}$$
\end{definition}

\begin{definition}
A set of vectors $v_{1}, \cdots, v_{n}$ is said to be \textbf{linearly independent} if the equation 
$$\alpha_{1} v_{1} + \cdots + \alpha_{n} v_{n} = 0$$
has only the trivial solution where all coefficients are 0. 
\end{definition}

\begin{definition}
A \textbf{basis} is a set of vectors $v_{1}, \cdots , v_{n} \in V$ such that any vector $u \in V$ has a \textit{unique} representation as a linear combination 
$$u = \alpha_{1} v_{1} + \cdots + \alpha_{n} v_{n}$$
The coefficients $\alpha_{1}, \cdots, \alpha_{n}$ are called \textit{coordinates} of $u$.
\end{definition}

Fundamentally, our definition of basis requires that it must be spanning and unique. In order for a representation to be unique, we know the basis must be linearly independent. 

\begin{theorem}
A set of vectors $v_{1}, \cdots, v_{p} \in V$ is a basis if and only if it is linearly independent and complete (spanning). 
\end{theorem}

\begin{proof}
We already know a basis must be linearly independent and spanning, so we just need to prove the other direction. 

Suppose the set $v_{1}, \cdots, v_{p}$ is linearly independent. Then we know for some vector $u \in V$:
$$u = \sum_{k=1}^{n} \alpha_{k} v_{k}$$

All that is remaining is to prove this representation is unique. 

Suppose there is another representation, $u = \sum_{k=1}^{n} \beta_{k} v_{k}$. Then, 
$$\sum_{k=1}^{n} (\alpha_{k} - \beta_{k}) v_{k} = v - v = 0$$
Since the set is linearly independent, we know $\alpha_{k} - \beta_{k} = 0$. Thus, the representation is unique. 
\end{proof}

\section{Linear Transformations}
\begin{definition}
A \textbf{transformation} $T$ from set $X$ to set $Y$ assigns a value $y \in Y$ for every value $x \in X$: $y = T(x)$. $X$ is called the \textit{domain} of $T$, $Y$ is called the \textit{codomain} of $T$, and the set of all $T(x)$ is called the \textit{range} of $T$. 

Let $V, W$ be vector spaces. A transformation $T: V \rightarrow W$ is \textbf{linear} if:
\begin{enumerate}
	\item $T(u + v) = T(u) + T(v)$
	\item $T(\alpha v) = \alpha T(v)$
\end{enumerate}

A mapping $T: \mathbb{F}^{n} \rightarrow \mathbb{F}^{m}$ is \textit{onto} $\mathbb{F}^{m}$ if each $b$ in $\mathbb{F}^{m}$ is the image of at least one $x$ in $\mathbb{F}^{n}$. 

A mapping $T: \mathbb{F}^{n} \rightarrow \mathbb{F}^{m}$ is \textit{one-to-one} if each $b$ in $\mathbb{F}^{m}$ is the image of at most one $x$ in $\mathbb{F}^{n}$. 
\end{definition}

We can represent linear transformations with matrices. To represent a transformation $T: \mathbb{F}^{n} \rightarrow \mathbb{F}^{m}$, we need to only know our $n$ basis vectors are transformed. To see this, note that any vector $u = \alpha_{1} v_{1} + \cdots + \alpha_{n} v_{n}$. So $T(u) = \alpha_{1} T(v_{1}) + \cdots + \alpha_{n} T(v_{n})$. If we join the vectors $T(v_{1}), \cdots, T(v_{n})$ in a matrix $A = \begin{bmatrix}
T(v_{1}) & \cdots & T(v_{n})
\end{bmatrix}$, we have captured all the information about $T$. 

\begin{definition}
There are two ways to approach \textbf{matrix-vector multiplication}: 

\textit{Column by coordinate rule:} Multiply each column of the matrix by the corresponding coordinate of the vector and add. 
$$\begin{bmatrix}
1 & 2 & 3 \\
3 & 2 & 1
\end{bmatrix} \begin{bmatrix}
1 \\
2 \\
3
\end{bmatrix} = 1 \begin{bmatrix}
1 \\
3
\end{bmatrix} + 2 \begin{bmatrix}
2 \\
2
\end{bmatrix} + 3 \begin{bmatrix}
3 \\
1
\end{bmatrix} = \begin{bmatrix}
14 \\
10
\end{bmatrix}$$

\textit{Row by column rule:} To get entry $k$ of the result, multiply row $k$ of the matrix with the vector. 
$$\begin{bmatrix}
1 & 2 & 3 \\
3 & 2 & 1
\end{bmatrix} \begin{bmatrix}
1 \\
2 \\ 
3
\end{bmatrix} = \begin{bmatrix}
1 \cdot 1 + 2 \cdot 2 + 3 \cdot 3 \\
3 \cdot 1 + 2 \cdot 2 + 1 \cdot 3
\end{bmatrix} = \begin{bmatrix}
14 \\
10
\end{bmatrix}$$
\end{definition}

\begin{definition}
The natural extension to \textbf{matrix multiplication} of two matrices $AB$ is to multiply $A$ by each column of $B$. 
$$AB = A \begin{bmatrix}
b_{1} & \cdots & b_{n}
\end{bmatrix} = \begin{bmatrix}
Ab_{1} & \cdots & Ab_{n}
\end{bmatrix}$$
Using the \textit{row by columns rule}, we can see that the $(AB)_{j,k} = (\text{row $j$ of $A$}) \cdot (\text{column $k$ of $B$})$. This also means $AB$ is only defined if $A$ is $m \times n$ and $B$ is $n \times r$. 

Let $A$ be an $m \times n$ matrix and let $B$ and $C$ have sizes for which the indicated sums and products are defined. Then:
\begin{enumerate}
	\item $A(BC) = (AB)C$
	\item $A(B + C) = AB + AC$
	\item $(B + C)A = BA + CA $
	\item $\alpha (AB) = (\alpha A)B = A(\alpha B)$
	\item $I_m A = A = AI_n$
\end{enumerate}
Warnings: 
\begin{enumerate}
	\item In general, $AB \neq BA$ 
	\item If $AB = AC$, then it is \textbf{not true} in general that $B=C$
	\item If $AB = 0$, then it is \textbf{not true} always that $A=0$ or $B=0$
\end{enumerate}
\end{definition}

\begin{definition}
The \textbf{transpose} of $A$ is the matrix whose columns are formed from the corresponding rows of $A$, denoted by $A^T$. 
\begin{enumerate}
	\item $(A^T)^T = A$
	\item $(A + B)^T = A^T + B^T$
	\item $(cA)^T = cA^T$
	\item $(AB)^T = B^T A^T$
\end{enumerate}

To understand the final property, let $AB$ denote a $n \times m$ matrix so that 
$$
\begin{aligned}
AB &= \begin{bmatrix}
A_{1*} \cdot B_{1} & A_{1*} \cdot B_{2} & \cdots & A_{1*} \cdot B_{m} \\
A_{2*} \cdot B_{1} & A_{2*} \cdot B_{2} & \cdots & A_{2*} \cdot B_{m} \\
\vdots & \vdots &  & \vdots \\
A_{n*} \cdot B_{1} & A_{n*} \cdot B_{2} & \cdots & A_{n*} \cdot B_{m} \\
\end{bmatrix} \\
(AB)^{T} &= \begin{bmatrix}
A_{1*} \cdot B_{1} & A_{2*} \cdot B_{1} & \cdots & A_{n*} \cdot B_{1} \\
A_{1*} \cdot B_{2} & A_{2*} \cdot B_{2} & \cdots & A_{n*} \cdot B_{2} \\
\vdots & \vdots &  & \vdots \\
A_{1*} \cdot B_{m} & A_{2*} \cdot B_{m} & \cdots & A_{n*} \cdot B_{m} \\

\end{bmatrix}
\end{aligned}
$$

where $A_{\alpha *}$ denotes the $\alpha$th row of a $A$. Note that $(AB)^{T}_{jk} = (\text{row $k$ of $A$}) \cdot (\text{column $j$ of $B$}) = (\text{row $j$ of $B^{T}$}) \cdot (\text{column $k$ of $A^{T}$}) $. 

\end{definition}

\begin{definition}
For a \textit{square} matrix $A$, its \textbf{trace} is the sum of its diagonal entries. 
$$trace(A) = \sum_{k=1}^{n} a_{k, k}$$
\end{definition}

\begin{theorem}
Let $A$ and $B$ be sizes $m \times n$ and $n \times m$, respectively. Then $$trace(AB) = trace(BA)$$
\end{theorem}

\begin{proof}
We need only show that the diagonal entries of $AB$ are the same as the diagonal entries of $BA$. 

$$tr(AB) = \sum_{i=1}^{m} (AB)_{ii} = \sum_{i=1}^{m} \sum_{j=1}^{n} A_{ij} B_{ji} = \sum_{j=1}^{n} \sum_{i=1}^{m} B_{ji} A_{ij} = \sum_{j=1}^{n} (BA)_{jj} = tr(BA)$$
\end{proof}

\subsection{Linear Transformations as a Vector Space}

Let's further abstract the notion of a linear transformation by considering the collection of \textit{all} linear transformations from $V$ to $W$, denoted $L(V, W)$. 

For any linear transformation $T \in L(V, W)$, we can define a new transformation $\alpha T$. We can prove this transformation is linear: 
$$
\begin{aligned} 
(\alpha T)(\alpha_{1} v_{1} + \alpha_{2} v_{2}) &= \alpha (T(\alpha_{1} v_{1} + \alpha_{2} v_{2})) \\
&= \alpha (\alpha_{1}Tv_{1} + \alpha_{2}Tv_{2}) \\
&= \alpha_{1} (\alpha T) v_{1} + \alpha_{2} (\alpha T) v_{2}
\end{aligned}
$$

where the second step follows from the linearity of $T$. 

A similar proof can be made to show that the sum of any $T_{1}, T_{2} \in L(V, W)$ is also linear, which means it's in the set $L(V, W)$. This means we have defined multiplication by a scalar and addition on $L(V, W)$, which means $L(V, W)$ is a vector space. 

\section{Invertible Transformations and Isomorphisms}
\begin{definition}
An $n \times n$ matrix $A$ is \textbf{invertible} if there is an $n \times n$ matrix $A^{-1}$ such that $A^{-1}A = I$ and $AA^{-1} = I$. 

An $n \times n$ matrix $A$ is \textit{left invertible} if there is matrix $B$ such that $BA = I$ and is \textit{right invertible} if there is a matrix $C$ such that $AC = I$. If $A$ is both left and right invertible, then $A$ is called \textbf{invertible}. 
\end{definition}

\begin{theorem}
If $A$ and $B$ are invertible (and $AB$ is defined), then the product $AB$ is invertible and 
$$(AB)^{-1} = B^{-1} A^{-1}$$
\end{theorem}

\begin{proof}
Direct computation:
$$(AB)(B^{-1}A^{-1}) = A(BB^{-1})A^{-1} = AA^{-1} = I$$
and 
$$(B^{-1}A^{-1})(AB) = B^{-1}(A^{-1}A)B = B^{-1}B = I$$
\end{proof}

\begin{theorem}
If $A$ is invertible, then $A^{T}$ is also invertible and 
$$(A^{-1})^{T} = (A^{T})^{-1}$$
\end{theorem}

\begin{proof}
Using $(AB)^{T} = B^{T} A^{T}$, 
$$(A^{-1})^{T} (A^{T}) = (AA^{-1})^{T} = I$$
and 
$$A^{T}(A^{-1})^{T} = (A^{-1}A)^{T} = I$$
\end{proof}

\begin{definition}
An invertible linear transformation $A: V \rightarrow W$ is called an \textbf{isomorphism}. The two vector spaces $V$ and $W$ for which $A$ is defined are called \textbf{isomorphic}, denoted $V \cong W$. 
\end{definition}

Isomorphic spaces can be understood as different representations of the \textit{same} space. To see this, 
\begin{theorem}
Let $A : V \rightarrow W$ be an isomorphism, and let $v_{1}, \cdots, v_{n}$ be a basis in $V$. Then $Av_{1}, \cdots, Av_{n}$ is a basis in $W$. 
\end{theorem}

\begin{proof}
Because $V$ and $W$ are isomorphic, every $w \in W$ can be represented as some $v \in V$ by applying $A^{-1}$. For arbitrary $w \in W$
$$A^{-1}w = v = \sum_{k=1}^{n} \alpha_{k} v_{k}$$
Then we apply $A$ to get 
$$Av = w = \sum_{k=1}^{n} \alpha_{k} Av_{k}$$
\end{proof}
\chapter{Systems of Linear Equations}
\section{Representations of Linear Systems}
One understanding of a \textit{linear system} is simply a collection of $m$ linear equations with $n$ unknowns $x_{1}, \cdots, x_{n}$. Solving this system entails finding all $n$-tuples of numbers $x_{1}, \cdots, x_{n}$ which satisfy the $m$ equations simultaneously. If we define 
$$A = \begin{bmatrix}
\alpha_{11} & \alpha_{12} & \cdots & \alpha_{1n} \\
\alpha_{21} & \alpha_{22} & \cdots & \alpha_{2n} \\
\vdots & \vdots & \ddots & \vdots \\
\alpha_{m1} & \alpha_{m2} & \cdots & \alpha_{mn}
\end{bmatrix}$$
then we can summarize our linear system in matrix form
$$Ax = b$$

The above is the \textbf{coefficient matrix}. If we want to contain all the information in a single matrix, we can use an \textbf{augmented matrix}
$$\begin{bmatrix}[cccc|c]
\alpha_{11} & \alpha_{12} & \cdots & \alpha_{1n} & b_{1}\\
\alpha_{21} & \alpha_{22} & \cdots & \alpha_{2n} & b_{2}\\
\vdots & \vdots & \ddots & \vdots & \vdots\\
\alpha_{m1} & \alpha_{m2} & \cdots & \alpha_{mn} & b_{m}
\end{bmatrix}$$

\section{Solving Linear Systems}
Linear systems are solved using \textbf{Gaussian elimination}. We can perform the following row operation on an augmented matrix:
\begin{enumerate}
	\item (Replace) Replace one row by the sum of itself and a multiple of another row. 
	\item (Interchange) Interchange two rows. 
	\item (Scale) Multiply all entries in a row by a nonzero constant. 
\end{enumerate}

These operations belong to the \textit{elementary matrices}: any operation can be described by applying the same operation to $I$ to get $E$ and then multiplying $EA$.

\begin{definition}
For an augmented matrix 
$$\begin{bmatrix}[ccc|c]
1 & 2 & 3 & 1 \\
3 & 2 & 1 & 7 \\
2 & 1 & 2 & 1
\end{bmatrix}$$
the \textbf{echelon form} is 
$$\begin{bmatrix}[ccc|c]
1 & 2 & 3 & 1 \\
0 & 1 & 2 & -1 \\
0 & 0 & 2 & -4
\end{bmatrix}$$
and the \textbf{reduced echelon form} is 
$$\begin{bmatrix}[ccc|c]
1 & 0 & 0 & 1 \\
0 & 1 & 0 & 3 \\
0 & 0 & 1 & -2
\end{bmatrix}$$

Formally, the \textbf{echelon form} requires that all zero rows are below all nonzero rows and that any nonzero row's \textbf{pivot} (its leading entry) is strictly to the right of the leading entry in the previous row. The particular echelon form above is called \textbf{triangular form} and is only possible when we have a square matrix. The \textbf{reduced echelon form} requires echelon form in addition to maintaining that all pivot entries are 1 and that all entries above each pivot are 0. 
\end{definition}

The \textit{existence} and \textit{uniqueness} of a solution can be determined by analyzing pivots in the echelon form of a matrix. When looking at the coefficient matrix: 
\begin{enumerate}
	\item A solution (if it exists) is unique if and only if there are no free variables, that is if the echelon form has a pivot in every \textit{column}. 
	\item A solution is consistent if and only if the echelon form has a pivot in every \textit{row}. 
\end{enumerate}
The first statement is trivial because free variables are responsible for all non-uniqueness. For the second statement, if we have a row with no pivots in the echelon form of a matrix, we have $\begin{bmatrix}[ccc|c]
0 & \cdots & 0 & b_{k}
\end{bmatrix}$, which certainly has no solution. Thus, in order for a solution to \textit{exist} and be \textit{unique}, the echelon form must have a pivot in \textit{every column and every row}. 

\begin{theorem}
Any linearly independent system of vectors in $\mathbb{F}^{n}$ cannot have more than $n$ vectors in it. 
\end{theorem}

\begin{proof}
Let a system $v_{1}, \cdots, v_{m} \in \mathbb{F}^{n}$ be linearly independent and let $A = \begin{bmatrix}
v_{1} & \cdots & v_{m}
\end{bmatrix}$ be $n \times m$. We must show that $x_{1} v_{1} + \cdots + x_{m} v_{m}= 0$, or equivalently $Ax = 0$, has unique solution $x = 0$. According to statement 1 above, a solution can only be unique if the echelon form has a pivot in every column. This is impossible if $m > n$.  
\end{proof}

\begin{theorem}
A matrix $A$ is invertible if and only if its echelon form has pivot in every column and every row. 
\end{theorem}

\begin{proof}
Since a matrix must have unique solution for $Ax = b$ for any $b$ in order to be invertible, it is necessary that the echelon form has pivot in every column and row, according to statements 1 and 2 above. 
\end{proof}
This directly implies that an invertible matrix \textbf{must be square}.

Since an invertible matrix must be square and must have pivots in every row and column in echelon form, any invertible matrix is row equivalent to the identity matrix. We can use this to get the following algorithm for finding $A^{-1}$:

\begin{enumerate}
	\item Form an \textit{augmented $n \times 2n$} matrix 
	$\begin{bmatrix}[c|c] A & I\end{bmatrix}$.
	\item Perform row operations to transform $A$ into $I$.
	\item The matrix that was originally $I$ will now be $A^{-1}$.
\end{enumerate}

To fully understand this algorithm, remember that every row operation can be expressed as the left multiplication by an elementary matrix. Let $E = E_{n} \cdots E_{2} E_{1}$ represent all the performed row operations. Since we know $E$ transforms $A$ to the identity matrix, we have $EA = I$, so $E = A^{-1}$. Since row operations affect the entire augmented matrix, we have $\begin{bmatrix}[c|c]
A & I
\end{bmatrix} \rightarrow \begin{bmatrix}[c|c]
EA & EI
\end{bmatrix} = \begin{bmatrix}[c|c]
I & A^{-1}
\end{bmatrix}$. 

\section{Fundamental Subspaces}

\begin{definition}
A \textbf{subspace} of vector space $V$ is a non-empty subset $V_{0} \subset V$ which is also a vector space. Subspaces must be non-empty because all vector spaces must contain the zero vector. 
\end{definition}

For any linear transformation $A : V \rightarrow W$, we can associate the following subspaces: 
\begin{enumerate}
	\item The \textit{null space}, or \textit{kernel}, of $A$ which consists of all vectors $v \in V$ such that $Av = 0$. 
	\item The \textit{range} of $A$ which is the set of all vectors $w \in W$ which can be represented as $w = Av$ for $v \in V$. 
\end{enumerate}

By the \textit{column by coordinate rule}, we know that any vector in $\Range(A)$ can be represented as a weighted sum of the column vectors of $A$, which is why the term \textit{column space} is sometimes used to refer to range. 

In addition, we can consider the corresponding subspaces of the transposed matrix. The term \textit{row space} is used to denote $\Range(A^{T})$, and the term \textit{left null space} is used to denote $\Null(A^{T})$. Together, these four subspaces are known as the \textbf{fundamental subspaces} of the matrix $A$. 

\begin{definition}
The \textbf{dimension} of a vector space $V$, denoted $\Dim(V)$, is the number of vectors in a basis. 
\end{definition}

\begin{theorem}[General solution of a linear equation] 
Let a vector $x_{1}$ denote a solution to the equation $Ax = b$, and let $H$ be the set of all solutions of $Ax = 0$. Then the set
$$x = x_{1} + x_{h} : x_{h} \in H$$
is the set of \textit{all solutions} of the equation $Ax = b$. 

In other words, 
$$\Big(\text{General solution of $Ax=b$}\Big) = \Big(\text{A particular solution of $Ax=b$}\Big) + \Big(\text{General solution of $Ax=0$}\Big)$$
\end{theorem}
\begin{proof}
We know $Ax_{1} = b$ and $Ax_{h} = 0$. For $x = x_{1} + x_{h}$,
$$Ax = A(x_{1} + x_{h}) = Ax_{1} + Ax_{h} = b + 0 = b$$

Therefore, any solution $x$ for $Ax = b$ can be represented as $x = x_{1} + x_{h}$ with some $x_{h} \in H$. 
\end{proof}

The power of this theorem is its generality -- it applies to all linear equations. Aside from showing the structure of the solution set, this theorem allows us to separate investigations of uniqueness from existence. To study uniqueness of a solution, we only need to analyze uniqueness of $Ax = 0$, which always has a solution. 

\begin{theorem}
In order to compute the fundamental subspaces, we need to do row reduction. Let $A$ be the original matrix and let $A_{e}$ be its echelon form. 
\begin{enumerate}
	\item The pivot \textit{columns} of the \textit{original matrix} $A$ (the columns where after row operations we will have pivots in echelon form) give us a basis for $\Range(A)$. 
	\item The pivot \textit{rows} of $A_{e}$ give us the basis in row space. 
	\item To find $\Null(A)$, we need to solve $Ax = 0$. 
\end{enumerate}
\end{theorem}

\begin{proof}
In turn, 
\begin{enumerate}
	\item We know the pivot columns of $A_{e}$ form a basis for $\Range(A_{e})$. Since $A_{e} = EA$ ($E$ is the matrix product of the elementary matrices representing the row operations completed), $A = E^{-1}A_{e}$. This means the corresponding columns in $A$ of $A_{e}$ is a basis of $A$. 
	
	\item We know that the pivot rows of the echelon form are linearly independent. Now we need only prove that they span the entirety of the row space. Notice that \textit{row operations do not change the row space}. To prove this, 
	$$A_{e} = EA$$
	where $A$ is $m \times n$ and $E$ is an $m \times m$ invertible matrix. 
	$$\Range(A_{e}^{T}) = \Range(A^{T} E^{T}) = A^{T} (\Range(E^{T})) = A^{T} (\mathbb{R}^{m}) = \Range(A^{T})$$
	where the final step follows from applying an $n \times m$ matrix to $\mathbb{R}^{m}$, which is just a transformation from $\mathbb{R}^{m}$ to $\Range(A^{T})$. 
	
	\item Solving for $Ax = 0$ certainly gives us a spanning set for  $\Null(A)$. To prove the set is linearly independent, multiply each vector by its corresponding free variable and add. For every free variable $x_{k}$, the entry $k$ is exactly $x_{k}$, so the only way the sum of the set is $0$ is if all the free variables are $0$. 
\end{enumerate}
\end{proof}

As an example of these computations, consider the matrix 
$$\begin{bmatrix}
1 & 1 & 2 & 2 & 1 \\
2 & 2 & 1 & 1 & 1 \\
3 & 3 & 3 & 3 & 2 \\
1 & 1 & -1 & -1 & 0
\end{bmatrix}$$

Performing row operations, we get the echelon form 
$$\begin{bmatrix}
1 & 1 & 2 & 2 & 1 \\
0 & 0 & -3 & -3 & -1 \\
0 & 0 & 0 & 0 & 0 \\
0 & 0 & 0 & 0 & 0
\end{bmatrix}$$

So the first and third columns of the \textit{original matrix} give us a basis for $\Range(A)$: 
$$\begin{bmatrix}
1 \\ 
2 \\
3 \\ 
1
\end{bmatrix}\ ,\  \begin{bmatrix}
2 \\ 
1 \\
3 \\
-1
\end{bmatrix}$$

We also know the basis for the row space of $A$ is the first and second row of the \textit{echelon form}:
$$\begin{bmatrix}
1 \\
1 \\
2 \\
2 \\
1
\end{bmatrix} \ , \ \begin{bmatrix}
0 \\
0 \\ 
-3 \\ 
-3 \\ 
-1
\end{bmatrix}$$

To find $\Null(A)$ we solve $Ax = 0$. The reduced echelon form is 
$$\begin{bmatrix}
1 & 1 & 0 & 0 & \frac{1}{3} \\
0 & 0 & 1 & 1 & \frac{1}{3} \\
0 & 0 & 0 & 0 & 0 \\
0 & 0 & 0 & 0 & 0
\end{bmatrix}$$
This means 
$$ \left\{ \begin{matrix}
x_{1} = -x_{2} - \frac{1}{3} x_{5} \\
x_{2} \text{ is free} \\
x_{3} = -x_{4} - \frac{1}{3} x_{5} \\
x_{4} \text{ is free} \\
x_{5} \text{ is free}
\end{matrix} \right\} \longrightarrow x_{2} \begin{bmatrix}
-1 \\
1 \\ 
0 \\
0 \\
0
\end{bmatrix} + x_{4} \begin{bmatrix}
0 \\ 
0 \\
-1 \\
1 \\
0
\end{bmatrix} + x_{5} \begin{bmatrix}
-\frac{1}{3} \\
0 \\
-\frac{1}{3} \\
0 \\
1
\end{bmatrix}$$

The vectors at each free variables form the basis for $\Null(A)$. 

\begin{definition}
The \textbf{rank} of a linear transformation $A$, denoted $\Rank(A)$, is the dimension of the range of $A$. 
$$\Rank(A) := \Dim(\Range(A))$$
\end{definition}

\begin{theorem}[The Rank Theorem]
For a matrix $A$ 
$$\Rank(A) = \Rank(A^{T})$$

The proof of this is trivial since rank of both column space and row space are dependent on the number of pivots in echelon form. 
\end{theorem}

\begin{theorem}[Rank-Nullity Theorem]
Let $A$ be an $m \times n$ matrix. Then 
\begin{enumerate}
	\item $\Dim(\Null(A)) + \Dim(\Range(A)) = n$ (dim of domain)
	\item $\Dim(\Null(A^{T})) + \Dim(\Range(A^{T})) = \Dim(\Null(A^{T})) + \Rank(A) = m$ (dim of codomain)
\end{enumerate}
\end{theorem}

\begin{proof}
In turn, 
\begin{enumerate}
	\item The first equality is simply that the number of free variables + the number of pivots = the number of columns. 
	\item The second equality applies the Rank Theorem to prove the row counterpart to the first equality. 
\end{enumerate}
\end{proof}

The following follows from the second statement in the above theorem. 
\begin{theorem}
Let $A$ be an $m \times n$ matrix. Then the equation 
$$Ax = b$$ 
has a solution for \textit{every} $b \in \mathbb{R}^{m}$ if and only if the \textit{dual} equation 
$$A^{T}x = 0$$ 
has only the trivial solution. 
\end{theorem}

\section{Change of Basis}
Let $V$ be a vector space with a basis $B := b_{1}, \cdots, b_{n}$. Recall that any vector $v \in V$ can be written
$$v = x_{1} b_{1} + \cdots + x_{n} b_{n}$$
where the numbers $x_{1}, \cdots, x_{n}$ are called the coordinates of $v$. We can write the \textit{coordinate vector} as 
$$[v]_{B} := \begin{bmatrix}
x_{1} \\
\vdots \\
x_{n}
\end{bmatrix} \in \mathbb{F}^{n}$$

Note that $v \rightarrow [v]_{B}$ is an isomorphism between $V$ and $\mathbb{F}^{n}$.

\begin{definition}
Let $T: V \rightarrow W$ be a linear transformation, and let $A = \{a_{1}, \cdots, a_{n}\}$, $B = \{ b_{1}, \cdots, b_{m} \}$ be bases in $V,W$ respectively. 

A \textbf{matrix of transformation $T$ in bases $A$ and $B$} is an $m \times n$ matrix, denoted by $[T]_{BA}$, 
$$[Tv]_{B} = [T]_{BA} [v]_{A}$$

The matrix $[T]_{BA}$ is easy - its $k$th column is just $[Ta_{k}]_{B}$. 
\end{definition}

\begin{definition}
For the above two bases $A$ and $B$, the \textbf{change of basis} is 
$$[v]_{B} = [I]_{BA} v_{A}$$
where $[I]_{BA}$ is the \textbf{change of basis matrix} whose $k$th column is $[a_{k}]_{B}$. 

Clearly, any change of basis is invertible and 
$$[I]_{BA} = ([I]_{AB})^{-1}$$
\end{definition}

\begin{definition}
We can use this to define \textbf{similar matrices} as matrices $A, B$ such that 
$$A = Q^{-1}BQ$$

This means we can treat similar matrices as different representations of the same linear operator. 
\end{definition}
\chapter{Determinants} 
\section{Properties}
In this chapter, we only consider determinants of $n \times n$ matrices. We will think of the determinant as the $n$-dimensional volume of the parallelepiped determined by our $n$ vectors, $v_{1}, \cdots, v_{n}$. For dimensions 2 and 3, ``volume" of the parallelepiped is determined with the \textit{base times height} rule: we pick one vector and define height to be the distance between this vector and the subspace spanned by the $n - 1$ remaining vectors. Then we define base to be the $(n-1)$-dimensional volume of the parallelepiped determined by the $n-1$ vectors. 

This understanding allows determinants the following properties: 
\begin{enumerate}
	\item \textbf{Linearity in each argument: } Multiplying some vector $v_{k}$ by $\alpha$ means the height is multiplied by $\alpha$ which means the determinant is multiplied by the same constant. Since the determinant is \textit{linear in each argument}, if we fix $n-1$ vectors the determinant is linear with respect to the final vector. 
	
	Linearity means that for an $n \times n$ matrix $A$,  $det(\alpha A) = \alpha^{n} det(A)$, because multiplying $A$ by $\alpha$ is equivalent to multiplying $n$ columns by $\alpha$. 
	\item \textbf{Preservation under column replacement: } 
	$$det(v_{1}, \cdots, v_{j} + \alpha v_{k}, \cdots, v_{k}, \cdots, v_{n}) = det(v_{1}, \cdots, v_{j}, \cdots, v_{k}, \cdots, v_{n})$$
	This is true because the ``height" of $v_{j} + \alpha v_{k}$ is the same as the "height" of $v_{j}$, since "height" is defined in relation to the distance from the remaining subspace. 
	\item \textbf{Antisymmetry: } Swapping two vectors means the determinant changes signs. 
	$$det(v_{1}, \cdots, v_{k}, \cdots, v_{j}, \cdots, v_{n}) = -det(v_{1}, \cdots, v_{j}, \cdots, v_{k}, \cdots, v_{n})$$
	
This does not seem natural at first, but we can prove it by applying preservation under column replacement thrice and then linearity. 
$$ 
\begin{aligned}
&det(v_{1}, \cdots, v_{k}, \cdots, v_{j}, \cdots, v_{n}) \\ 
&= det(v_{1}, \cdots, v_{k}, \cdots, v_{j} - v_{k}, \cdots, v_{n}) \\
&= det(v_{1}, \cdots, v_{k} + (v_{j} - v_{k}), \cdots, v_{j} - v_{k}, \cdots, v_{n}) \\
&= det(v_{1}, \cdots, v_{j}, \cdots, v_{j} - v_{k} - (v_{j}), \cdots, v_{n}) \\
&= det(v_{1}, \cdots, v_{j}, \cdots, -v_{k}, \cdots, v_{n}) \\
&= -det(v_{1}, \cdots, v_{j}, \cdots, v_{k}, \cdots, v_{n}) 
\end{aligned}
$$

	\item \textbf{Normalization: } For the standard basis, the corresponding parallelepiped is the $n$-dimensional unit cube so its volume is 1. 
	$$det(I) = 1$$
\end{enumerate}

Using these, we can derive additional basic properties of determinants for a square matrix $A$:
\begin{enumerate}
	\item If $A$ has a zero column, then $det(A) = 0$. 
	\item If $A$ has two equal columns, then $det(A) = 0$. 
	\item If one column of $A$ is a multiple of another, then $det(A) = 0$. 
\end{enumerate}

\section{Computing the Determinant}
The \textbf{determinant of diagonal matrices} is the product of the diagonal entries because any diagonal matrix $\{a_{1}, \cdots, a_{k}\}$ can be obtained by multiplying column $k$ of the identity matrix by $a_{k}$. 

The \textbf{determinant of triangular matrices} is also the product of the diagonal entries. This is because an upper or lower triangular matrix can be reduced to a diagonal matrix with the same diagonal entries through column operations. 

\begin{theorem}
$det(A) = 0$ if and only if $A$ is not invertible. 
\end{theorem}

\begin{proof}
Recall that we can only use column operations when reducing a matrix to find the determinant, which is equivalent to doing row operations on $A^{T}$. If the echelon form of $A^{T}$ does not have pivots in every column and row, then the product of diagonal entries will be 0. Not having pivots in every column and row also means the matrix is not invertible, so the two conclusions are equivalent. 
\end{proof}

We will now prove some nontrivial properties of determinants, but to do so we will need the following two lemmas. 

\begin{lemma}
For a square matrix $A$ and elementary matrix $E$, 
$$det(AE) = det(A) det(E)$$
\end{lemma}

\begin{lproof}
Right multiplication of an elementary matrix is simply a column operation. Since a column operation is obtained from the identity matrix by the column operation, its determinant is 1 times the effect of the column operation. 
\end{lproof}

\begin{lemma}
Any invertible matrix is a product of elementary matrices.
\end{lemma}

\begin{lproof}
We know that any invertible matrix is \textit{row equivalent} to the identity matrix, which is its reduced echelon form. So 
$$I = E_{n} E_{n-1} \cdots E_{1} A$$
which means we can write $A$ in terms of the identity and the inverses of some elementary  matrices 
$$A = E_{1}^{-1} \cdots E_{n-1}^{-1} E_{n}^{-1} I = E_{1}^{-1} \cdots E_{n-1}^{-1} E_{n}^{-1}$$
Since the inverse of an elementary matrix is an elementary matrix, the proof is complete. 
\end{lproof} 

Now for two important theorems:

\begin{theorem}
For a square matrix $A$, 
$$det(A) = det(A^{T})$$
\end{theorem}

\begin{proof}
A key observation is that $det(E) = det(E^{T})$ for any elementary matrix $E$. 

Notice also that it is sufficient to prove the theorem \textit{only} for \textit{invertible matrices} since if $A$ is not invertible then $A^{T}$ is also not invertible and both determinants are 0, trivially proving the theorem. 

Now, by the above lemma we can write
$$A = E_{1} E_{2} \cdots E_{n}$$
which means 
$$det(A) = det(E_{1}) det(E_{2}) \cdots det(E_{n})$$

We can also write 
$$A^{T} = E_{n}^{T} \cdots E_{2}^{T} E_{1}^{T} = E_{n} \cdots E_{2} E_{1}$$
which means 
$$det(A^{T}) = det(E_{n}) \cdots det(E_{2}) det(E_{1})$$
which is equivalent to $det(A)$. 
\end{proof}

This theorem means that column operations have the same effect on determinants as row operations, so we can use either when reducing matrices to compute determinants. 

\begin{theorem}
For $n \times n$ matrices $A, B$, 
$$det(AB) = det(A) det(B)$$
\end{theorem}

\begin{proof}
Two cases: 

\textbf{Case 1: } $B$ is invertible. 

This means we can write 
$$B = E_{1} E_{2} \cdots E_{n}$$
and so 
$$det(AB) = det(A)[det(E_{1}) det(E_{2}) \cdots det(E_{n})] = det(A) det(B)$$

\textbf{Case 2: } $B$ is not invertible. 
If $B$ is not invertible, we will prove that the product $AB$ is also not invertible so $det(AB) = det(A) det(B)$ simplifies to $0 = 0$. 

We proceed by contradiction. Assume $AB = C$ is invertible. Then we left multiply both sides by $C^{-1}$ to get $C^{-1} AB = I$, which means $C^{-1}A$ is the left inverse of $B$, but because $B$ is square, $C^{-1}A$ is the inverse of $B$. We know $B$ is not invertible so we have a contradiction. 
\end{proof}

\section{Cofactor Expansion}

For an $n \times n$ matrix $A$, let $A_{j, k}$ denote the $(n - 1) \times (n - 1)$ matrix obtained from $A$ by crossing out row $j$ and column $k$. 

\begin{theorem}[Cofactor expansion of determinant]
For each $j, 1 \leq j \leq n,$ the determinant of $A$ can be expanded in the row number $j$ as 
$$det(A) = a_{j, 1} (-1)^{j+1} det(A_{j, 1}) + a_{j, 2} (-1)^{j+2} det(A_{j, 2}) + \cdots + a_{j, n} (-1)^{j + n} det(A_{j, n})$$

A similar expansion can be done for columns. 
\end{theorem}

\begin{proof}
We will prove the expansion for row 1. This can be generalized by swapping row 1 with another row. Additionally, since $det(A) = det(A^{T})$, column expansion follows automatically. 

Consider the special case when the first row has \textit{only one} nonzero term, $a_{1, 1}$. Performing column operations on columns $2, \cdots, n$, we transform $A$ to lower triangular form. Now
$$det(A) = (\text{product of diagonal entries}) \times (\text{correcting factor from column operations})$$
but since the \textit{product of diagonal entries} except $a_{1,1}$ times the \textit{correcting factor} is exactly $det(A_{1,1})$, we can write 
$$det(A) = a_{1,1} det(A_{1,1})$$

Now consider the case when all entries in the first row except $a_{1,2}$ are zeros. We can reduce this to the previous case by swapping columns 1 and 2, so $det(A) = (-1) a_{1,2} det(A_{1,2})$. 

If $a_{1,3}$ is the only nonzero term in the first row, we can reduce this to the previous case by swapping columns 2 and 3, so $det(A) = a_{1,3} det(A_{1,3})$. We do this instead of swapping columns 1 and 3 to maintain the order of the $n-1$ other columns. 

These special cases are important because we have linearity of the determinant. If the matrix $A^{(k)}$ is obtained by replacing all $A$'s entries in the first row with 0 except for $a_{1,k}$, then linearity of the determinant implies 
$$det(A) = det(A^{(1)}) + \cdots + det(A^{(n)}) = \sum_{k=1}^{n} det(A^{(k)})$$

Based on our analysis of special cases, we know 
$$det(A^{(k)}) = (-1)^{1 + k} a_{1, k} det(A_{1, k})$$ 
so 
$$det(A) = \sum_{k=1}^{n} (-1)^{1 + k} a_{1, k} det(A_{1,k})$$

To get the expansion for the second row, we swap rows so multiply by -1. For the third row, multiply by -1 again to get the original equation, and so on. 
\end{proof}

Cofactor expansion is not practical for anything larger than a $3 \times 3$ matrix, but it has great theoretical importance. 

\begin{definition}
Formally, the numbers 
$$C_{j, k}= (-1)^{j + k} det(A_{j,k})$$
are called \textbf{cofactors}.

The matrix $C = \{ C_{j, k}\}^{n}_{j, k = 1}$ whose entries are \textit{cofactors} of a given matrix $A$ is called the \textbf{cofactor matrix} of $A$. 
\end{definition}

\begin{theorem}[Cofactor formula for inverse]
Let $A$ be an invertible matrix and let $C$ be its cofactor matrix. Then 
$$A^{-1} = \frac{C^{T}}{det(A)}$$
\end{theorem}

\begin{proof}
Let us find the product $AC^{T}$. 

The $j$th diagonal entry is obtained by multiplying the $j$th row of $A$ by the $j$th row of $C$,
$$(AC^{T})_{j, j} = a_{j, 1} C_{j, 1} + \cdots + a_{j, n} C_{j, n} = det(A)$$
by cofactor expansion. 

To get the off-diagonal terms, we multiply the $k$th row of $A$ with the $j$th row of $C$, $j \neq k$, 
$$a_{k, 1} C_{j, 1} + \cdots + a_{k, n} C_{j, n}$$

If we look at this as a cofactor expansion of the $j$th row, this is the determinant of the matrix $A$ except that we replace row $j$ with row $k$. Since two rows of our matrix coincide, the determinant will be 0, which means all off-diagonal terms will be 0, thus 
$$AC^{T} = det(A) I$$
\end{proof}

Since for invertible matrices, $Ax = b$ has a unique solution, we have 
$$x = A^{-1} b = \frac{C^{T}b}{det(A)}$$

\begin{theorem}[Cramer's Rule]
For invertible matrix $A$, entry $k$ of the solution to $Ax = b$ is given by 
$$x_{k} = \frac{det(B_{k})}{det(A)} $$
where $B_{k}$ is obtained from $A$ by replacing column $k$ with $b$. 
\end{theorem}

\begin{proof}
After our above theorem, we need only prove that entry $k$ of $C^{T}b = det(B_{k})$. 

We know entry $k$ of $C^{T} b$ is equivalent to the product of the $k$th row of $C^{T}$ and $b$, which is equivalent to the product of the $k$th column of $C$ and $b$. 

$C_{j,k}$ is obtained by crossing out the $j$th row and $k$th column of $A$ and computing the determinant of the remaining matrix. Multiplying the $k$th column of $C$ with $b$ is equivalent to 
$$b_{1} C_{1, k} + \cdots + b_{n} C_{n, k}$$ 
which is the same as the cofactor expansion of $B_{k}$. 
\end{proof}

One application of the cofactor formula is a shortcut to inverting $2 \times 2$ matrices. 
For the matrix 
$$A = \begin{bmatrix}
a & b \\
c & d
\end{bmatrix}$$
The cofactor matrix is made up of 4 individual $1 \times 1$ matrices,
$$C = \begin{bmatrix}
d & -c \\
-b & a
\end{bmatrix}$$
which means 
$$A^{-1} = \frac{1}{det(A)} \begin{bmatrix}
d & -c \\
-b & a
\end{bmatrix}$$

%%SKIPPED THE LAST COROLLARY BECAUSE IT DOESN'T SEEM IMPORTANT AND DOESN'T MAKE SENSE (HOW CAN RANK BE DEFINED FOR A POINT?)
%\section{Minors and Rank}
%\begin{definition}
%For a matrix $A$, consider a $k \times k$ submatrix obtained by taking $k$ rows and $k$ columns. The determinant of this submatrix is called the \textbf{minor of order $k$}. An $m \times n$ matrix has ${m \choose k} \cdot {n \choose k}$ different submatrices with different minors of order $k$. 
%\end{definition}

%\begin{theorem}
%For a nonzero matrix $A$, its rank equals the maximum value of $k$ such that there exists a nonzero minor of rank $k$. 
%\end{theorem}

%\begin{proof}
%First, we prove that if $k > Rank(A)$, then a minor of order $k = 0$. Since $Rank(A) < k$, any submatrix with $k$ columns will be linearly dependent, so its determinant will be 0. 

%Now we just need to show that there exists a nonzero minor of order $k = Rank(A)$. One simple construction of a submatrix that fulfills this criterion is the submatrix made up of pivot rows and columns of $A$. Since it is made up of only pivot rows and columns, it has nonzero determinant and same rank as $A$. 
%\end{proof}

%This theorem doesn't seem very useful at first, but it is important because it can be used to prove the following. 
\chapter{Spectral Theory}
Spectral theory will be our main tool for analyzing linear operators. In this chapter, we only consider transformations $A: V \rightarrow V$ ($n \times n$ matrices). 

\section{Definitions}

\begin{definition}
A scalar $\lambda$ is called an \textbf{eigenvalue} of operator $A: V \rightarrow V$ if there exists a \textit{nonzero} vector $v \in V$ such that 
$$Av = \lambda v$$

The vector $v$ is called an \textbf{eigenvector} of $A$ (corresponding to the eigenvalue of $\lambda$).

Once we know the eigenvalues, finding the eigenvectors is equivalent to solving 
$$(A - \lambda I)v = 0$$
$Null(A - \lambda I)$, the set of all eigenvectors and 0, is called the \textbf{eigenspace}. 

The set of all eigenvalues of an operator is called the \textbf{spectrum} of $A$, denoted $\sigma (A)$. 
\end{definition}

Since the matrix $A$ is square, $A - \lambda I$ has a nontrivial null space if and only if it is not invertible, which means its determinant will be 0. Thus, for any eigenvalue $\lambda$ of $A$, 
$$det(A - \lambda I) = 0$$

\begin{definition}
If $A$ is an $n \times n$ matrix, $det(A - \lambda I)$ is a degree-$n$ polynomial of variable $\lambda$. This is called the \textbf{characteristic polynomial} of $A$. Finding the spectrum of $A$ requires finding the roots to the characteristic polynomial. 

Using $(\lambda I - A)v = 0$ as the characteristic equation always yields a monic polynomial, whereas our current definition differs by a factor of $(-1)^{n}$. This makes no difference for properties like having eigenvalues located at roots so the two definitions are usually interchangeable. 

This means any operator in $\mathbb{C}^{n}$ has $n$ eigenvalues, though some may be repeated. 
\end{definition}

\begin{theorem}
An $n \times n$ matrix $A$ is invertible if and only if it doesn't have an eigenvalue of 0. 
\end{theorem}

\begin{proof}
Proving if: 

If $A$ doesn't have an eigenvalue of 0, then $det(A - 0I) \neq 0 \rightarrow det(A) \neq 0$, which implies $A$ is invertible. 

Proving only if: 

If $A$ is invertible, then $det(A) \neq 0$, which implies $det(A - 0I) \neq 0$. 
\end{proof}

\begin{theorem}
Let $A$ be an $n \times n$ matrix, and let $\lambda_{1}, \cdots, \lambda_{n}$ be its complex eigenvalues (counting multiplicities). Then
$$det(A) = \lambda_{1} \cdots \lambda_{n}$$
\end{theorem}


\begin{proof}
Since $det(A - \lambda I)$ is a degree-$n$ polynomial of variable $\lambda$ and we know $A$ will have $n$ eigenvalues, we can write 

$$det(\lambda I - A) = (\lambda - \lambda_{1}) \cdots (\lambda - \lambda_{n})$$

Plugging in $\lambda = 0$ gives us 
$$det(-A) = (-1)^{n} det(A) = (- \lambda_{1}) \cdots (- \lambda_{n})$$ 

which simplifies to 
$$det(A) = (\lambda_{1}) \cdots (\lambda_{n})$$
\end{proof}


\begin{theorem}
Let $A$ be an $n \times n$ matrix, and let $\lambda_{1}, \cdots, \lambda_{n}$ be its complex eigenvalues (counting multiplicities). Then
$$trace(A) = \lambda_{1} + \cdots + \lambda_{n}$$ 
\end{theorem}

\begin{proof}
Let us begin by analyzing $det(A - \lambda I)$. Notice that in any cofactor expansion, if we pick any element $a_{i, j}$, such that $j \neq k$, then the highest degree of the resulting cofactor will be $n - 2$. This is because cofactoring removes the row and column the chosen entry is on, and since $j \neq k$, we remove the variables $a_{j,j} - \lambda$ and $a_{k,k} - \lambda$. After cofactor expansion, the  $\lambda^{n-1}$ term will be formed by only this equation
$$(a_{1,1} - \lambda) \cdots (a_{n, n} - \lambda) = (-1)^{n} (\lambda - a_{1,1}) \cdots (\lambda - a_{n,n})$$
so the coefficient of $\lambda^{n-1}$ amounts to choosing the $\lambda$ variable $n -1$ times and choosing one of the other coefficients to get 
\begin{equation}
(-1)^{n}(a_{1, 1} \lambda^{n-1})\cdots (a_{n, n} \lambda^{n-1}) = (-1)^{n} (a_{1, 1} + \cdots + a_{n, n}) \lambda^{n-1}
\end{equation}


Note we can rewrite the characteristic equation as
$$det(A - \lambda I) = (\lambda_{1} - \lambda) \cdots (\lambda_{n} - \lambda) = (-1)^{n} (\lambda - \lambda_{1}) \cdots (\lambda - \lambda_{n})$$

Now let us identify the coefficient of the $\lambda^{n-1}$ term 
\begin{equation}
(-1)^{n} (\lambda_{1} \lambda^{n-1} + \lambda_{2} \lambda^{n-1} + \cdots + \lambda_{n} \lambda^{n-1}) = (-1)^{n} (\lambda_{1} + \cdots + \lambda_{n}) \lambda^{n-1}
\end{equation}

Comparing coefficients in Equations 4.1 and 4.2, $trace(A) = \lambda_{1} + \cdots + \lambda_{n}$. 
\end{proof}

\section{Diagonalization}
We can use spectral theory to find the diagonalization of operators, which means that given an operator, we find the basis in which the matrix of the operator is diagonal. This makes powers of an operator much easier to compute. 

\begin{theorem}
A matrix $A$ in $\mathbb{F}^{n}$ can be written as $A = PDP^{-1}$, where $D$ is a diagonal matrix and $P$ is invertible, if and only if there exists a basis in $\mathbb{F}^{n}$ of eigenvectors of $A$. 

In this case, the diagonal entries of $D$ are the eigenvalues of $A$ and the columns of $P$ are the corresponding eigenvectors. 
\end{theorem}

\begin{proof}

To understand the intuition behind this, note that $P = [I]_{S, B}$, where $S$ is the standard basis and $B$ is the basis for the eigenspace, since each column is the representation of a basis vector written in $S$. Rewriting $A = PDP^{-1}$ as $D = P^{-1}AP = [I]_{B, S} A [I]_{S,B}$ which means $D = [A]_{B, B}$, which is a diagonal operator if and only if its diagonal entries are eigenvalues whose corresponding eigenvectors are $b_{k}$. Think of the operator $[I]_{B, S} A [I]_{S,B}$ as converting a vector to a basis of eigenvectors, scaling those eigenvectors appropriately by their eigenvalues, and then converting back to the standard basis. 

A simpler, more direct proof is to rewrite $AP = PD$. 
$$AP = \begin{bmatrix}
Ab_{1} & \cdots & Ab_{n}
\end{bmatrix} = \begin{bmatrix}
\lambda_{1} b_{1} & \cdots & \lambda_{n} b_{n}
\end{bmatrix}$$

$$PD = \begin{bmatrix}
b_{1} \lambda_{1} & \cdots & b_{1} \lambda_{1}
\end{bmatrix} = \begin{bmatrix}
\lambda_{1} b_{1} & \cdots & \lambda_{n} b_{n}
\end{bmatrix}$$
\end{proof}

Of course, for $P$ to be invertible, the eigenvectors $b_{1}, \cdots, b_{n}$ must be linearly independent. Luckily, we can easily check if this is the case with the following theorem. 

\begin{theorem}
Let $\lambda_{1}, \cdots, \lambda_{n}$ be \textbf{distinct} eigenvalues for $A$, and let $b_{1}, \cdots, b_{n}$ be their corresponding eigenvectors. Then $b_{1}, \cdots, b_{n}$ are linearly independent. 
\end{theorem}

\begin{proof}
We proceed by induction over the $n$ eigenvectors of $A$. 

\textbf{Base case: $n = 1$}

This is trivial because by definition, an eigenvector is nonzero. Any set consisting of a single nonzero vector is linearly independent.

\textbf{Inductive Hypothesis:}

Assume it holds for $n = k$. 

\textbf{Inductive Step: $n = k + 1$}

Suppose there exists a non-trivial solution to 
$$ \sum_{i = 1}^{k + 1} c_{i} b_{i} = 0$$

We can apply $(A - \lambda_{k + 1} I)$ to both sides to get 
$$ \sum_{i = 1}^{k + 1} c_{i} (A - \lambda_{k + 1} I) b_{i} = 0$$

Since $(A - \lambda_{k + 1}I) b_{k + 1} = 0$, we can write 
$$\sum_{i = 1}^{k} c_{i} (A - \lambda_{k + 1} I ) b_{i} = \sum_{i = 1}^{k} c_{i} (\lambda_{i} - \lambda_{k + 1} ) b_{i} = 0$$ 

By the inductive hypothesis, we know the first $k$ eigenvectors are linearly independent, so the coefficient $c_{i} (\lambda_{i} - \lambda_{k + 1} )$ must be 0 for $0 \leq i \leq k$, and since eigenvalues are distinct, $c_{i} = 0$ for $0 \leq i \leq k$. 

Now we can reduce our original summation 
$$\sum_{i = 1}^{k+1} c_{i} b_{i} = c_{k+1} b_{k+1} = 0$$
This means that $c_{k+1}$ must be 0, so the summation only has the trivial solution, which means the eigenvectos are linearly independent. 
\end{proof}
%%% DID NOT DO RIGID MOTIONS AND COMPLEXIFICATION!!
%%% DID NOT DO RIGID MOTIONS AND COMPLEXIFICATION!!
%%% DID NOT DO RIGID MOTIONS AND COMPLEXIFICATION!!

\chapter{Inner Product Spaces}

Keep in mind that theory for inner product space is only developed for $\mathbb{R}$ and $\mathbb{C}$, so $\mathbb{F}$ will always denote one of those two fields in the next two chapters. 

\section{Inner Product}

\begin{definition}
We define the \textbf{norm} of a vector to be the generalization of \textit{length}. That is, the norm of a vector $x \in \mathbb{R}^{n}$ is 
$$\norm{x} = \sqrt{x_{1}^{2} + \cdots + x_{n}^{2}}$$

For any complex number $z = x + iy$, we can write $\abs{z}^{2} = x^{2} + y^{2} = z \overline{z}$, where $\overline{z}$ denotes the complex conjugate of $z$. So for any $z$ in a complex field $\mathbb{C}^{n}$, we can write 
$$z = \begin{bmatrix}
z_{1} \\
\vdots \\
z_{n}
\end{bmatrix} = \begin{bmatrix}
x_{1} + iy_{1} \\
\vdots \\
x_{n} + iy_{n}
\end{bmatrix}$$
so it is natural to define the norm $\norm{z}$ as 
$$\norm{z}^{2} = \sum_{k = 1}^{n} (x_{k}^{2} + y_{k}^{2}) = \sum_{k=1}^{n} \abs{z_{k}}^{2}$$
\end{definition}

\begin{definition}
The \textbf{inner product} of two vectors $x, y \in \mathbb{R}^{n}$ is
$$(x, y) = x_{1} y_{1} + \cdots + x_{n} y_{n} = x^{T} y = y^{T} x$$

This yields another definition for the \textbf{norm}: 
$$\norm{x} = \sqrt{(x, x)}$$

For complex fields, we need a definition of inner product such that $\norm{z}^{2} = (z, z)$. One definition that is consistent with this requirement will be our definition for the \textbf{\textit{standard} inner product in $\mathbb{C}^{n}$}, 
$$(z, w) = z_{1} \overline{w_{1}} + \cdots + z_{n} \overline{w_{n}}$$

To simplify this, we will define the \textbf{Hermitian adjoint}, or simply \textbf{adjoint} $A^{*}$, by $A^{*} = \overline{A}^{T}$. 

Using this, we can write 
$$(z, w) = w^{*} z$$ 
\end{definition}

The inner products we defined for $\mathbb{R}^{n}$ and $\mathbb{C}^{n}$	have the following properties: 
\begin{enumerate}
	\item Symmetry: $(x, y) = \overline{(y, x)}$
	\item Linearity: $(\alpha x + \beta y, z) = \alpha (x, z) + \beta (y, z)$
	\item Non-negativity: $(x, x) \geq 0$ 
	\item Non-degeneracy: $(x, x) = 0$ if and only if $x = 0$
\end{enumerate}

Note that properties 1 and 2 imply that 
$$(x, \alpha y + \beta z) = \overline{(\alpha y + \beta z, x)} = \overline{\alpha} (x, y) + \overline{\beta} (x, z)$$

\begin{lemma}
Let $x$ be a vector in $V$. Then $x = 0$ if and only if  
$$(x, y) = 0 \qquad \forall y \in V$$
\end{lemma}

\begin{lproof}
Since $(0, y) = 0$, we need to only show that $x = 0$ if $(x, y) = 0$. Subbing in $y = x$, we get $(x, x) = 0$ and property 3 asserts that $x = 0$. 
\end{lproof}

\begin{lemma}
Let $x, y$ be vectors in $V$. Then $x = y$ if and only if  
$$(x, z) = (y, z) \qquad \forall z \in V$$
\end{lemma}

\begin{lproof}
Using the above lemma, if we set $(x - y, z) = 0 \quad \forall z \in V$, then it follows that $x = y$ and $(x, z) = (y, z)$. 

\end{lproof}

\begin{theorem}
Suppose two operators $X, Y : A \rightarrow B$ satisfy 
$$(Ax, y) = (Bx, y) \qquad \forall x \in X, \forall y \in Y$$
Then $A = B$. 
\end{theorem}

\begin{proof}
Using the previous lemma, we can fix $x$ and take all $y \in Y$, which means $Ax = Bx$. Since this is true for all $x$, $A$ and $B$ are the same operator. 
\end{proof}

\begin{theorem}[Cauchy-Schwartz Inequality]
$$\abs{(x, y)} \leq \norm{x} \cdot \norm{y}$$
\end{theorem}

\begin{proof}
If $x$ or $y$ is 0, then the proof is trivial. Assuming neither is 0, we will prove both the real and complex cases. But first consider only the real case: 
$$0 \leq \norm{x - ty}^{2} = (x - ty, x - ty) = \norm{x}^{2} - 2t(x, y) + t^{2} \norm{y}^{2} $$
Taking the derivative with respect to $t$ and setting it to 0 gives us $t = \frac{(x, y)}{\norm{y}^{2}}$. We will use this same $t$ value for the following proof of the real and complex cases:

$$
\begin{aligned}
0 \leq \norm{x-ty}^{2} &= (x - ty, x - ty) \\ 
&= (x, x - ty) -t(y, x - ty) \\
&= \norm{x}^{2} - \overline{t} (x, y) -t (y, x) + \abs{t}^{2} \norm{y}^{2}
\end{aligned}
$$

Using property 1 of inner products, we have 
$$t = \frac{(x, y)}{\norm{y}^{2}} = \frac{\overline{(y, x)}}{\norm{y}^{2}}$$
Subbing in $t$, we get 
$$ 0 \leq \norm{x}^{2} - \frac{\abs{(x y)}^{2}}{\norm{y}^{2}}$$
which completes the proof. 
\end{proof}

\begin{theorem}[Triangle Inequality]
$$\norm{x, y} \leq \norm{x} + \norm{y}$$
\end{theorem}

\begin{proof}
$$
\begin{aligned}
\norm{x + y}^{2} = (x + y, x + y) &= \norm{x}^{2} + \norm{y}^{2} + (x, y) + (y, x) \\
&\leq \norm{x}^{2} + \norm{y}^{2} + 2 \abs{(x, y)} \\
&\leq \norm{x}^{2} + \norm{y}^{2} + 2 \norm{x} \cdot \norm{y} \\
&= (\norm{x} + \norm{y})^{2}
\end{aligned}
$$
\end{proof}

\begin{theorem}
The following \textbf{polarization identities} allow us to construct the inner product from the norm: 

For $x, y \in \mathbb{R}^{n}$, 
$$(x, y) = \frac{1}{4} \Big( \norm{x + y}^{2} - \norm{x - y}^{2} \Big) $$

For $x, y \in \mathbb{C}^{n}$, 
$$(x, y) = \frac{1}{4} \Big( \norm{x + y}^{2} - \norm{x - y}^{2} +i \norm{x + iy}^{2} -i \norm{x - iy}^{2} \Big)$$
\end{theorem}

\begin{proof}
For the real case, 
$$ 
\begin{aligned}
\norm{x + y}^{2} - \norm{x - y}^{2} &= (x + y, x + y) - (x - y, x - y) \\
&= \norm{x}^{2} + \norm{y}^{2} + 2 (x, y) - \norm{x}^{2} - \norm{y}^{2} + 2 (x, y) \\ 
&= 4 (x, y)
\end{aligned}
$$

For the complex case, 
$$
\begin{aligned}
\sum_{k = 0}^{3} i^{k} \norm{x + i^{k}y}^{2} &= \sum_{k = 0}^{3} i^{k} (x + i^{k}y, x + i^{k}y) \\
&= \sum_{k = 0}^{3} i^{k} \Big( \norm{x}^{2} + \norm{y}^{2} + (x, i^{k}y) + (i^{k}y, x) \Big) \\
&= \sum_{k = 0}^{3} \Big( i^{k} \norm{x}^{2} + i^{k} \norm{y}^{2} + (x, y) + (i^{2k} y, x) \Big) \\
&= 4 (x, y)
\end{aligned}
$$
where the last step follows from 
$$ \sum_{k = 0}^{3} i^{k} = \sum_{k = 0}^{3} i^{2k} = 0$$
\end{proof}

\begin{theorem}[Parallelogram Identity]
Another important property of the norm is the parallelogram identity. For vectors $u$ and $v$: 
$$\norm{u + v}^{2} + \norm{u - v}^{2} = 2( \norm{u}^{2} + \norm{v}^{2} )$$
\end{theorem}

\begin{proof}
The theorem follows easily from the fact that the sum of the diagonals of a parallelogram equal the sum of all four sides. 
\end{proof}

To review, we have so far proved the following properties about the norm $\norm{u}$: 
\begin{enumerate}
	\item Homogeneity: $\norm{\alpha u} = \abs{\alpha} \cdot \norm{u}$ 
	\item Triangle inequality: $\norm{u + v} \leq \norm{u} + \norm{v}$
	\item Non-negativity: $\norm{u} \geq 0$ 
	\item Non-degeneracy: $\norm{u} = 0$ if and only if $u = 0$
\end{enumerate}

In a vector space $V$, if we assign to each vector $u$ a number $\norm{u}$ that satisfies these 4 properties, we can say that the space $V$ is a \textbf{normed space}. 

\section{Orthogonality} 
\begin{definition}
Two vectors $u$ and $v$ are \textbf{orthogonal}, denoted $u \perp v$, if and only if $(u, v) = 0$
\end{definition}

\begin{theorem}
If $u \perp v$, then 
$$\norm{u + v}^{2} = \norm{u}^{2} + \norm{v}^{2}$$
\end{theorem}

\begin{proof}
$$\norm{u + v}^{2} = \norm{u}^{2} + \norm{v}^{2} + (u, v) + (v, u) = \norm{u}^{2} + \norm{v}^{2}$$
Since $(u,v) = (v, u) = 0$ because of orthogonality. 
\end{proof}

\begin{definition}
A vector $u$ is \textbf{orthogonal to vector space} $V$ if $u$ is orthogonal to all vectors in $V$. 
\end{definition}

\begin{theorem}
Let $V$ be spanned by $v_{1}, \cdots, v_{n}$. Then $u \perp V$ if and only if 
$$u \perp v_{k} \qquad \forall k = 1, \cdots, n$$
\end{theorem}

\begin{proof}
Proving ``only if'' is trivial by the definition of $u \perp V$. Proving ``if'' comes easily after noticing that any vector can be rewritten as a linear combination of the basis vectors, so if $u$ is perpendicular to all the basis vectors, then it is perpendicular to any other vector in $V$. 
\end{proof}

\begin{definition}
A set of vectors $v_{1}, \cdots, v_{n}$ are orthogonal if any two vectors in the set are orthogonal to each other. If $\norm{v_{k}} = 1$ for all $k$, we call the set orthonormal. 
\end{definition}

\begin{lemma}[Generalized Pythagorean Theorem]
Let $v_{1}, \cdots, v_{n}$ be an orthogonal system. Then 
$$\norm{\sum_{k=1}^{n} a_{k} v_{k}}^{2} = \sum_{k=1}^{n} \abs{a_{k}}^{2} \norm{v_{k}}^{2}$$
\end{lemma}

\begin{lproof}
$$\norm{\sum_{k=1}^{n} a_{k} v_{k}}^{2} = \Big( \sum_{k=1}^{n} a_{k} v_{k}, \sum_{j=1}^{n} a_{j} v_{j} \Big) = \sum_{k=1}^{n} \sum_{j=1}^{n} a_{k} \overline{a_{j}} (v_{k}, v_{j})$$

Since the set is orthogonal, $(v_{k}, v_{j})$ is only nonzero when $k = j$, so 
$$= \sum_{k=1}^{n} \abs{a_{k}}^{2} \norm{v_{k}}^{2}$$
\end{lproof}

\begin{definition}
An orthogonal set of vectors that is also a basis is called an \textbf{orthogonal basis}. 
\end{definition}

Typically, to find coordinates of a vector in a basis, we need to solve a system of equations. For orthogonal bases, it is much simpler. Suppose $v_{1}, \cdots, v_{n}$ is an orthogonal basis and let 
$$x = \alpha_{1} v_{1} + \cdots + \alpha_{n} v_{n}$$

Taking the inner product with $v_{1}$ yields 
$$(x, v_{1}) = (\sum_{j=1}^{n} \alpha_{j} (v_{j}, v_{1}) = \alpha_{1} (v_{1}, v_{1}) = \alpha_{1} \norm{v_{1}}^{2}$$

Thus, to find any coordinate $\alpha_{k}$ of a vector $x$ in orthogonal basis $v_{1}, \cdots, v_{n}$: 
$$ \alpha_{k} = \frac{(x, v_{k})}{\norm{v_{k}}^{2}}$$

This is a simple example of abstract orthogonal Fourier decomposition -- simple because classical Fourier decomposition deals with infinite orthonormal systems. 

\section{Orthogonal Projection and Gram-Schmidt Orthogonalization} 

\begin{definition}
The \textbf{orthogonal projection} of a vector $v$ onto the subspace $E$ is the vector $w := P_{E} v$ such that $w \in E$ and $v - w \perp E$. 
\end{definition}

\begin{theorem}
The orthogonal projection $w = P_{E} v$ minimizes the distance from $v$ to $E$. In other words, 
$$\norm{v - w} \leq \norm{v - x} \qquad \forall x \in E$$
Additionally, if for some $x \in E$
$$\norm{v - w} = \norm{v - x}$$
then $x = w$. 
\end{theorem}

\begin{proof}
Let $y = w - x \in E$. Then 
$$v - x = v - w + w - x = v - w + y$$
Since $v - w \perp E$, we know $y \perp v - w$. By the Pythagorean Theorem, 
$$\norm{v - x}^{2} = \norm{v - w}^{2} + \norm{y}^{2} \geq \norm{v - w}^{2}$$

To finish the proof, note that equality only arises when $y = 0$, ie when $x = w$. 
\end{proof}

There is a formula for finding an orthogonal projection if we know an orthogonal basis in $E$. Let $v_{1}, \cdots, v_{n}$ be an orthogonal basis in $E$. Then the projection $P_{E} v$ of a vector $v$ is 
$$P_{E} v = \sum_{k=1}^{n} a_{k} v_{k} \qquad where \qquad a_{k} = \frac{(v, v_{k})}{\norm{v_{k}}^{2}}$$

In other words, 
$$P_{E} v = \sum_{k=1}^{n} \frac{(v, v_{k})}{\norm{v_{k}}^{2}} v_{k}$$

This is great if we have an orthogonal basis, but if even if we only have a basis in $E$, we can use the following algorithm to find an orthogonal basis. 

\begin{theorem}[Gram-Schmidt Orthogonalization Algorithm]
Suppose we have linearly independent system $x_{1}, \cdots, x_{n}$. The Gram-Schmidt algorithm constructs from this an orthogonal system $v_{1}, \cdots, v_{n}$ such that 
$$span(x_{1}, \cdots, x_{n}) = span(v_{1}, \cdots, v_{n})$$
Additionally, for all $r \leq n$ 
$$span(x_{1}, \cdots, x_{r}) = span(v_{1}, \cdots, v_{r})$$

The algorithm is as follows: 
\begin{enumerate}
	\item Define $v_{1} := x_{1}$. 
	
	Define $E_{1} := span(v_{1}) = span(x_{1})$. 
	\item Define $v_{2} := x_{2} - P_{E_{1}} x_{2} = x_{2} - \frac{(x_{2}, v_{1})}{\norm{v_{1}}^{2}} v_{1}$. 
	
	Define $E_{2} := span(v_{1}, v_{2}) = span(x_{1}, x_{2})$. 
	\item Define $v_{3} := x_{3} - P_{E_{2}} x_{3} = x_{3} - \frac{(x_{3}, v_{1})}{\norm{v_{1}}^{2}} v_{1} - \frac{(x_{3}, v_{2})}{\norm{v_{2}}^{2}} v_{2}$. 
	
	Define $E_{3} := span(v_{1}, v_{2}, v_{3}) = span(x_{1}, x_{2}, x_{3})$. 
	
	\item Continue until we have $n$ vectors and $span(v_{1}, \cdots, v_{n}) = span(x_{1}, \cdots, x_{n})$. The formula for vector $v_{r + 1}$ given $v_{1}, \cdots, v_{r}$ is 
	$$ v_{r + 1} := x_{r + 1} - P_{E_{r}} x_{r + 1} = x_{r + 1} - \sum_{k = 1}^{r} \frac{(x_{r + 1}, v_{k})}{\norm{v_{k}}^{2}} v_{k}$$
\end{enumerate}

Note that at each step, we are adding in $x_{r + 1}$ which means the resulting vector will not exist in $E_{r}$. 
\end{theorem}

\begin{proof}
At each step, we add in $x_{r + 1}$ and then subtract its projection the subspace spanned by $x_{1}, \cdots, x_{r}$, meaning each additional vector is orthogonal to the ones previously defined. Since we set $v_{1} = x_{1}$, we have proved the algorithm by induction. 
\end{proof}

Since multiplication by a scalar does not change orthogonality, we can multiply vectors $v_{k}$ returned by Gram-Schmidt by any non-zero numbers. One use case is to normalize the orthogonal vectors by dividing by their norms $\norm{v_{k}}$ to yield an orthonormal system. 

\begin{definition}
For a subspace $E$, its \textbf{orthogonal complement $E^{\perp}$} is the set of all vectors orthogonal to $E$. Since at least 0 is orthogonal to $E$, $E^{\perp}$ is always a subspace. 
\end{definition}

By the definition of orthogonal projection, any vector in an inner product space $V$ has a unique representation of the form 
$$v = v_{1} + v_{2} \qquad v_{1} \in E, v_{2} \in E^{\perp}$$
This statement is usually written as $V = E \oplus E^{\perp}$.

\begin{theorem}
For subspace $E$ of $V$, 
$$(E^{\perp})^{\perp} = E$$
\end{theorem}

\begin{proof}
We will show $E \subseteq (E^{\perp})^{\perp}$ and $(E^{\perp})^{\perp} \subseteq E$.

Let $u \in E$. Then $(u, v) = 0$ for all $v \in E^{\perp}$. Since $u$ is orthogonal to every vector $v \in E^{\perp}$, then $u \in (E^{\perp})^{\perp}$ so $E \subseteq (E^{\perp})^{\perp}$. 

Now let $u \in (E^{\perp})^{\perp}$. Since $V = E \oplus E^{\perp}$, we can write $u = v + w$, where $v \in E$ and $w \in E^{\perp}$. This means that $u - v = w \in E^{\perp}$. Since we know $E \subseteq (E^{\perp})^{\perp}$, we have $u \in  (E^{\perp})^{\perp}$ and $v \in (E^{\perp})^{\perp}$, which means $u - v \in  (E^{\perp})^{\perp}$. Therefore, $u - v \in E^{\perp} \cap (E^{\perp})^{\perp}$. Since the only vector that is orthogonal to itself is 0, $u = v$, and because $v \in E$, $(E^{\perp})^{\perp} \subseteq E$. 
\end{proof}

\section{Least Square Solution} 
Recall that $Ax = b$ has a solution if and only if $b \in Range(A)$. In real life, it is impossible to avoid errors. The simplest way to approximate a solution is to choose an approximation $\hat{x}$ to minimize the error $e = \norm{A \hat{x} - b}$. This is the \textbf{least square solution}. 

We know $A \hat{x}$ is the orthogonal projection $P_{Range(A)} b$ if and only if $b - A \hat{x} \perp Range(A)$. Using the column space interpretation of range, this is equivalent to 
$$b - A \hat{x} \perp a_{k} \qquad \forall k = 1, \cdots, n$$
That means 
$$0 = (b - A \hat{x}, a_{k}) = a^{*}_{k} (b - A \hat{x}) \qquad \forall k = 1, \cdots, n$$ 
We can join the rows $a^{*}_{k}$ together to get
$$A^{*} (b - A \hat{x}) = 0$$
which is equivalent to the \textbf{normal equation} 
$$A^{*} A \hat{x} = A^{*} b$$

The solution $\hat{x}$ to this equation grants us the least square solution of $A \hat{x} = b$. This makes it easy to notice that the least square solution is unique if and only if $A^{*} A$ is invertible. 

If $\hat{x}$ is the solution to the normal equation, then $A \hat{x} = P_{Range(A)} b$. So in order to find the actual projection of $b$ onto $Range(A)$, we need to solve the normal equation and then multiply the solution by $A$. Formally, 
$$P_{Range(A)} b = A (A^{*} A)^{-1} A^{*} b$$
Because this is true for all $b$, the formula for the matrix of the orthogonal projection onto $Range(A)$ is 
$$P_{Range(A)} = A (A^{*} A)^{-1} A^{*}$$

\begin{theorem}
For an $m \times n$ matrix $A$ 
$$Ker(A) = Ker(A^{*}A)$$

Recall Kernel is equivalent to Null Space.
\end{theorem}

\begin{proof}
We will show $Ker(A) \subseteq Ker(A^{*} A)$ and $Ker(A^{*} A) \subseteq Ker(A)$. 

To prove the latter, suppose we have a vector $u \in Ker(A)$ so that $Au = 0$. Then $A^{*} A u = A^{*} (Au) = A^{*} 0 = 0$, which means $u \in Ker(A^{*} A)$. 

To prove the former, suppose we have a vector $v \in Ker(A^{*} A)$. We want to show that $Av = 0$. One way of doing so is to show that its norm is 0. 
$$\norm{Av}^{2} = (Av, Av) = (A^{*} v^{*}, A^{*} v^{*}) = A^{*} (v^{*}, A^{*} v^{*}) = A^{*} (Av, v) = (A^{*} Av, v) = (0, v) = 0$$
\end{proof}

\section{Adjoint of a Linear Transformation} 
Recall that the \textit{Hermitian adjoint} $A^{*}$ of matrix $A$ is defined as the complex conjugate of each entry in $A^{T}$. 

\begin{theorem}
$$(Ax, y) = (x, A^{*}y) \qquad \forall x \in \C^{n}, \forall y \in \C^{m}$$
\end{theorem}

\begin{proof}
$$(Ax, y) = y^{*} Ax = (A^{*} y)^{*} x = (x, A^{*} y)$$

The second equality uses the fact that because the adjoint consists of a transpose, we have $(AB)^{*} = B^{*} A^{*}$ and $(A^{*})^{*} = A$.  
\end{proof}

This identity is used to define the adjoint operator. 

\begin{lemma}
The adjoint is unique. 
\end{lemma}

\begin{lproof}
Suppose $B$ satisfies $(Ax, y) = (x, By) \qquad \forall x, y$, then we can write 
$$(Ax, y) = (x, A^{*} y) = (x, By)$$
which means $A^{*} = B$.
\end{lproof}

\textbf{Properties of the adjoin operators (matrices):}
\begin{enumerate}
	\item $(A + B)^{*} = A^{*} + B^{*}$ 
	\item $(\alpha A)^{*} = \overline{\alpha} A^{*}$ 
	\item $(AB)^{*} = B^{*} A^{*}$ 
	\item $(A^{*})^{*} = A$
	\item $(y, Ax) = (A^{*}y, x)$ 
\end{enumerate}

\begin{theorem}[Relation between fundamental subspaces]
Let $A: V \rightarrow W$ be an operator acting from one inner product space to another. Then 
\begin{enumerate}
	\item $Ker(A^{*}) = (Range(A))^{\perp}$ 
	\item $Ker(A) = (Range(A^{*}))^{\perp}$ 
	\item $Range(A) = (Ker(A^{*}))^{\perp}$ 
	\item $Range(A^{*}) = (Ker(A))^{\perp}$
\end{enumerate}

Note that earlier we defined the fundamental subspaces using $A^{T}$ instead of $A^{*}$ because when discussing only $\R$ there was no difference. 
\end{theorem}

\begin{proof}
Note that statements 1/3 and 2/4 are equivalent because for any subspace $E$, we have $(E^{\perp})^{\perp} = E$. Also note that statement 2 is exactly statement 1 applied to the operator $A^{*}$ since $(A^{*})^{*} = A$. 

Thus we only need to prove statement 1. 

A vector $x \in (Range(A))^{\perp}$ means that $x$ is orthogonal to all vectors of the form $Ay$, that is 
$$(x, Ay) = 0 \qquad \forall y$$ 
Since $(x, Ay) = (A^{*}x, y)$, this is equivalent to 
$$(A^{*} x, y) = 0 \qquad \forall y$$

This means that $A^{*}x = 0$, which means $x \in Ker(A^{*})$. 
\end{proof}

\section{Isometries and Unitary Operators}
\begin{definition}
An operator $U: X \rightarrow Y$ is called an \textbf{isometry} if it preserves the norm, 
$$\norm{Ux} = \norm{x} \qquad \forall x \in X$$
\end{definition}

\begin{theorem}
An operator $U: X \rightarrow Y$ is an isometry if and only if it preserves the inner product, ie if and only if 
$$(x, y) = (Ux, Uy) \qquad \forall x, y \in X$$
\end{theorem}

\begin{proof}
We use the polarization identities previously described. If $X$ is a complex space

$$
\begin{aligned}
(Ux, Uy) &= \frac{1}{4} \sum_{\alpha = \pm 1, \pm i} \alpha \norm{Ux + \alpha Uy}^{2} \\ 
&= \frac{1}{4} \sum_{\alpha = \pm 1, \pm i} \alpha \norm{U(x + \alpha y)}^{2} \\
&= \frac{1}{4} \sum_{\alpha = \pm 1, \pm i} \alpha \norm{x + \alpha y}^{2} = (x, y)
\end{aligned}
$$

If $X$ is a real space 
$$\begin{aligned} 
(Ux, Uy) &= \frac{1}{4}(\norm{Ux + Uy}^{2} - \norm{Ux - Uy}^{2}) \\
&= \frac{1}{4} (\norm{U(x + y)}^{2} - \norm{U(x - y)}^{2}) \\
&= \frac{1}{4} (\norm{x + y}^{2} - \norm{x - y}^{2}) = (x, y)
\end{aligned}
$$
\end{proof}

\begin{lemma}
An operator $U: X \rightarrow Y$ is an isometry if and only if $U^{*}U = I$. 
\end{lemma}

\begin{lproof}
If $U^{*}U = I$, then 
$$(x, x) = (U^{*}Ux, x) = (Ux, Ux) \qquad \forall x \in X$$
Since $\norm{x} = \norm{Ux}$, $U$ is an isometry. 

If $U$ is an isometry, then by the above theorem and definition of adjoint
$$(U^{*} Ux, y) = (Ux, Uy) = (x, y) \qquad \forall x, y \in X$$
which means $U^{*}U = I$. 
\end{lproof}

This lemma implies that an isometry is always left invertible since $U^{*}U = I$. 

\begin{definition}
An isometry $U: X \rightarrow Y$ is called a \textbf{unitary operator} if it is invertible.  
\end{definition}

\begin{lemma}
An isometry $U: X \rightarrow Y$ is a unitary operator if and only if $dim(X) = dim(Y)$. 
\end{lemma}

\begin{lproof}
If $dim(X) = dim(Y)$, then $U$ is square. Since we know $U$ is left invertible, it must also then be invertible. 

If $U$ is unitary, it is invertible, so $dim(X) = dim(Y)$ since only square matrices are invertible. 
\end{lproof}

Properties of unitary operators that follow from our proofs: 
\begin{enumerate}
	\item $U^{-1} = U^{*}$ 
	\item If $U$ is unitary, $U^{*} = U^{-1}$ is also unitary. 
	\item If $U$ is an isometry and $v_{1}, \cdots, v_{n}$ is an orthonormal basis, then $Uv_{1}, \cdots, Uv_{n}$ is an orthonormal basis. 
	\item The product of unitary operators is a unitary operator. 
\end{enumerate}

\begin{lemma}
$$det(A^{*}) = \overline{det(A)}$$
\end{lemma}

\begin{lproof}
Recall that the determinant of a matrix is equal to the product of its eigenvalues. We will show that the for any eigenvalue $\lambda$ of $A$, $\overline{\lambda}$ is an eigenvalue of $A^{*}$. 

Note that $\lambda$ is \textbf{not} an eigenvalue of $A$ if and only if $A - \lambda I$ is invertible, which happens if and only if there exists an operator $B$ such that 
$$B(A - \lambda I) = (A - \lambda I)B = I$$
Taking the adjoints of all three sides means the above is equivalent to 
$$(A^{*} - \overline{\lambda}I)B^{*} = B^{*} (A^{*} - \overline{\lambda}I) = I$$

Thus $A - \lambda I$ is invertible if and only if $A^{*} - \overline{\lambda}I$ is invertible, which means if $\lambda$ is an eigenvalue of $A$, $\overline{\lambda}$ is an eigenvalue of $A^{*}$. 
\end{lproof}

\begin{theorem}
If $U$ is a unitary matrix, then 
$$ det(U) = \pm 1$$
If $\lambda$ is an eigenvalue of $U$, then 
$$\lambda = \pm 1$$
\end{theorem}

\begin{proof}
Let $det(U) = z$. Since $det(U^{*}) = \overline{det(U)}$, we have 
$$\abs{z}^{2} = \overline{z}z = det(U^{*} U) = det(I) = 1$$

To prove statement 2, notice that if $Ux = \lambda x$, then 
$$\norm{Ux} = \norm{\lambda x} = \abs{\lambda} \cdot \norm{x}$$
which means $\abs{\lambda} = 1$ since $\norm{Ux} = \norm{x}$. 
\end{proof}

\begin{definition}
Operators $A$ and $B$ are called \textbf{unitarily equivalent} if there exists a unitary operator $U$ such that $A = UBU^{*}$. Since for any unitary $U$, we have $U^{-1} = U^{*}$, any two unitarily equivalent matrices are similar as well. 

The converse is \textbf{not} true. 
\end{definition}

The following theorem gives a way to construct a counter example to prove similar matrices are not always unitarily equivalent. 

\begin{theorem}
A matrix $A$ is unitarily equivalent to a diagonal one if and only if it has an orthogonal (orthonormal) basis of eigenvectors. 
\end{theorem}

\begin{proof}
Using diagonalization, we can write $A = UBU^{*}$ and let $Bx = \lambda x$. Then $AUx = UBx = \lambda Ux$, which means $Ux$ is an eigenvector of $A$. 

Only if: Let $A$ be unitarily equivalent to a diagonal matrix $D$, ie $A = UDU^{*}$. Because $D$ is diagonal, the vectors $e_{k}$ of the standard basis are eigenvectors of $D$, so $Ue_{k}$ are eigenvectors of $A$. Since $U$ is unitary, $Ue_{1}, \cdots, Ue_{n}$ is an orthonormal basis. 

If: Let $A$ have an orthogonal basis $u_{1}, \cdots, u_{n}$ of eigenvectors. By dividing each vector by its norm, we can assure we have an orthonormal basis. By letting $D$ be the matrix $A$ in the basis $u_{1}, \cdots, u_{n}$, we know $D$ will be a diagonal matrix. 

By setting $U$ to be the matrix with columns $u_{1}, \cdots, u_{n}$, we know $U$ is unitary since its columns form an orthonormal basis (orthogonality implies invertibility and normality implies norm preservation). The change of coordinate formula implies 
$$A = [A]_{SS} = [I]_{SB} [A]_{BB} [I]_{BS} = UDU^{-1} = UDU^{*}$$
where the last step follows from $U^{-1} = U^{*}$ for unitary matrices. 
\end{proof}

%%% DID NOT DO RIGID MOTIONS AND COMPLEXIFICATION!!
%%% DID NOT DO RIGID MOTIONS AND COMPLEXIFICATION!!
%%% DID NOT DO RIGID MOTIONS AND COMPLEXIFICATION!!
\chapter{Structure of Operators}

\section{Schur Representation} 

\begin{theorem}
Any operator $A: X \rightarrow X$ has an orthonormal basis $u_{1}, \cdots, u_{n}$ in $X$ such that $A$ in this basis is upper triangular. In other words, any $n \times n$ matrix $A$ can be written as 
$$A = UTU^{*}$$
where $U$ is unitary and $T$ is an upper triangular matrix. 
\end{theorem}

\begin{proof}
We proceed by induction on the dimension of $X$. 

\textbf{Base case:} $dim(X) = 1$

Then $A$ is simply a $1 \times 1$ matrix which is trivially upper triangular. 

\textbf{Inductive hypothesis:} Assume for $X$ with $dim(X) = n$, $A$ can be written as upper triangular in an orthonormal basis. 

\textbf{Inductive step:} We prove the statement for $dim(X) = n + 1$. 

Let $\lambda_{1}$ be an eigenvalue of $A$ and let $u_{1}$ be the corresponding normalized eigenvector of $A$, $Au_{1} = \lambda_{1} u_{1}$. Define $E := u_{1}^{\perp}$ and suppose $v_{2}, \cdots, v_{n + 1}$ is an orthonormal basis of $E$. Note $dim(E) = dim(X) - 1 = n$, so $u_{1}, v_{2}, \cdots, v_{n + 1}$ is an orthonormal basis of $X$. In this basis, $A$ has the form 
$$\begin{bmatrix}[c|c]
\lambda_{1} & *** \\
\hline 
\quad 0 \quad &  \\
\vdots & \qquad A_{1} \qquad \\
0 & 
\end{bmatrix}
$$
The $***$ denote values we don't care about and $A_{1}$ is an $n \times n$ block. Since $A_{1}$ denotes an operator in $E$, we can use our hypothesis to conclude there is an orthonormal basis $u_{2}, \cdots, u_{n + 1}$ in which $A_{1}$ is upper triangular. So in basis $u_{1}, \cdots, u_{n + 1}$, matrix $A_{1}$ is upper triangular, so matrix $A$ is also upper triangular.
\end{proof}

One use of the Schur representation is to present a more intuitive proof that the determinant of a matrix is the product of its eigenvalues and that the trace of a matrix is the sum of its eigenvalues. 

\section{Spectral Theorem for Self-Adjoint and Normal Operators}

\begin{definition}
An operator $A$ is \textbf{self-adjoint} if $A = A^{*}$. The matrix of a self-adjoint operator is called a \textbf{Hermitian matrix}. There is not a huge distinction between these two terms: self-adjoint refers to a general transformation while Hermitian refers to its matrix representations.
\end{definition}

\begin{theorem}
Let $A$ be a self-adjoint operator in \textbf{real or complex} inner product space $X$. Then all eigenvalues of $A$ are real and there exists an orthonormal basis of eigenvectors of $A$ in $X$. In matrix form, 
$$A = UDU^{*}$$
where $U$ is unitary and $D$ is diagonal with real entries. Additionally, if $A$ is real, $U$ can be chosen to be real. 
\end{theorem}

\begin{proof}
We apply our previous theorem to generate a Schur representation of $A$ in some orthonormal basis. What upper triangular matrices are self-adjoint? Only diagonal matrices with real entries meet this condition. This is a sufficient proof of the entire theorem. 

For an independent proof that the eigenvalues of self-adjoint operators are real, let $A = A^{*}$ and $Ax = \lambda x, x \neq 0$. Then 
$$(Ax, x) = (\lambda x, x) = \lambda (x, x) = \lambda \norm{x}^{2}$$
Additionally, 
$$(Ax, x) = (x, A^{*}x) = (x, Ax) = (x, \lambda x) = \overline{\lambda} (x, x) = \overline{\lambda} \norm{x}^{2}$$

This means $\lambda \norm{x}^{2} = \overline{\lambda} \norm{x}^{2}$, which means $\lambda = \overline{\lambda}$ since $x \neq 0$. 

We can also independently prove that eigenvectors of self-adjoint operators are orthogonal. Let $A = A^{*}$ be a self-adjoint operator and let $u, v$ be distinct eigenvectors such that $Au = \lambda u$ and $Av = \mu v$. Then 
$$(Au, v) = \lambda (u, v)$$ 
On the other hand, 
$$(Au, v) = (u, A^{*} v) = (u, Av) = (u, \mu v) = \overline{\mu} (u, v) = \mu (u, v)$$
Thus, we have $\lambda (u, v) = \mu (u, v)$. Since $\lambda \neq \mu$, $(u, v) = 0$. 
\end{proof}

What matrices are unitarily equivalent to a diagonal matrix? We know that $D^{*} D = DD^{*}$ for any diagonal matrix $D$. Therefore, $A^{*} A = AA^{*}$ if matrix $A$ is diagonal in some orthonormal basis. 

\begin{definition}
An operator $N$ is \textbf{normal} if $N^{*}N = NN^{*}$. Clearly, any self-adjoint operator ($A^{*}A = AA^{*}$) and any unitary operator ($U^{*}U = UU^{*} = I$) are normal. Note, however, that a normal operator must act in one space, so any unitary operator acting from one space to another is not normal. 
\end{definition}

\begin{theorem}
Any normal operator $N$ in a complex vector space has an orthogonal basis of eigenvectors. In other words, 
$$N = UDU^{*}$$
where $U$ is unitary and $D$ is diagonal. Note that even if $N$ is real, we do not claim that $U$ or $D$ are real. Also note that if $D$ is real, then by the previous theorem, $N$ is self-adjoint. 
\end{theorem}

\begin{proof}
We apply Schur representation to get an orthonormal basis such that $N$ in this basis is upper triangular. Now all we need to show is that an upper triangular normal matrix must be diagonal. 

We proceed by induction on dimension of $N$. 

\textbf{Base case: } $dim(N) = 1$

A $1 \times 1$ matrix is trivially upper triangular. 

\textbf{Inductive hypothesis: } Assume that any $n \times n$ upper triangular normal matrix is diagonal. 

\textbf{Inductive step: } Prove an $(n+1) \times (n+1)$ upper triangular normal matrix $N$ is also diagonal. 

$$N = \begin{bmatrix}[c|c]
a_{11} & a_{12} \cdots a_{1n} \\
\hline
0 & \\
\vdots & N_{1} \\
0 & 
\end{bmatrix}$$
where $N_{1}$ is an upper triangular $n \times n$ matrix. (This is the strongest assumption Schur's representation allows us to make.)

We know $N$ is normal and that for normal matrices $N^{*}N = NN^{*}$. Direct computation yields 
$$(N^{*} N)_{11} = \overline{a_{11}} a{11} = \abs{a_{11}}^{2}$$
and 
$$(NN^{*})_{11} = \abs{a_{11}}^{2} + \cdots + \abs{a_{1n}}^{2}$$
so $(N^{*}N)_{11} = (NN^{*})_{11}$ if $a_{12}, \cdots, a_{1n}$ are all 0. Therefore, 
$$N = \begin{bmatrix}[c|c]
a_{11} & 0 \quad \cdots \quad 0 \\
\hline
0 & \\
\vdots & N_{1} \\
0 & 
\end{bmatrix}$$

Since $N$ is normal, 

$$N^{*}N = \begin{bmatrix}[c|c]
\abs{a_{11}}^{2} & 0 \quad \cdots \quad 0 \\
\hline
0 & \\
\vdots & N_{1}^{*} N_{1} \\
0 & 
\end{bmatrix} \qquad \text{ and } \qquad NN^{*} = \begin{bmatrix}[c|c]
\abs{a_{11}}^{2} & 0 \quad \cdots \quad 0 \\
\hline
0 & \\
\vdots & N_{1} N_{1}^{*} \\
0 & 
\end{bmatrix}$$

so $N_{1}^{*} N_{1} = N_{1} N_{1}^{*}$, which means $N_{1}$ is also normal, and by the inductive hypothesis, it is diagonal. Thus, matrix $N$ is diagonal. 
\end{proof}

The following lemma demonstrates a useful characteristic of normal operators. 

\begin{lemma}
An operator $N: X \rightarrow X$ is normal if and only if 
$$\norm{Nx} = \norm{N^{*} x} \qquad \forall x \in X$$
\end{lemma}

\begin{lproof}
Only if: Let $N$ be normal so $N^{*}N = NN^{*}$. Then 
$$\norm{Nx}^{2} = (Nx, Nx) = (N^{*}Nx, x) = (NN^{*} x, x) = (N^{*} x, N^{*} x) = \norm{N^{*} x}^{2} $$
so $\norm{Nx} = \norm{N^{*} x}$.

If: Suppose $\norm{Nx} = \norm{N^{*} x}$ and we want to show $N^{*}N = NN^{*}$. The Polarization Identities imply that $ \forall x, y \in X$, 

$$
\begin{aligned} 
(N^{*}N x, y) = (Nx, Ny) &= \frac{1}{4} \sum_{\alpha = \pm 1, \pm i} \alpha \norm{Nx + \alpha Ny}^{2} \\
&= \frac{1}{4} \sum_{\alpha = \pm 1, \pm i} \alpha \norm{N( x + \alpha y)}^{2} \\
&= \frac{1}{4} \sum_{\alpha = \pm 1, \pm i} \alpha \norm{N^{*} ( x + \alpha y)}^{2} \\
&= \frac{1}{4} \sum_{\alpha = \pm 1, \pm i} \alpha \norm{N^{*} x + \alpha N^{*} y}^{2} \\
&= (N^{*} x, N^{*} y) = (NN^{*} x, y)
\end{aligned} 
$$
and therefore, $N^{*}N = NN^{*}$. 
\end{lproof}

\section{Singular Value and Polar Decompositions}

\begin{definition}
A self-adjoint operator $A: X \rightarrow X$ is called \textbf{positive definite} if 
$$(Ax, x) > 0 \qquad \forall x \neq 0$$
and is called \textbf{positive semidefininte} or a \textbf{positive operator} if 
$$(Ax, x) \geq 0 \qquad \forall x \in X$$

We will use $A > 0$ for positive definite operators and $B \geq 0$ for positive semidefinite operators. 
\end{definition}

\begin{theorem}
Let $A = A^{*}$. Then 
\begin{enumerate}
	\item $A > 0$ if and only if all eigenvalues of $A$ are positive. 
	\item $A \geq 0$ if and only if all eigenvalues of $A$ are non-negative. 
\end{enumerate}
\end{theorem}

\begin{proof}
We can rewrite a self-adjoint matrix in some orthonormal basis as a diagonal matrix. Note that a diagonal matrix is positive definite or semidefinite if and only if all its diagonal entries (which are also eigenvalues) are positive or non-negative, respectively.  
\end{proof}

\begin{lemma}
Let $A = A^{*} \geq 0$ be positive semidefinite. There exists a unique positive semidefinite operator $B$ such that $B^{2} = A$. $B$ is called the positive square root of $A$, denoted $\sqrt{A}$.
\end{lemma}

\begin{lproof}
First, we'll prove the existence of $\sqrt{A}$. Let $v_{1}, \cdots, v_{n}$ be an orthonormal basis of eigenvectors of $A$ with corresponding eigenvalues $\lambda_{1}, \cdots, \lambda_{n}$. Since $A \geq 0$, all the eigenvalues are non-negative. In this basis, $A$ is a diagonal matrix with diagonal entries $\lambda_{1}, \cdots, \lambda_{n}$. We can define $B =\sqrt{A}$ as the matrix with diagonal entries $\sqrt{\lambda_{1}}, \cdots, \sqrt{\lambda_{n}}$. Clearly, $B^{2} = A$ and $B = B^{*} \geq 0$. 

To prove uniqueness, suppose there exists $C = C^{*} \geq 0$ such that $C^{2} = A$. Suppose $C$ is a diagonal matrix in orthonormal basis of eigenvectors $u_{1}, \cdots, u_{n}$ with diagonal entries $\mu_{1}, \cdots, \mu_{n}$. Since $A = C^{2}$, $A$ must be in the same basis as $C$, so $C$'s eigenvalues are of the form $\sqrt{\lambda_{1}}, \cdots, \sqrt{\lambda_{n}}$, which means $B = C$. 
\end{lproof}

\begin{definition}
For an operator $A: X \rightarrow Y$, its \textbf{Hermitian square $A^{*} A$} is a positive semidefinite operator in $X$. It is self-adjoint:
$$(A^{*} A)^{*} = A^{*} A^{**} = A^{*} A$$
and positive semidefinite:
$$(A^{*} Ax, x) = (Ax, Ax) = \norm{Ax}^{2} \geq 0 \qquad \forall x \in X$$

Therefore, by the above lemma there exists a unique positive semidefinite square root $R = \sqrt{A^{*} A}$. $R$ is called the \textbf{modulus} of $A$, denoted $\abs{A}$. The modulus denotes how ``big" $A$ is. 
\end{definition}

\begin{theorem}
For linear operator $A: X \rightarrow Y$, 
$$\norm{\: \abs{A} \: x} = \norm{Ax} \qquad \forall x \in X$$ 
\end{theorem}

\begin{proof}
For any $x \in X$, 
$$\norm{\: \abs{A} \: x}^{2} = (\abs{A} x, \abs{A} x) = (\abs{A}^{*} \abs{A} x, x) = (\abs{A}^{2} x, x) = (A^{*} A x, x) = (Ax, Ax) = \norm{Ax}^{2}$$
\end{proof}

\begin{theorem}
$$Ker(A) = Ker(\abs{A}) = (Ran(\abs{A}))^{\perp}$$
\end{theorem}

\begin{proof}
The first equality follows from the previous lemma. The second equality follows from the identity $Ker(T) = (Ran(T^{*}))^{\perp}$ which is true because $\abs{A}$ is self-adjoint so $A = A^{*}$. 
\end{proof}

\subsection{Schmidt Decomposition}

\begin{definition}
Eigenvalues of $\abs{A}$ are called \textbf{singular values} of $A$. In other words, if $\lambda_{1}, \cdots, \lambda_{n}$ are eigenvalues of $A^{*}A$, then $\sqrt{\lambda_{1}}, \cdots, \sqrt{\lambda_{n}}$ are singular values of $A$. 
\end{definition}

Consider an operator $A : X \rightarrow Y$, and let $\sigma_{1}, \cdots, \sigma_{n}$ be the singular values of $A$. Let $\sigma_{1}, \cdots, \sigma_{r}$ be the \textit{non-zero} singular values of $A$. This means $\sigma_{k} = 0 \quad \forall k > r$. 

We know $\sigma_{1}^{2}, \cdots, \sigma_{n}^{2}$ are eigenvalues of $A^{*}A$. Let $v_{1}, \cdots, v_{n}$ be an orthonormal basis of eigenvectors of $A^{*}A$, such that $A^{*}Av_{k} = \sigma_{k}^{2} v_{k}$. This basis exists since $A^{*}A$ is self-adjoint. 

\begin{lemma}
The system 
$$w_{k} := \frac{1}{\sigma_{k}} Av_{k} \qquad \forall k = 1, \cdots, r$$
is an orthonormal system. 
\end{lemma}

\begin{lproof}
For any two eigenvectors $v_{j}, v_{k}$, 
$$\frac{1}{\sigma_{j} \sigma_{k}} (Av_{j}, Av_{k}) = \frac{1}{\sigma_{j} \sigma_{k}} (A^{*} Av_{j}, v_{k}) = \frac{1}{\sigma_{j} \sigma_{k}} (\sigma_{j}^{2} v_{j}, v_{k}) = \frac{1}{\sigma_{j} \sigma_{k}} \sigma_{j}^{2} (v_{j}, v_{k})$$
which evaluates to 0 if $j \neq k$ or $1$ if $j = k$. 
\end{lproof}

\begin{definition}
We can use our formulation of $w_{k} = \frac{1}{\sigma_{k}} Av_{k}$ to rewrite 
$$A = \sum_{k = 1}^{r} \sigma_{k} w_{k} v_{k}^{*}$$
To prove this equality is true, if we multiply both sides by $x$, we can write 
$$Ax = \sum_{k = 1}^{r} \sigma_{k} (x, v_{k}) w_{k}$$
Since $v_{1}, \cdots, v_{n}$ is an orthonormal basis in $X$, we can substitute $x = v_{j}$ to get 
$$\sum_{k = 1}^{r} \sigma_{k} (v_{j}, v_{k}) w_{k} = \sigma_{j} (v_{j}, v_{j}) w_{j} = \sigma_{j} w_{j} = Av_{j} \qquad \forall j = 1, \cdots, r$$
and 
$$\sum_{k = 1}^{r} \sigma_{k} (v_{j}, v_{k}) w_{k} = 0 = Av_{j} \qquad \forall j > r$$
Since the equality holds for the orthonormal basis $v_{1}, \cdots, v_{n}$, it is true. This representation is called \textbf{Schmidt Decomposition}.
\end{definition}

\begin{theorem}
Let
$$A = \sum_{k=1}^{r} \sigma_{k} w_{k} v_{k}^{*}$$
where $\sigma_{k} > 0$, and $v_{1}, \cdots, v_{r}$ and $w_{1}, \cdots, w_{r}$ are orthonormal systems. Then, this representation gives a Schmidt decomposition of $A$. 
\end{theorem}

\begin{proof}
We only need to show that $v_{1}, \cdots, v_{r}$ are eigenvectors of $A^{*}A$. 

Since $w_{1}, \cdots, w_{r}$ is an orthonormal system, 
$$w_{k}^{*} w_{j} = (w_{j}, w_{k})$$ 
which is 1 if $j = k$ and 0 otherwise. 

Therefore, we can write 
$$A^{*} A = \sum_{k=1}^{r} \sum_{j=1}^{r} \sigma_{k} \sigma_{j}    v_{k} w_{k}^{*} w_{j} v_{j}^{*} = \sum_{k=1}^{r} \sigma_{k}^{2} v_{k} v_{k}^{*}$$

If we multiply both sides by $v_{j}$, we get 
$$A^{*} A v_{j} = \sum_{k=1}^{r} \sigma_{k}^{2} v_{k} v_{k}^{*}v_{j} = \sigma_{j}^{2} v_{j}$$

which means $v_{1}, \cdots, v_{r}$ are eigenvectors of $A^{*} A$. 
\end{proof}

A corollary of this result is that if 
$$A = \sum_{k=1}^{r} \sigma_{k} w_{k} v_{k}^{*}$$
is a Schmidt decomposition of $A$, then 
$$A^{*} = \sum_{k=1}^{r} \sigma_{k} v_{k} w_{k}^{*}$$
is a Schmidt decomposition of $A^{*}$. 

\subsection{Singular Value Decomposition}

Let $A : \F^{n} \rightarrow \F^{m}$ and $\sigma_{1}, \cdots, \sigma_{r}$ be the non-zero singular values of $A$, and let 
$$A = \sum_{k=1}^{r} \sigma_{k} w_{k} v_{k}^{*}$$ 
be a Schmidt decomposition of $A$. We can rewrite this in matrix form as 
$$A = \widetilde{W} \, \widetilde{\Sigma} \, \widetilde{V}^{*}$$
where $\widetilde{\Sigma}$ is the diagonal matrix of $\sigma_{1}, \cdots, \sigma_{r}$, and $\widetilde{V}$ and $\widetilde{W}$ are matrices with columns $v_{1}, \cdots, v_{r}$ and $w_{1}, \cdots, w_{r}$, respectively. Note that $\widetilde{V}$ is an $n \times r$ matrix and $\widetilde{W}$ is an $m \times r$ matrix. 

\begin{theorem}
The number of nonzero singular values of $A$ equals the rank of $A$.
\end{theorem}

\begin{proof}
Recall that the singular values of $A$ are the eigenvalues of $A^{*}A$, and the number of nonzero eigenvalues of a square matrix correspond with the matrix's rank.  

Now we'll show $Ker(A) = Ker(A^{*}A)$. For $x \in Ker(A)$, we have $A^{*} (Ax) = 0$. For $x \in Ker(A^{*}A)$, we have $A^{*}Ax = 0$. Left-multiplying by $x^{*}$ gives us $x^{*} A^{*} A x = (Ax)^{*} (Ax) = 0$, so $Ax = 0$. 

Since $Ker(A) = Ker(A^{*}A)$, the rank-nullity theorem asserts that $A^{*}A$ and $A$ have the same rank. 
\end{proof}

If $A$ is invertible, then $m = n = r$, so $\widetilde{V}$ and $\widetilde{W}$ are unitary, and $\widetilde{\Sigma}$ is an invertible diagonal matrix. 

Further, we can always write a similar representation with matrices $V$ and $W$, where $V$ is $n \times n$ and $W$ is $m \times m$. 

To find this representation, we need to complete the bases $v_{1}, \cdots, v_{r}$ and $w_{1}, \cdots, w_{r}$ to orthogonal bases in $\F^{n}$ and $\F^{m}$, respectively. To do this for $V$, we need to only find an orthogonal basis $v_{r+1}, \cdots, v_{n}$ in $Ker(V^{*})$, and we can always do this using Gram-Schmidt. 

Thus, we can write 
$$A = W \Sigma V^{*}$$
where $V$ is $n \times n$, $W$ is $m \times m$, and $\Sigma$ is a ``diagonal" $m \times n$ matrix of singular values. This representation is the \textbf{singular value decomposition} of $A$. The previous representation, $A = \widetilde{W} \, \widetilde{\Sigma} \, \widetilde{V}^{*}$, is the \textbf{reduced} or \textbf{compact} SVD of $A$.

Looking at the singular value decomposition, it is clear that the diagonal entries of $\Sigma$ are the singular values of $A$ (square root of eigenvalues of $A^{*}A$). To see this, we can write 
$$A^{*} A = V \Sigma W^{*} W \Sigma V^{*} = V \Sigma^{2} V^{*}$$
so that spectral decomposition tells us that $\Sigma^{2}$ is the diagonal matrix of eigenvalues and $v_{1}, \cdots, v_{n}$ are the corresponding eigenvectors of $A^{*} A$. Recall that if $\sigma_{k} \neq 0$ then $w_{k} = \frac{1}{\sigma_{k}} Av_{k}$. This means that $A = W \Sigma V^{*}$ can be obtained through the Schmidt decomposition as described earlier. 

Similarly, we can understand the reduced singular value decomposition as a \textit{matrix form} of the Schmidt decomposition for a \textit{non-invertible} matrix $A$. 

\begin{theorem}
If we have the singular value decomposition $A = W \Sigma V^{*}$, we can write the polar decomposition
$$A = (WV^{*}) (V \Sigma V^{*}) = U \abs{A}$$
where $\abs{A} = V \Sigma V^{*}$ and $U = WV^{*}$ is unitary. 
\end{theorem}

\begin{proof}
Notice that 
$$A^{*} A = V \Sigma W^{*} W \Sigma V^{*} = V \Sigma \Sigma V^{*} = V \Sigma V^{*} V \Sigma V^{*} = (V \Sigma V^{*})^{2}$$
This indicates that $\abs{A} = V \Sigma V^{*}$. 

Finally, since $WV^{*}$ is the product of unitary operators, it must also be unitary. Note that this only works if $A$ is square because $W$ and $V$ must also be square so that $WV^{*}$ is defined. 
\end{proof}

I have skipped an in-depth explanation of the polar decomposition in this chapter. It will be revisited in the next chapter. 

\section{Applications of Singular Value Decomposition}

The SVD $A = W \Sigma V^{*}$ can be seen as the matrix $A$ in two different orthonormal bases: $v_{1}, \cdots, v_{n}$ and $w_{1}, \cdots, w_{n}$, ie $\Sigma = [A]_{B, A}$. Because this is with respect to two \textit{different} bases, it doesn't tell us much about $A$'s spectral properties. However, singular values do tell us a lot about the \textit{metric properties} of a linear transformation, as we'll see. 

\subsection{Operator Norm}

Given $A: X \rightarrow Y$, we want to find the maximum of $\norm{Ax}$ on the closed unit ball $B = \{ x \in X : \norm{x} \leq 1 \}$. 

First, consider a diagonal matrix $A$ with non-negative entries. The maximum is exactly the maximal diagonal entry. If there are $r$ non-zero diagonal entries of $A$ with $s_{1}$ being the maximum, we know $\norm{Ax} \leq s_{1} \norm{x}$ because 
$$\norm{Ax}^{2} = \sum_{k=1}^{r} s_{k}^{2} \abs{x_{k}}^{2} \leq s_{1}^{2} \sum_{k=1}^{r} \abs{x_{k}}^{2} = s_{1}^{2} \norm{x}^{2}$$

For the general case, we can consider the SVD, $A = W \Sigma V^{*}$, where $W, V$ are unitary and $\Sigma$ is diagonal with non-negative entries. Since unitary transformations preserve the norm, we know the maximum of $\norm{Ax}$ on $B$ is the maximal diagonal entry of $\Sigma$, which is the maximal singular value of $A$. 

\begin{definition}
The maximum of $\norm{Ax}$ such that $\norm{x} \leq 1$ is called the \textbf{operator norm} of $A$, denoted $\norm{A}$. We can easily verify that it satisfies the 4 properties of the norm (Symmetry, Linearity, Non-negativity, Non-degeneracy). 

One of the main properties of the operator norm is 
$$\norm{Ax} \leq \norm{A} \cdot \norm{x}$$
which follows from realizing $\norm{Ax} - \norm{A} \leq 0$, so the inequality reduces to $0 \leq \norm{x}$. In fact, this often used to define the operator norm, since this property means the operator norm is the smallest number $C \geq 0$ such that 
$$\norm{Ax} \leq C \norm{x} \qquad \forall x \in X$$
\end{definition}

There is another common norm on the space of linear transformations. 

\begin{definition}
The \textbf{Hilbert-Schmidt inner product} on operators $M$ and $N$ is defined as 
$$ (M, N) = trace(N^{*} M)$$
so the induced norm is 
$$\norm{M}_{HS} = \sqrt{(M, M)} = \sqrt{trace(M^{*} M)}$$
\end{definition}

To compare these two norms, let $s_{1}, \cdots, s_{r}$ be non-zero singular values of $A$ with $s_{1}$ being the largest. We know $s_{1}^{2}, \cdots, s_{r}^{2}$ are the non-zero eigenvalues of $A^{*} A$. Since the trace is the sum of the eigenvalues, we know that 
$$\norm{A}_{HS}^{2} = trace(A^{*} A) = \sum_{k=1}^{r} s_{k}^{2}$$
We also know that the operator norm equals its largest singular value ($\norm{A} = s_{1}$), so we conclude that the operator norm of a matrix cannot be more than its Hilbert-Schmidt norm. 

\subsection{Moore-Penrose Pseudoinverse}

Recall that finding a least square solution amounts to solving the normal equation 
$$A^{*} A \hat{x} = A^{*} b$$
If $A = \widetilde{W} \, \widetilde{\Sigma} \, \widetilde{V}^{*}$ is the reduced singular value decomposition of $A$, then 
$$ x_{0} = \widetilde{V} \, \widetilde{\Sigma}^{-1} \, \widetilde{W}^{*} b$$

Indeed, $x_{0}$ is a least square solution of $Ax = b$, which means $Ax_{0} = P_{Ran(A)} b$. To see this, 
$$Ax_{0} = \widetilde{W} \, \widetilde{\Sigma} \, \widetilde{V}^{*} \widetilde{V} \, \widetilde{\Sigma}^{-1} \, \widetilde{W}^{*} b = \widetilde{W} \widetilde{W}^{*} b = P_{Ran(\widetilde{W})} b = P_{Ran(A)} b $$

Our final step uses the fact that $\widetilde{W} \widetilde{W}^{*} = P_{Ran(\widetilde{W})}$: 
$$P_{Ran(\widetilde{W})} = \widetilde{W} (\widetilde{W}^{*} \widetilde{W})^{-1} \widetilde{W}^{*} = \widetilde{W} \widetilde{W}^{*}$$
and that $Ran(\widetilde{W}) = Ran(A)$:
$$
\begin{aligned}
P_{Ran(A)} &= (\widetilde{W} \widetilde{\Sigma} \widetilde{V}^{*}) \big( (\widetilde{W} \widetilde{\Sigma} \widetilde{V}^{*})^{*} (\widetilde{W} \widetilde{\Sigma} \widetilde{V}^{*}) \big)^{-1} (\widetilde{W} \widetilde{\Sigma} \widetilde{V}^{*})^{*} \\
&= \widetilde{W} \widetilde{\Sigma} \widetilde{V}^{*} ( \widetilde{V} \widetilde{\Sigma}^{*} \widetilde{W}^{*} \widetilde{W} \widetilde{\Sigma} \widetilde{V}^{*} )^{-1} \widetilde{V} \widetilde{\Sigma}^{*} \widetilde{W}^{*} \\
&= \widetilde{W} \widetilde{\Sigma} \widetilde{V}^{*} \widetilde{V} \widetilde{\Sigma}^{-1} \widetilde{\Sigma}^{*^{-1}} \widetilde{V}^{*}  \widetilde{V} \widetilde{\Sigma}^{*} \widetilde{W}^{*} \\
&= \widetilde{W} \widetilde{W}^{*} = P_{Ran(\widetilde{W}}
\end{aligned}
$$

\begin{definition}
The operator $A^{\dagger} = \widetilde{V} \, \widetilde{\Sigma}^{-1} \, \widetilde{W}^{*} $ is called the \textbf{Moore-Penrose pseudoinverse} of the operator $A$. In other words, the Moore-Pensrose pseudoinverse gives the least square solution to the equation $Ax = b$. 

The following properties of the Moore-Penrose pseudoinverse can be easily proved: 
\begin{enumerate}
	\item $AA^{\dagger}A = A$
	\item $A^{\dagger} A A^{\dagger} = A^{\dagger}$
	\item $(AA^{\dagger})^{*} = AA^{\dagger}$
	\item $(A^{\dagger} A)^{*} = A^{\dagger} A$
\end{enumerate}
\end{definition}

\begin{lemma}
Let matrix $P$ be self-adjoint ($P^{*} = P = P^{-1}$) and let $P^{2} = P$. Then $P$ is the matrix of an orthogonal projection.
\end{lemma}

\begin{lproof}
$P$ is an orthogonal projection if and only if $(u - Pu, Pu) = 0$. This is equivalent to 
$$(u, Pu) = (Pu, Pu)$$
which is simplified to 
$$(u, Pu) = (u, P^{*}Pu)$$
which holds because $P$ is self-adjoint and $P^{2} = P$. 
\end{lproof}

\begin{theorem}
The operator $A^{\dagger}$ satisfying the above 4 properties is unique. 
\end{theorem}

\begin{proof}
Left and right multiplying property 1 by $A^{\dagger}$ gives us: 
$$(A^{\dagger} A)^{2} = A^{\dagger} A$$ 
and
$$ (AA^{\dagger})^{2} = AA^{\dagger}$$
Properties 3 and 4 mean these two projections are also self-adjoint, so by the previous lemma, they are orthogonal projections. 

Clearly, $Ker(A) \subset Ker(A^{\dagger}A)$. To prove $Ker(A^{\dagger} A) \subset Ker(A)$, suppose we have some $u \in Ker(A^{\dagger} A)$. Right-multiplying property 1 by $u$ gives us $AA^{\dagger}Au = Au$, which simplifies to $A0 = Au$, so $u \in Ker(A)$. Thus, $Ker(A) = Ker(A^{\dagger}A)$, which means $A^{\dagger}A$ is the orthogonal projection onto $Ker(A)^{\perp} = Ran(A^{*})$, 
$$A^{\dagger}A = P_{Ran(A^{*})}$$

Property 1 implies that $AA^{\dagger} y = y$ for any $y \in Ran(A)$, so $Ran(A) \subset Ran(AA^{\dagger})$. Clearly, $RAn(AA^{\dagger}) \subset Ran(A)$, so $AA^{\dagger}$ is the orthogonal projection onto $Ran(A)$, 
$$AA^{\dagger} = P_{Ran(A)}$$

Now we can rewrite property 2 as 
$$P_{Ran(A^{*})} A^{\dagger} = A^{\dagger} \text{ or } A^{\dagger} P_{Ran(A)} = A^{\dagger}$$

Combining these gives us 
$$P_{Ran(A^{*})} A^{\dagger} P_{Ran(A)} = A^{\dagger}$$

This means that for any $b \in Ran(A)$, 
$$x_{0} := A^{\dagger} b = P_{Ran(A^{*})} A^{\dagger} b \in Ran(A^{*})$$
and that 
$$Ax_{0} = AA^{\dagger} b = P_{Ran(A)} b$$
In other words, $x_{0}$ is a least square solution to $Ax = b$. However, since $x_{0} \in Ran(A^{*}) = Ker(A)^{\perp}$, $x_{0}$ is the least square solution of minimal norm, which we know is given by $A^{\dagger} = \widetilde{V} \, \widetilde{\Sigma}^{-1} \, \widetilde{W}^{*}$.
\end{proof}

\textbf{Generalize Moore-Penrose to the SVD: $A = W\Sigma V^{*}$. }

\section{Structure of Orthogonal Matrices}

An orthogonal matrix $U$ with $det(U) = 1$ is called a \textbf{rotation}. The following theorem illustrates why. 

\begin{theorem}
Let $U$ be an orthogonal operator in $\R^{n}$ and let $det(U) = 1$. Then there exists an orthonormal basis $v_{1}, \cdots, v_{n}$ such that the matrix of $U$ in this basis has the \textit{block diagonal form}:
$$\begin{bmatrix}
R_{\varphi 1} & & & & \makebox{\Huge 0} \\
 & R_{\varphi 2} & & & \\ 
 & & \ddots & & \\
 & & & R_{\varphi k} & \\
 \makebox{\Huge 0} & & & & I_{n - 2k}
\end{bmatrix}$$
where $R_{\varphi k}$ are 2-dimensional rotations, 
$$R_{\varphi k} = \begin{bmatrix}
\cos \ \varphi k & -\sin \  \varphi k \\
\sin \ \varphi k & \cos \ \varphi k
\end{bmatrix}$$
\end{theorem}

\begin{proof}
This is a long proof!

We know that if $p$ is a polynomial with real coefficients and $\lambda$ is its complex root ($p(\lambda) = 0$), then $\overline{\lambda}$ is also a root ($p(\overline{\lambda}) = 0$). This means all complex eigenvalues of a real matrix $A$ can be split into pairs $\lambda_{k}, \overline{\lambda_{k}}$. Since we also know eigenvalues of unitary matrices have absolute value 1, we can write any eigenvalue as $\lambda_{k} = \cos a_{k} + i \sin a_{k}$ and $\overline{\lambda_{k}} = \cos a_{k} -i \sin a_{k}$.

We fix a pair of eigenvalues $\lambda$ and $\overline{\lambda}$, and let $u$ be the eigenvector of $\lambda$, so $U u = \lambda u$. This means $U \overline{u} = \overline{\lambda} \overline{u}$. Now, we split $u$ into \textit{real} and \textit{imaginary} parts. Define  
$$x := Re(u) = \frac{u + \overline{u}}{2} \qquad y := Im(u) = \frac{u - \overline{u}}{2i}$$ 
so that $u = x + iy$. Then 
$$Ux = \frac{1}{2} U (u + \overline{u}) = \frac{1}{2} (\lambda u + \overline{\lambda} \overline{u}) = Re(\lambda u)$$ 
and 
$$Uy = \frac{1}{2i} U(u - \overline{u}) = \frac{1}{2i} (\lambda u - \overline{\lambda} \overline{u}) = Im(\lambda u)$$ 
Finally, because $\lambda = \cos a + i \sin a$, we can write 
$$\lambda u = (\cos a + i \sin a)(x + iy) = ((\cos a) x - (\sin a) y) + i((\cos a) y + (\sin a) x)$$
which means 
$$Ux = Re(\lambda u) = (\cos a) x - (\sin a) y \qquad Uy = Im(\lambda u) = (\cos a) y + (\sin a) x$$

In other words, $U$ preserves the subspace spanned by $x$ and $y$, denoted $E_{\lambda}$, and the matrix of the \textit{restriction of} $U$ onto this subspace is the rotation matrix 
$$R_{a} = \begin{bmatrix}
\cos a & \sin a \\
-\sin a & \cos a
\end{bmatrix}$$

Since $u$ and $\overline{u}$ are orthogonal eigenvectors of a unitary matrix, we can use the Pythagorean Theorem to write 
$$\norm{x} = \norm{y} = \frac{\sqrt{2}}{2} \norm{u}$$

It is clear from their definitions that $x$ and $y$ are orthogonal, so they form an orthogonal basis in $E_{\lambda}$. Since we can multiply each vector in a basis by a non-zero constant and not change the matrices of linear transformations, we can assume $\norm{x} = \norm{y} = 1$. 

Now we'll complete the orthonormal system $v_{1} = x, v_{2} = y$ to an orthonormal basis in $\R^{n}$. 

Since $U E_{\lambda} \subset E_{\lambda}$ (because $E_{\lambda}$ is an invariant subspace of $U$), the matrix of $U$ in this basis has the \textit{block triangular form} 
$$\begin{bmatrix}[c|c]
 \: R_{a} \:  & * \\
\hline 
0 & \:  U_{1} \:  \\
\end{bmatrix}$$
where 0 represents the $(n - 2) \times 2$ zero matrix. 

Since the rotation matrix $R_{a}$ is invertible, we can write $UE_{\lambda} E_{\lambda}$. Thus, 
$$U^{*} E_{\lambda} = U^{-1} E_{\lambda} = E_{\lambda}$$
so the matrix $U$ becomes 
$$\begin{bmatrix}[c|c]
 \: R_{a} \:  & 0 \\
\hline 
0 & \:  U_{1} \:  \\
\end{bmatrix}$$

Because we know $U$ is unitary, we can write 
$$I = U^{*}U = \begin{bmatrix}[c|c]
\: I \: & 0 \\
\hline
0 & \: U_{1}^{*} U_{1} \:
\end{bmatrix}$$
and because $U_{1}$ is square, it must also be unitary. 

If $U_{1}$ has complex eigenvalues, we can continue applying this procedure to decrease its size by 2 until we are left with a block with only real eigenvalues. Since real eigenvalues are either 1 or -1, in some orthonormal basis the matrix of $U$ has the form 
$$\begin{bmatrix}
R_{a_{1}} & & & & & \makebox{\Huge 0} \\
 & R_{a_{2}} & & & \\ 
 & & \ddots & & \\
 & & & R_{a_{d}} & \\
 & & & & -I_{r} & \\
 \makebox{\Huge 0} & & & & & I_{l}
\end{bmatrix}$$
where $I_{r}$ and $I_{l}$ are identity matrices of size $r$ and $l$, respectively. Sicne $det(U) = 1$, the multiplicity of eigenvalue -1 must be even. 

Finally, because a negative $2 \times 2$ identity matrix can be interpreted as the rotation through the angle $\pi$, we arrive at the form given in the theorem with $\varphi_{k} = \pi$. 
\end{proof}
\chapter{Outer Product Representation} 

This chapter is largely adapted from Chapter 2 of Nielsen and Chuang. Much of it will be spent providing an additional perspective on the core ideas in the previous two chapters. New concepts, like tensor products, that are introduced in the second half of this chapter will be developed more formally later in the book. 
\section{Dirac Notation} 

We can use inner products to derive a useful representations of linear operators. Before doing so, we'll define a convenient notation. 

\begin{definition}
The \textbf{Dirac notation} represents vectors as 
$$\ket*{\psi}$$ which is also known as a \textbf{ket}. The dual of this same vector (dual spaces will be defined later in the book) is represented as 
$$\bra*{\psi}$$ which is also known as a \textbf{bra}. For now, it suffices to know that the dual of a vector is simply the complex conjugate transpose of that vector. This notations allows us to represent the inner product of two vectors as 
$$\bra*{\phi}\ket*{\psi}$$
\end{definition}

Using this new notation, we can define the outer product. 

\begin{definition}
The \textbf{outer product representation} is a representation of linear operators which uses the inner product. Suppose $\ket*{v}$ is a vector in inner product space $V$, and $\ket*{w}$ is a vector in inner product space $W$. We define $\ket{w} \bra*{v}$ to be the \textit{linear operator} from $V$ to $W$ defined by 
$$\Big(\ket*{w} \bra*{v} \Big) \ket*{v_{1}} \equiv \ket*{w} \bra*{v} \ket*{v_{1}} = \bra*{v} \ket*{v_{1}} \ket*{w}$$
\end{definition}

To make this representation feel less abstract, we consider an important result using the outer product. 

\begin{lemma}[Completeness Relation]
For an orthonormal basis $\ket*{i}$, 
$$\sum_{i} \ket*{i} \bra*{i} = I$$
\end{lemma}

\begin{lproof}
We know that for any $v \in V$ can be written as $v = \sum_{i} v_{i} \ket*{i}$ and that $\bra*{i} \ket*{v} = v_{i}$. Then 

$$\Big( \sum_{i} \ket*{i} \bra*{i} \Big) \ket*{v} = \sum_{i} \ket*{i} \bra*{i} \ket*{v} = \sum_{i} v_{i} \ket*{i} = \ket*{v}$$

Because this is true for all $\ket*{v}$, it must be the identity operator. 
\end{lproof}

One application of the Completeness Relation is the representation of any operator in the outer product notation. Suppose $A: V \rightarrow W$ is a linear operator and $\ket*{v_{i}}$ and $\ket*{w_{j}}$ are orthonormal bases for $V$ and $W$, respectively. By using the Completeness Relation twice, we get
$$A = I_{W} A I_{V} = \sum_{ij} \ket*{w_{j}} \bra*{w_{j}} A \ket*{v_{i}} \bra*{v_{i}} = \sum_{ij} \bra*{w_{j}} A \ket*{v_{i}} \ket*{w_{j}} \bra*{v_{i}}$$
which is the \textbf{outer product representation of $A$}. This equation also shows that $A$ has matrix element $\bra*{w_{j}}A\ket*{v_{i}}$ in the $i$th column and $j$th row, with respect to input basis $\ket*{v_{i}}$ and output basis $\ket*{w_{j}}$.

To gain more familiarity with the outer product representation, we will prove the Cauchy-Schwarz Inequality. 

\begin{theorem}[Cauchy-Schwarz Inequality]
For any two vectors $\ket*{v}$ and $\ket*{w}$, $$\abs{\bra*{v}\ket*{w}}^{2} \leq \bra*{v}\ket*{v} \bra*{w}\ket*{w}$$
\end{theorem}

\begin{proof}
We use Gram-Schmidt to obtain an orthonormal basis $\ket{i}$, such that the first member of the basis is $\ket*{w} / \sqrt{\bra*{w}\ket*{w}}$. Then 
$$
\begin{aligned} 
\bra*{v}\ket*{v} \bra*{w}\ket*{w} &= \sum_{i} \bra*{v}\ket*{i} \bra*{i}\ket*{v} \bra*{w}\ket*{w} \\
&\geq \frac{\bra*{v}\ket*{w} \bra*{w}\ket*{v}}{\bra*{w}\ket*{w}} \bra*{w}\ket*{w} \\ 
&= \bra*{v}\ket*{w} \bra*{w}\ket*{v} = \abs{\bra*{v}\ket*{w}}^{2}
\end{aligned}
$$

where the inequality in the second step follows from the fact that we only use the first member of our basis and thus drop some non-negative terms.
\end{proof}

\section{Diagonalization}

The spectral theorem for normal operators proved in the previous chapter ($N = UDU^{\dagger}$ where $U$ is a unitary matrix of eigenvectors and $D$ is a diagonal matrix of corresponding eigenvalues) states an operator is diagonalizable if and only if it is normal. We can also write the diagonal representation of a normal matrix in outer product representation.  

\begin{theorem}
A diagonal representation for a normal operator $N: V \rightarrow V$ is 
$$N = \sum_{i}^{n} \lambda_{i} \ket*{i} \bra*{i}$$
where $\ket*{i}$ are an orthonormal set of eigenvectors of $V$ with corresponding eigenvalues $\lambda_{i}$. 
\end{theorem}

\begin{proof}
Let $\ket*{e}$ be an orthonormal basis of $N$. Since each basis vector $\ket*{e}$ can be understood as providing a coordinate of $\ket*{v}$ in that basis through $\bra*{e}\ket*{v}$, we know $\bra*{e}\ket*{e}$ is the 0 matrix except with a 1 on the $e$th diagonal entry. Using the spectral theorem, we get 
$$N = UDU^{\dagger} = U \Bigg( \sum_{i=1}^{n} \lambda_{i} \ket*{e}\bra*{e} \Bigg) U^{\dagger} = \sum_{i=1}^{n} \lambda_{i} U \ket*{e} \Bigg( U \ket*{e} \Bigg)^{\dagger}$$

Now we define $U \ket*{e}$ to be the normalized eigenvectors $\ket*{i}$ to get 
$$N = \sum_{i=1}^{n} \lambda_{i} \ket*{i} \bra*{i}$$
\end{proof}

As an example, we will introduce the Pauli $Z = \begin{bmatrix}
1 & 0 \\
0 & -1
\end{bmatrix}$ matrix, which is significant in quantum computation, and write its diagonal representation.  

The characteristic equation $det(A - \lambda I)$ determines that the eigenvalues for $Z$ are 1 and -1. The corresponding eigenvectors are $\begin{bmatrix}
1 \\ 0
\end{bmatrix}$ and $\begin{bmatrix}
0 \\ 1
\end{bmatrix}$, which we will denote $\ket*{0}$ and $\ket*{1}$, respectively. This means our diagonal representation is 
$$Z = \begin{bmatrix}
1 & 0 \\
0 & -1
\end{bmatrix} = \ket*{0}\bra*{0} - \ket*{1}\bra*{1}$$

\section{Adjoints and Hermitian Operators} 

Recall, if $A$ is a linear operator on $V$, then there exists a unique linear operator $A^{\dagger}$ (synonymous with $A^{*}$ in quantum physics notation which we are adopting for this chapter) on $V$ such that for all vectors $\ket*{v}, \ket*{w} \in V$, 
$$(\ket*{v}, A\ket*{w}) = (A^{\dagger}\ket*{v}, \ket*{w})$$

This linear operator is the \textbf{adjoint} or \textbf{Hermitian conjugate} of $A$. Recall earlier that we also defined $\ket*{v}^{\dagger} \equiv \bra*{v}$.

\begin{definition}
A mapping $f: V \rightarrow W$ between complex vector spaces is \textbf{antilinear} if 
$$f(ax + by) = \overline{a} f(x) + \overline{b} f(y)$$

From Wikipedia: ``antilinear maps occur in quantum mechanics in the study of time reversal and in spinor calculus."
\end{definition}

\begin{lemma}[Antilinearity of the adjoint]
$$\Bigg( \sum_{i} a_{i} A_{i} \Bigg)^{\dagger} = \sum_{i} a_{i}^{*} A_{i}^{\dagger}$$
\end{lemma}

\begin{lproof}
On the LHS: we know $(a + b)^{*} = a^{*} + b^{*}$. Thus, $(\sum_{i} a_{i}A_{i})^{\dagger} = \sum_{i} (a_{i}A_{i})^{\dagger}$.

On the RHS: $\sum_{i} a_{i}^{*} A_{i}^{\dagger} = (a_{1} A_{1})^{\dagger} + \cdots + (a_{n} A_{n})^{\dagger} = \sum_{i} (a_{i} A_{i})^{\dagger}$. 
\end{lproof}

Recall, an operator $A$ whose adjoint is itself is known as a \textbf{Hermitian} or \textbf{self-adjoint} operator. 

An interesting usage of Hermitian operators that hint at their importance is the following theorem. 

\begin{theorem}
An arbitrary operator $A$ can be decomposed into the sum $B + iC$ where $B$ and $C$ are Hermitian.
\end{theorem}

\begin{proof}
Define $B = \frac{A + A^{\dagger}}{2}$ and $C = \frac{A - A^{\dagger}}{2i}$. 

To prove $B$ is Hermitian: $B^{\dagger} = \frac{A^{\dagger} + A}{2} = B$. To prove $C$ is Hermitian: $C^{\dagger} = \frac{A^{\dagger} - A}{-2i} = C$. 

Finally, 
$$B + iC = \frac{A + A^{\dagger}}{2} + i \frac{A - A^{\dagger}}{2i} = \frac{A + A}{2} = A$$
\end{proof}

Recall, \textit{projector} operators are Hermitian operators. We can demonstrate this using outer product representation. 

Suppose $W$  is a $k$-dimensional vector \textit{subspace} of the $d$-dimensional vector space $V$. We use Gram-Schmidt to construct an orthonormal basis $\ket*{1}, \cdots, \ket*{d}$ for $V$ such that $\ket*{1}, \cdots, \ket*{k}$ is an orthonormal basis for $W$. Then, we define 
$$P = \sum_{i=1}^{k} \ket*{i} \bra*{i}$$
to be the projector onto subspace $W$. Notice that the Completeness Relation states that if $k = d$, $P = I$, which is exactly what a projector from $V \rightarrow V$ would be. This outer product representation of $P$ makes it clear that $P$ is Hermitian: since $(\ket*{v} \bra*{v})^{\dagger} = (\ket*{v} \bra*{v})$, antilinearity means
$$P^{\dagger} = \Bigg( \sum_{i}^{k} \ket*{i} \bra*{i} \Bigg)^{\dagger} = \sum_{i}^{k} (\ket*{i} \bra*{i}) = \sum_{i}^{k} \ket*{i} \bra*{i} = P$$

The Completeness Relation also makes it clear that the orthogonal complement of $P$ is defined as $Q = I - P$, and that $Q$ is a projector onto the vector space spanned by $\ket*{k + 1}, \cdots, \ket*{d}$. 

Recall, an operator $A$ is \textbf{normal} if $AA^{\dagger} = A^{\dagger}A$. Clearly, a Hermitian operator is also normal. 

Recall, an operator $U$ is \textbf{unitary} if $U^{\dagger}U = I$, so unitary operators are also normal. In the Chapter 5, we proved unitary operators preserve the norm, but we can generalize this to prove that they also preserve inner products. 

\begin{theorem}
Let $U$ be unitary. Then 
$$(U \ket*{v}, U \ket*{w}) = (\ket*{v}, \ket*{w})$$
\end{theorem}

\begin{proof}
$$ (U \ket*{v}, U \ket*{w}) = \bra*{v} U^{\dagger} U \ket*{w} = \bra*{v} I \ket*{w} = \bra*{v}\ket*{w}$$
\end{proof}

This result allows an elegant outer product representation of $U$. Let $\ket*{v_{i}}$ be an orthonormal basis. We define $$\ket*{w_{i}} = U \ket*{v_{i}}$$ so that $\ket*{w_{i}}$ is also an orthonormal basis because unitary operators preserve inner products. Right multiplying both sides by $\bra*{v_{i}}$ and using the Completeness Relation, we get 
$$\ket*{w_{i}} \bra*{v_{i}} = U \ket*{v_{i}} \bra*{v_{i}} \rightarrow U = \sum_{i} \ket*{w_{i}} \bra*{v_{i}}$$

Recall, an operator $A$ such that $(\ket*{v}, A\ket*{v})$ is real and non-negative is called a \textbf{positive operator} or positive semidefinite. If this same inner product is \textit{strictly} positive for all $\ket*{v} \neq 0$, then $A$ is \textbf{positive definite}. 

\begin{theorem}[Hermiticity of positive operators] 
A positive operator $A$ is Hermitian. 
\end{theorem} 

\begin{proof}
Since we can write $A = B + iC$ where $B$ and $C$ are Hermitian, we can use definition of a positive operator to write 
$$
\begin{aligned}
(\ket*{v}, A \ket*{v}) &= (\ket*{v}, B\ket*{v} +iC\ket*{v}) \\ 
&= (\ket*{v}, B\ket*{v}) + (\ket*{v}, iC\ket*{v}) \\
&= (B\ket*{v}, \ket*{v}) + (-iC\ket*{v}, \ket*{v}) 
\end{aligned}
$$

This implies that $A = B - iC$, so  
$$A^{\dagger} = (B - iC)^{\dagger} = B^{\dagger} + i C^{\dagger} = B + iC = A$$
\end{proof}

\section{Tensor Products}

Tensor products are probably the most unintuitive Linear Algebra concept I've come across thus far. This section is a very top-level overview of the subject; later chapters will cover this topic in more detail. 

\begin{definition}
Suppose $V$ and $W$ are $m$ and $n$ dimensional vector spaces, respectively. Then $V \otimes W$ is the \textbf{tensor product} of these two spaces and is an $mn$ dimensional vector space. The elements of $V \otimes W$ are linear combinations of \textbf{tensor products} $\ket*{v} \otimes \ket*{w}$. If $\ket*{i}$ and $\ket*{j}$ are orthonormal bases for spaces $V$ and $W$, then $\ket*{i} \otimes \ket*{j}$ is a basis for $V \otimes W$. 
\end{definition}

The tensor product satisfies the following properties: 
\begin{enumerate}
	\item For scalar $z$, $\ket*{v} \in V$, and $\ket*{w} \in W$, $$z (\ket*{v} \otimes \ket*{w}) = (z\ket*{v}) \otimes \ket*{w} = \ket*{v} \otimes (z\ket*{w})$$
	\item For $\ket*{v_{1}}, \ket*{v_{2}} \in V$ and $\ket*{w} \in W$, $$(\ket*{v_{1}} + \ket*{v_{2}}) \otimes \ket*{w} = \ket*{v_{1}} \otimes \ket*{w} + \ket*{v_{2}} \otimes \ket*{w}$$
	\item For $\ket*{v} \in V$ and $\ket*{w_{1}}, \ket*{w_{2}} \in W$, $$\ket*{v} \otimes (\ket*{w_{1}} + \ket*{w_{2}}) = \ket*{v} \otimes \ket*{w_{1}} + \ket*{v} \otimes \ket*{w_{2}}$$
\end{enumerate}

If $A$ and $B$ are linear operators on $V$ and $W$, respectively, then we can define a linear operator $A \otimes B$ on $V \otimes W$ as 
$$(A \otimes B)(\ket*{v} \otimes \ket*{w}) \equiv A\ket*{v} \otimes B\ket*{w}$$

We can extend this definition of $A \otimes B$ to ensure linearity of $A \otimes B$ so 
$$(A \otimes B) \Bigg( \sum_{i} \alpha_{i} \ket*{v_{i}} \otimes \ket*{w_{i}} \Bigg) \equiv \sum_{i} \alpha_{i} A \ket*{v_{i}} \otimes B \ket*{w_{i}}$$

An arbitrary linear operator $C$ mapping $V \otimes W$ to $V^{'} \otimes W^{'}$ can be represented as a linear combination of tensor products of operators mapping $A: V \rightarrow V^{'}$ and $B: W \rightarrow W^{'}$, 
$$C = \sum_{i} c_{i} A_{i} \otimes B_{i}$$

We can use the inner products on $V$ and $W$ to define an inner product on $V \otimes W$, 
$$\Bigg( \sum_{i} \alpha_{i} \ket*{v_{i}} \otimes \ket*{w_{i}} , \sum_{j} \beta_{j} \ket*{v_{j}^{'}} \otimes \ket*{w_{j}^{'}} \Bigg) \equiv \sum_{ij} \alpha_{i}^{*} b_{j} \bra*{v_{i}}\ket*{v_{j}^{'}} \bra*{w_{i}}\ket*{w_{j}^{'}}$$

To make all this less abstract, we consider a matrix representation known as the \textit{Kronecker product}. Suppose $A$ is an $m \times n$ matrix and $B$ is a $p \times q$ matrix. Then 
$$A \otimes B \equiv \overbrace{\left.\begin{bmatrix}
A_{11}B & A_{12}B & \cdots & A_{1n}B \\ 
A_{21}B & A_{22}B & \cdots & A_{2n}B \\
\vdots & \vdots & & \vdots \\ 
A_{m1}B & A_{m2}B & \cdots & A_{mn}B
\end{bmatrix} \right\}}^{nq} mp
$$

where $A_{11}B$ denotes the $p \times q$ submatrix $B$ multiplied by the constant $A_{11}$. 

For example, 
$$\begin{bmatrix}
1 \\ 2
\end{bmatrix} \otimes \begin{bmatrix}
2 \\ 3
\end{bmatrix} = \begin{bmatrix}
1 \times 2 \\
1 \times 3 \\
2 \times 2 \\
2 \times 3
\end{bmatrix} = \begin{bmatrix}
2 \\ 
3 \\ 
4 \\ 
6
\end{bmatrix}$$

and the tensor product of Pauli $X$ and $Y$ is 
$$X \otimes Y = \begin{bmatrix}
0 \cdot Y & 1 \cdot Y \\
1 \cdot Y & 0 \cdot Y
\end{bmatrix} = \begin{bmatrix}
0 & 0 & 0 & -i \\
0 & 0 & i & 0 \\
0 & -i & 0 & 0 \\
i & 0 & 0 & 0
\end{bmatrix}$$

The notation $\ket*{\psi}^{\otimes k}$ denotes $\ket*{\psi}$ tensored with itself $k$ times. 

Do problems from Nielsen and Chuang. 

\section{Operator Functions}

Given a function $f: C \rightarrow C$, we can define a corresponding matrix function. Let $A = \sum_{a} a \ket*{a} \bra*{a}$, then 
$$f(A) \equiv \sum_{a} f(a) \ket*{a} \bra*{a}$$

We can use this construction to define the square root of a positive operator, the logarithm of a positive-definite operator, or the exponential of a normal operator. 

One important matrix function is \textit{trace}. Since we know trace is \textit{cyclic} ($tr(AB) = tr(BA)$) then $tr(UAU^{\dagger}) = tr(U^{\dagger}UA) = tr(A)$. 

Suppose $\ket*{\psi}$ is a unit vector and $A$ is an arbitrary operator. To evaluate $tr(A\ket*{v} \bra*{v})$ use Gram-Schmidt to extend $\ket*{\psi}$ to an orthonormal basis $\ket*{i}$ which includes $\ket*{\psi}$ as the first element. Then 
$$tr(A \ket*{\psi} \bra*{\psi}) = \sum_{i} \bra*{i}A\ket*{\psi} \bra*{\psi}\ket*{i} = \bra*{\psi}A\ket*{\psi}$$

\section{Commutator and Anti-commutator} 

\begin{definition}
The \textbf{commutator} between two operators $A$ and $B$ is defined to be 
$$[A, B] \equiv AB - BA$$

If $[A, B] = 0$, then we say $A$ \textbf{commutes} with $B$. The \textbf{anti-commutator} is defined to be 
$$\{A, B\} \equiv AB + BA$$

We say $A$ \textit{anti-commutes} with $B$ if $\{ A, B\} = 0$.
\end{definition}

Many important properties of pairs of operators can be deduced from their commutator or anti-commutator. A useful relation is the connection between the commutator and the property of being able to \textit{simultaneously diagonalize} Hermitian operators $A$ and $B$, that is, write $A = \sum_{i} a_{i} \ket*{i} \bra*{i}, B = \sum_{i} b_{i} \ket*{i} \bra*{i}$, where $\ket*{i}$ is a common set of eigenvectors for $A$ and $B$. 

\begin{theorem}[Simultaneous diagonalization theorem]
Suppose $A$ and $B$ are Hermitian operators. Then $[A, B] = 0$ if and only if there exists an orthonormal basis such that both $A$ and $B$ are diagonal with respect to that basis. 
\end{theorem}

\begin{proof}
Proving if is simple, since if $A= UA_{1}U^{-1}$ and $B = UB_{1}U^{-1}$, then clearly $AB - BA = 0$. 

To prove only if, let $\ket*{a, j}$ be an orthonormal basis for the eigenspace $V_{a}$ of $A$ with eigenvalue $a$; the index $j$ being used to label possible degeneracies. Because we know $AB = BA$, we can write
$$AB \ket*{a, j} = BA \ket*{a, j} = aB \ket*{a, j}$$

which means $B\ket*{a, j}$ is an element of the eigenspace $V_{a}$. Let $P_{a}$ denote the projector onto $V_{a}$ and define $B_{a} = P_{a} BP_{a}$. Since projectors are Hermitian, we know $B_{a}$ is also Hermitian so it must have a spectral decomposition in terms of an orthonormal set of eigenvectors that span $V_{a}$. Let's call these eigenvectors $\ket*{a, b, k}$, where $a$ and $b$ label the eigenvalues of $A$ and $B_{a}$, and $k$ indexes degeneracies. Note that $B \ket*{a, b, k}$ is in $V_{a}$, so $B \ket*{a, b, k} = P_{a} B \ket*{a, b, k}$. Additionally, we know $P_{a} \ket*{a, b, k} = \ket*{a, b, k}$, so 
$$B \ket*{a, b, k} = P_{a} B P_{a} \ket*{a, b, k} = b \ket*{a, b, k}$$

This means $\ket*{a, b, k}$ is an eigenvector of $B$ with eigenvalue $b$, and therefore $\ket*{a, b, k}$ is a spanning orthonormal set of eigenvectors of both $A$ and $B$. Thus, $A$ and $B$ are simultaneously diagonalizable. 
\end{proof}

Another use of the commutator and anti-commutator is the following identity: 
$$AB = \frac{[A, B] + \{A, B\}}{2}$$

\begin{lemma}
Suppose $[A, B] = 0, \{A, B\} = 0,$ and $A$ is invertible. Then $B$ is 0. 
\end{lemma}

\begin{lproof}
Using the identity we just defined, we know $AB = \frac{0 + 0}{2}$. Since $A$ is invertible we left multiply both sides by $A^{-1}$ to get $B = 0$. 
\end{lproof}

Two more easily verifiable identities: 
\begin{enumerate}
	\item $[A, B]^{\dagger} = [B^{\dagger}, A^{\dagger}]$
	\item $[A, B] = - [B, A]$ 
	\item If $A, B$ are Hermitian, then $i[A, B]$ is also Hermitian. 
\end{enumerate}

\section{Polar Decomposition}

We don't understand \textit{general} linear operators incredibly well, but we do have a good understanding of \textit{unitary} and \textit{positive} operators. The polar and singular value decompositions allow us to take advantage of this by breaking general linear operators into products of unitary and positive operators. 

\begin{theorem}[Polar decomposition]
Let $A$ by a linear operator on $V$. Then there exists unitary $U$ and positive operators $J$ and $K$ such that 
$$A = UJ = KU$$
where $J = \sqrt{A^{\dagger}A}$ and $K = \sqrt{AA^{\dagger}}$. Additionally, if $A$ is invertible, then $U$ is unique.
\end{theorem}

\begin{proof}
We know $J$ (which is the modulus of $A$) is a positive operator (which we know are normal), so we can give it a spectral decomposition, 
$$J = \sum_{i} \lambda_{i} \ket*{i} \bra*{i} \qquad (\lambda_{i} \geq 0)$$

Define $\ket*{\psi_{i}} = A \ket*{i}$, so that $\bra*{\psi_{i}}\ket*{\psi_{i}} = \lambda_{i}^{2}$. 

Now consider only the positive eigenvalues. With these eigenvalues, define $\ket*{e_{i}} = \ket*{\psi_{i}} / \lambda_{i}$, so the $\ket*{e_{i}}$ are normalized and are orthogonal, since if $i \neq j$, then $\bra*{e_{i}}\ket*{e_{j}} = \bra*{i}A^{\dagger}A\ket*{j} / \lambda_{i} \lambda_{j} = 0$. 

Now we use Gram-Schmidt to extend the orthonormal set $\ket*{e_{i}}$ to form an orthonormal basis, which we'll also call $\ket*{e_{i}}$. 

We define the unitary operator $U = \sum_{i} \ket*{e_{i}} \bra*{i}$. When $\lambda_{i} \neq 0$, we have $UJ\ket*{i} = \lambda_{i} \ket*{e_{i}} = \ket*{\psi_{i}} = A \ket*{i}$. When $\lambda_{i} = 0$, we have $UJ \ket*{i} = 0 = \ket*{\psi_{i}}$. Since $A$ and $UJ$ agree on the basis $\ket*{i}$, we know $A = UJ$. 

$J$ is unique since left-multiplying $A = UJ$ by $A^{\dagger} = JU^{\dagger}$ gives $A^{\dagger}A = J^{2}$, from which we get $J = \sqrt{A^{\dagger}A}$. The equation $A = UJ$ also means that if $A$ is invertible, so is $J$, which means $U$ is uniquely determined by $U = AJ^{-1}$. 

Proving the right polar decomposition $A = KU$ follows since $$A = UJ = UJU^{\dagger}U = KU$$
where $K = UJU^{\dagger}$ is a positive operator. And since $AA^{\dagger} = KUU^{\dagger}K = K^{2}$, we have $K = \sqrt{AA^{\dagger}}$. 
\end{proof} 

\section{Singular Value Decomposition} 

The singular value decomposition combines polar decomposition and the spectral theorem. 

\begin{theorem}[Singular value decomposition]
Let $A$ be a linear operator on $V$. Then there exist unitary matrices $U$ and $V$, and a diagonal matrix $D$ with non-negative entries such that $$A = UDV$$
The diagonal entries of $D$ are called the \textit{singular values} of $A$. 
\end{theorem}

\begin{proof}
By the polar decomposition, $A = SJ$, for unitary $S$ and positive operator $J$. By the spectral theorem, $J = TDT^{\dagger}$, for unitary $T$ and diagonal $D$ with non-negative entries. 

Now we just set $U = ST$ and $V = T^{\dagger}$ to get 
$$A = SJ = STDT^{\dagger} = UDV$$
\end{proof}
\chapter{Bilinear and Quadratic Forms - WIP}

\section{Main Definitions}

\begin{definition}
A \textbf{bilinear form} on $\R^{n}$ is a function $L(x, y)$ of arguments $x, y \in \R^{n}$ which is linear in each argument. That is 
\begin{enumerate}
	\item $L(\alpha x_{1} + \beta x_{2}, y) = \alpha L(x_{1}, y) + \beta L(x_{2}, y)$
	\item $L(x, \alpha y_{1} + \beta y_{2}) = \alpha L(x, y_{1}) + \beta L(x, y_{2})$
\end{enumerate}

If $x = \begin{bmatrix}
x_{1} \\
\vdots \\
x_{n}
\end{bmatrix}$ and $y = \begin{bmatrix}
y_{1} \\
\vdots \\
y_{n}
\end{bmatrix}$, a bilinear form can be written as 
$$L(x, y) = \sum_{j, k = 1}^{n} a_{jk} x_{k} y_{j}$$
or in matrix form 
$$L(x, y) = (Ax, y)$$
where the matrix $A$ is uniquely determined by the bilinear form $L$.
\end{definition}

\begin{definition}
There are many definitions of a quadratic form. 

One can define a quadratic form on $\R^{n}$ as the ``diagonal" of a bilinear form $L$, that is, any quadratic form $Q$ is defined as $Q[x] = L(x, x) = (Ax, x)$. 

Another definition is to say a quadratic form is a homogeneous second degree polynomial, that is, $Q$ is a polynomial of $n$ variables $x_{1}, \cdots, x_{n}$ having only terms of degree 2 (only terms $ax_{k}^{2}$ and $cx_{j} x_{k}$ are allowed).

There are infinitely many ways to write a quadratic form as $Q[x] = (Ax, x)$, but if we require $A$ to be \textit{symmetric} ($A^{T} = A$), then $A$ will be unique. 

Any \textbf{quadratic form $Q[x]$} on $\R^{n}$ admits a unique representation $Q[x] = (Ax, x)$ where $A$ is a real symmetric matrix. 

We can extend this definition to $\C^{n}$ by taking the self-adjoint transformation, $A = A^{*}$ and defining $Q[x] = (Ax, x)$. Unless explicitly noted, all our theorems will be true in the complex case as well. 

\end{definition}

The only essential difference is that in the complex case we have no choice: a real quadratic form corresponds to a Hermitian matrix. 

\begin{theorem}
$(Ax, x)$ is real for all $x \in \C^{n}$ if and only if $A = A^{*}$.  
\end{theorem}

\begin{proof}
To prove if: 
$$(Ax, x) = (x, A^{*} x) = (x, Ax) = \overline{(Ax, x)}$$

To prove only if: Consider the expression $(A (x + zy), x + zy)$. We assume it is real for all $z \in \C$. Now we can write 
$$
\begin{aligned}
\Big( A (x + zy), x + zy \Big) &= ( Ax + Azy, x + zy) \\
&= \overline{(x, Ax)} + \overline{(zy, Ax)} + \overline{(x, Azy)} + \overline{(zy, Azy)} \\ 
&= (Ax, x) + \overline{z} (Ax, y) + z (Ay, x) + z \overline{z} (Ay, y)
\end{aligned}
$$

We know that the final sum must be real. Since only the middle two terms can contribute to an imaginary solution, we know 
$$\overline{z} (Ax, y) + z (Ay, x) \in \R$$
Because $z$ can be any complex number, we know $z \neq \overline{z}$, so for the imaginary parts to cancel, we must have
$$(Ax, y) = (Ay, x)$$
and because we know $(Ax, y) = (x, A^{*} y)$, this means that $A^{*} = A$. 
\end{proof}

\section{Diagonalization of Quadratic Forms}

Quadratic forms are common; they appear in the study of curves in $\R^{2}$ and some surfaces in $\R^{3}$, for example. If we are given a set in $\R^{n}$ defined by $Q[x] = 1$, where $Q$ is some quadratic form, we might want to understand the structure of this set. We will try to do so using a change of variables. 

\subsection{Orthogonal Diagonalization}

Suppose we have a quadratic form $Q[x] = (Ax, x)$ in $\F^{n}$. We define $y = S^{-1} x$, where $S$ is any invertible $n \times n$ matrix. Now, we have
$$Q[x] = Q[Sy] = (ASy, Sy) = (S^{*} AS y, y)$$
so when written in variable $y$, the quadratic form has matrix $S^{*}AS$. 

Now our goal is to find an invertible matrix $S$ such that $S^{*} AS$ is diagonal. Using diagonalization, we can write $A = UDU^{*}$ because $A$ is symmetric, which means it's unitary and for unitary matrices $U^{*} = U^{-1}$. Then, we have $D = U^{*} AU$, so in $y = U^{-1} x$, the quadratic form has diagonal matrix. 

Geometrically, the columns of $U$ form an orthonormal basis in $\F^{n}$, which we'll call $B$. The change of coordinate matrix $[I]_{S, B}$ from $B$ to the standard basis is exactly $U$. Since $y = U^{-1} x$, the coordinates $y_{1}, \cdots, y_{n}$ can be interpreted as coordinates of the vector $x$ in the new basis $B$. So, orthogonal diagonalization allows us to visualize the set $Q[x] = 1$ very well if we can visualize it for diagonal matrices. 

\textbf{Example:} Consider $Q[x, y] = 2x^{2} + 2y^{2} + 2xy$. We want to describe the set of points $(x, y) \in \R^{2} : Q(x, y) = 1$. 

The matrix $A$ of $Q$ is 
$$A = \begin{bmatrix}
2 & 1 \\
1 & 2
\end{bmatrix}$$ 
After orthogonally diagonalizing $A$, we can write 
$$A = U \begin{bmatrix}
3 & 0 \\
0 & 3
\end{bmatrix} U^{*} \qquad \text{ where } U = \frac{1}{\sqrt{2}} \begin{bmatrix}
1 & -1 \\
1 & 1
\end{bmatrix}$$
which means
$$U^{*} AU = \begin{bmatrix}
3 & 0 \\
0 & 1
\end{bmatrix} = D$$

This tells us the set $\{y : (Dy, y) = 1\}$ is the ellipse with half-axes $\frac{1}{\sqrt{3}}$ and 1. Thus, the set $\{x \in \R^{2}: (Ax, x) = 1 \}$ is the same ellipse but in the basis $\begin{bmatrix}
\frac{1}{\sqrt{2}} \\
\frac{1}{\sqrt{2}}
\end{bmatrix}, \begin{bmatrix}
-\frac{1}{\sqrt{2}} \\
\frac{1}{\sqrt{2}}
\end{bmatrix}$, or in other words, the same ellipse rotated $\frac{\pi}{4}$. 

\subsection{Non-orthogonal Diagonalization} 

Orthogonal diagonalization requires computing eigenvalues and eigenvectors which can be hard to do for large $n$ without computers. Non-orthogonal diagonalization requires only finding an invertible $S$ without requiring $S^{-1} = S^{*}$ such that $D = S^{*}AS$ is diagonal, which is much easier to do. We will cover two of the most used methods of non-orthogonal diagonalization. 

\subsubsection{Diagonalization by Completion of Squares} 

We will demonstrate this method on real quadratic forms (forms on $
R^{n}$), but it can also be used in the complex case. 

Consider again the quadratic form of two variables $Q[x] = 2x_{1}^{2} + 2x_{1} x_{2} + 2x_{2}^{2}$ (the same equation as before except we use $x_{1}, x_{2}$ instead of $x, y$). Since 
$$2 \Big( x_{1} + \frac{1}{2} x_{2} \Big)^{2} = 2 \Big( x_{1}^{2} + 2 x_{1} \frac{1}{2} x_{2} + \frac{1}{4} x_{2}^{2} \Big) = 2x_{1}^{2} + 2x_{1}x_{2} + \frac{1}{2} x_{2}^{2}$$
where the first two terms coincide with the first two terms of $Q$, we can write 
$$Q[x] = 2 \Big( x_{1} + \frac{1}{2} x_{2} \Big)^{2} + \frac{3}{2} x_{2}^{2} = 2y_{1}^{2} + \frac{3}{2} y_{2}^{2}$$
where $y_{1} = x_{1} + \frac{1}{2} x_{2}$ and $y_{2} = x_{2}$. We can use this same method for quadratic forms of more than 2 variables. Note that we can always split a product of two variables into a corresponding square using the identity, 

$$4 x_{1} x_{2} = (x_{1} + x_{2})^{2} - (x_{1} - x_{2})^{2}$$

\subsubsection{Diagonalization Using Row/Column Operations}

Our second method is to perform row operations on the matrix $A$ of the quadratic form. Unlike normal Gauss-Jordan row reduction, after each row operation, we need to perform the same column operation (because we want to ensure $S^{*} AS$ is diagonal). 

\textbf{Maybe insert an example. }

To understand why this works, realize that a row operation corresponds to left multiplying by an elementary matrix while a column operation is equivalent to right multiplying by the transpose of the same elementary matrix. Thus, performing row operations $E_{1}, \cdots, E_{n}$ along with the same column operations gives us 
$$E_{n} \cdots E_{1} A E_{1}^{*} \cdots E_{n}^{*} = EAE^{*}$$

\textbf{Prove this including the complex case. }

As for the identity matrix in the right side of the augmented matrix, we only performed row operations on it so we have 
$$E_{n} \cdots E_{1} I = EI = E$$

Now if we set $E^{*} = S$, we know $S^{*}AS$ is a diagonal matrix, and the matrix $E = S^{*}$ is the ``right half" of the transformed augmented matrix. 

A tricky operation to implement is the swapping of two rows. Consider 
$$A = \begin{bmatrix}
0 & 1 \\
1 & 0
\end{bmatrix}$$
Swapping rows 1 and 2 would diagonalize the matrix, but we would also be required to swap columns 1 and 2, which means we end up with the original matrix. A simple idea is to use row operations to get a non-zero entry on the diagonal. For example, 
$$\begin{bmatrix}[cc|cc]
0 & 1 & 1 & 0 \\
1 & 0 & 0 & 1
\end{bmatrix} \rightarrow \begin{bmatrix}[cc|cc] 
1 & 1 & 1 & \frac{1}{2} \\
1 & 0 & 0 & 1
\end{bmatrix} \rightarrow \begin{bmatrix}[cc|cc] 
1 & 0 & 1 & \frac{1}{2} \\
0 & -1 & -1 & \frac{1}{2}
\end{bmatrix}
$$
where the first step is adding half of row 2 to row 1 (and then the corresponding column operation of adding half of column 2 to column 1), and the second step is subtracting row 1 from row 2 (and then subtracting column 1 from column 2). 

\section{Silvester's Law of Inertia} 

We now know there are many different ways to diagonalize a quadratic form. For example, if we have a diagonal matrix $D$, we can take another diagonal matrix $S$ and transform $D$ to 
$$S^{*} DS = \text{diag}\{ s_{1}^{2} \lambda_{1}, \cdots, s_{n}^{2} \lambda_{n} \}$$

This transformation changes the diagonal entries of $D$, but it does not change the \textit{signs} of the entries. The following says this is always the case. 

\begin{theorem}[Silvester's Law of Inertia]
For a Hermitian matrix $A$ (for a quadratic form $Q[x] = (Ax, x)$) and any of its diagonalizations $D = S^{*} AS$, the number of positive, negative, and zero diagonal entries of $D$ depends only on $A$, but not on a particular choice of diagonalization. 
\end{theorem}

We will need some more help to prove this. 

\begin{definition}
Given an $n \times n$ Hermitian matrix $A = A^{*}$ (a quadratic form $Q[x] = (Ax, x)$ on $\F^{n}$), we call a subspace $E \subset \F^{n}$ \textbf{positive} if 
$$(Ax, x) > 0$$ 
for all $x \in E, x \neq 0$. To emphasize the role of $A$ we will say $A$-positive. There are similar definitions for $A$-negative and $A$-neutral. 
\end{definition}

\begin{lemma}
Let $D$ be a diagonal matrix. Then the number of positive and negative diagonal entries of $D$ coincides with the maximal dimensions of a $D$-positive and $D$-negative subspace, respectively. 
\end{lemma}

\begin{lproof}
By rearranging the standard basis in $\F^{n}$, we can assume that the positive diagonal entries of $D$ are the first $r_{+}$ diagonal entries. 

Now consider the subspace $E_{+}$ spanned by these $r_{+}$ coordinate vectors $e_{1}, \cdots, e_{r_{+}}$. Clearly, $E_{+}$ is a $D$-positive subspace with dimension $r_{+}$. 

Now we'll show that for any other $D$-positive subspace $E$, we have $\Dim(E) \leq r_{+}$. Consider the orthogonal projection $P$ onto $E_{+}$, 
$$P[x_{1}, \cdots, x_{n}]^{T} = [x_{1}, \cdots, x_{r_{+}}, 0, \cdots, 0]^{T}$$

Now define an operator $T: E \rightarrow E_{+}$ by 
$$Tx = Px \qquad \forall x \in E$$

In other words, since $P$ is defined on the whole space, $T$ is the restriction of $P$ to domain $E$ and target space $E_{+}$. 

Now we'll find the null space of $T$ so that we can apply the Rank Theorem. For any $x$ such that $Tx = Px = 0$, we know $x_{1} = \cdots = x_{r_{+}} = 0$, so 
$$(Dx, x) = \sum_{r_{+} + 1}^{n} \lambda_{k} x_{k}^{2} \leq 0$$
since $\lambda_{k} \leq 0$ for $k > r_{+}$. Because $x$ belongs to $D$-positive subspace $E$, the inequality only holds for $x = 0$, so $Ker(T) = \{0\}$. 

We know $Rank(T) = \Dim(Ran(T)) \leq \Dim(E_{+}) = r_{+}$ since $Ran(T) \subset	E_{+}$. By the Rank Theorem, $\Dim(Ker(T)) + Rank(T) = \Dim(E)$. But we just proved $\Dim(Ker(T)) = 0$ which means 
$$\Dim(E) = Rank(T) \leq r_{+}$$

To prove the lemma for negative entries, we just prove the above for $-D$. 
\end{lproof}

This result proves the positive and negative diagonal entries of $D$ coincide with maximal dimensions of $D$-positive and $D$-negative subspaces. The following proves Silvester's Law of Inertia. 

\begin{theorem}
Let $A$ be an $n \times n$ Hermitian matrix, and let $D = S^{*} AS$ be its diagonalization by an invertible matrix $S$. Then the number of positive and negative diagonal entries of $D$ coincides with the maximal dimensions of an $A$-positive and $A$-negative subspace, respectively. 
\end{theorem}

\begin{proof}
Since $D = S^{*}AS$ is a diagonalization of $A$, we know 
$$(Dx, x) = (S^{*}ASx, x) = (ASx, Sx)$$
which means that for any $D$-positive subspace $E$, the subspace $SE$ is an $A$-positive subspace. The same identity implies that for any $A$-positive subspace $F$, the subspace $S^{-1}F$ is $D$-positive. 

Since $S$ is an invertible transformation, $\Dim(E) = \Dim(SE)$ and $\Dim(F) = \Dim(S^{-1}F)$. Thus, for any $D$-positive subspace $E$, we can find an $A$-positive subspace ($SE$ for example) of the same dimension, and for any $A$-positive subspace $F$, we can find a $D$-positive subspace ($S^{-}F$ for example) of the same dimension. Thus, the maximal possible dimensions of an $A$-positive and a $D$-positive subspace coincide, so by the above lemma, the theorem is proved. The same reasoning supports $A$-negative subspaces. 
\end{proof}

\section{Minimax Characterization of Eigenvalues and Silvester's Criterion of Positivity}

\begin{definition}
A quadratic form is called 
\begin{itemize}
	\item \textbf{positive definite} if $Q[x] > 0 \quad \forall x \neq 0$
	\item \textbf{positive semidefinite} if $Q[x] \geq 0$
	\item \textbf{negative definite} if $Q[x] < 0 \quad \forall x \neq 0$ 
	\item \textbf{negative semidefinite} if $Q[x] \leq 0$ 
	\item \textbf{indefinite} if it takes on both positive and negative values 
\end{itemize}
\end{definition}

A Hermitian matrix $A = A^{*}$ is called positive definite if the corresponding quadratic forms $Q[x] = (Ax, x)$ is positive definite. 

\begin{theorem}
Any Hermitian matrix $A$ is positive definite if and only if its eigenvalues are all positive.
\end{theorem}

\begin{proof}
By orthogonal diagonalization, we know there is a basis in which $A$ is diagonal, and it is clear that a diagonal matrix is positive definite if its eigenvalues are all positive. 
\end{proof}

The above reasoning also applies to positive semidefinite, negative definite, negative semidefinite, and indefinite quadratic forms. 

In fact, because of Silvester's Law of Inertia, we don't need to compute eigenvalues. We can just find a non-orthogonal diagonalization and look at those diagonal entries. 

\subsection{Silvester's Criterion of Positivity}

We'll begin by looking for a simple requirement for positivity. Let's analyze the matrix 
$$\begin{bmatrix}
a & b \\
\overline{b} & c
\end{bmatrix}$$
This matrix is positive definite if and only if $a > 0$ and $det(A)= ac - \abs{b}^{2} > 0$. To see this, note that if the above conditions are met, then $c > 0$, so $trace(A) = a + c > 0$. If $\lambda_{1}, \lambda_{2}$ are eigenvalues of $A$, then their product is positive ($det(A) > 0$) and their sum is positive($trace(A) > 0$), which is only possible if both eigenvalues are positive. \textbf{Prove only if direction.}

\begin{theorem}[Silvester's Criterion of Positivity] 
A matrix $A = A^{*}$ is positive definite if and only if 
$$det(A_{k}) > 0 \qquad \forall k = 1, \cdots, n$$
where $A_{k}$ denotes the upper left $n \times n$ submatrix of $A$. 
\end{theorem}

\begin{proof}
If $A > 0$, then $A_{k} > 0 \quad \forall k$, because all of $A$'s eigenvalues are positive. 
\textbf{Finish proof!}
\end{proof}

\subsection{Minimax Characterization of Eigenvalues}

We will use the term \textbf{codimension} to denote the dimension of the orthogonal complement. For a subspace $E \subset X$, if $\Dim(X) = n$, then $\Codim(E) + \Dim(E) = n$. 

\begin{theorem}[Minimax Characterization of Eigenvalues]
Let $A = A^{*}$ be an $n \times n$ matrix, and let $\lambda_{1}, \cdots, \lambda_{n}$ be its eigenvalues in decreasing order. Then, 
$$\lambda_{k} = \max_{E: \Dim(E) = k} \quad \min_{x \in E: \norm{x} = 1} \quad (Ax, x) \quad = \quad \min_{F: \Codim(F)=k-1} \quad \max_{x \in F: \norm{x} = 1} \quad (Ax, x)$$

\end{theorem}

The first expression considers all subspaces $E$ with dimension $k$, and then for each subspace, finds the $x$ with norm 1 that results in the minimum $(Ax, x)$. Now, out of the chosen $(Ax, x)$, we pick the subspace that results in the maximal value. The second expression is defined similarly. 

An important question regarding this theorem is: why must a maximum and minimum exist? One explanation is the set $x \in E : \norm{x} = 1$ is the unit sphere, which is closed (contains its limit points) and bounded (contains a highest and lowest point). Since $Q[x] = (Ax, x)$ is a continuous function, it will have a maximum and minimum on any compact set. \textbf{Remove if the proof proves existence.}

\begin{proof}
We can pick an orthonormal basis so that matrix $A$ is diagonal, $A = diag\{ \lambda_{1}, \cdots, \lambda_{n} \}$. We will also reorder the eigenvalues so that $\lambda_{1} \geq \cdots \geq \lambda_{n}$.

We choose subspaces $E$ and $F$ such that $\Dim(E) = k$ and $\Codim(F) = k -1$. This means that $\Dim(E) + \Dim(F) = n + 1 > n$ so there exists a non-zero vector $x_{0} \in E \cap F$. After normalizing, we get $\norm{x} = 1$, and because $x$ also belongs to both $E$ and $F$, 
$$\min_{x \in E: \norm{x} = 1} (Ax, x) \leq (Ax_{0}, x_{0}) \leq \max_{x \in F: \norm{x} = 1} (Ax, x)$$

Since we only assumed dimensions of subspaces $E$ and $F$, the above inequality holds for all pairs of $E$ and $F$ with appropriate dimensions. Now we define 
$$E_{0} := \Span\{e_{1}, \cdots, e_{k} \} \qquad F_{0} := \Span \{e_{k}, \cdots, e_{n} \}$$ 

Since for any self-adjoint matrix $B$, the maximum and minimum of $(Bx, x) : \norm{x} = 1$ are the maximum and minimum eigenvalues, we get 
$$\min_{x \in E_{0} : \norm{x} = 1} (Ax, x) = \max_{x \in F_{0} : \norm{x} = 1} (Ax, x) = \lambda_{k}$$

It follows from our above inequality that for any subspace $E$ with $\Dim(E) = k$, 
$$\min_{x \in E : \norm{x} = 1} (Ax, x) \leq \max_{x \in F_{0} : \norm{x} = 1} (Ax, x) = \lambda_{k}$$
and for any subspace $F$ with $\Codim(F) = k - 1$, 
$$\max_{x \in F : \norm{x} = 1} (Ax, x) \geq \min_{x \in E_{0} : \norm{x} = 1} (Ax, x) = \lambda_{k}$$

But because on subspaces $E_{0}$ and $F_{0}$ both the maximum and minimum are $\lambda_{k}$, we know that $\min \max = \max \min = \lambda_{k}$. 
\end{proof}
\chapter{Matrix Functions - WIP} 

A \textbf{matrix function} is a function that maps one matrix to another. 

\section{Matrix Exponential}

Most of this section is from the Wikipedia article on the matrix exponential and Dan Klain's notes available at 

\begin{definition}
The \textbf{matrix exponential} is a matrix function on square matrices similar to the scalar exponential function. It is used to solve systems of linear differential equations. 

Let $X$ be an $n \times n$ matrix. The exponential is given by the power series, 
$$e^{X} = \sum_{k=0}^{\infty} \frac{1}{k!} X^{k} = I + X + \frac{1}{2!} X^{2} + \frac{1}{3!} X^{3} + \cdots $$
where $X^{0}$ is the identity matrix. Note that if $X$ is a $1 \times 1$ matrix, the exponential of $X$ is a $1 \times 1$ matrix whose element is the exponential of $X$'s sole element. 
\end{definition}

Two properties that are clear from this definition are 
\begin{enumerate}
	\item $exp(0) = I$ 
	\item $exp(X^{*}) = (exp(X))^{*}$
\end{enumerate}

Another useful property is the following result.

\begin{lemma}
If $Y$ is invertible, then 
$$e^{YXY^{-1}} = Ye^{X}Y^{-1}$$
\end{lemma}

\begin{lproof}
$$
\begin{aligned}
e^{YXY^{-1}} &= I + YXY^{-1} + \frac{1}{2!} (YXY^{-1})^{2} + \frac{1}{3!} (YXY^{-1})^{3} + \cdots \\
&= I + YXY^{-1} + Y \frac{X^{2}}{2!} Y^{-1} + Y \frac{X^{3}}{3!} Y^{-1} + \cdots \\
&= Y \Big( I + X + \frac{X^{2}}{2!} + \frac{X^{3}}{3!} + \cdots \Big) Y^{-1} \\
&= Ye^{X} Y^{-1}
\end{aligned}
$$
\end{lproof}

Not all properties of the scalar exponential function translate to the matrix exponential. For example, we know $e^{a + b} = e^{a} e^{b}$ when $a$ and $b$ are scalars, but the matrix exponential version of this typically does not hold. There are a few exceptions, however, as the following lemma illustrates. 

\begin{lemma}
Let $A$ be a square matrix, and let $s, t \in \C$. Then 
$$e^{A(s + t)} = e^{As} e^{At}$$
\end{lemma} 

\begin{lproof}
$$
\begin{aligned}
e^{As} e^{At} &= \Big( I + As + \frac{A^{2} s^{2}}{2!} + \cdots \Big) \Big( I + At + \frac{A^{2} t^{2}}{2!} + \cdots \Big) \\
&= \Big( \sum_{j=0}^{\infty} \frac{A^{j} s^{j}}{j!} \Big) \Big( \sum_{k=0}^{\infty} \frac{A^{k} t^{k}}{k!} \Big) \\
&= \sum_{j=0}^{\infty} \sum_{k=0}^{\infty} \frac{A^{j + k} s^{j} t^{k}}{j! k!} \Big)
\end{aligned}
$$

If we let $n = j + k$ so that $k = n - j$, we can write 
$$e^{As} e^{At} = \sum_{j=0}^{\infty} \sum_{n=j}^{\infty} \frac{A^{n} s^{j} t^{n - j}}{j! (n - j)!} = \sum_{n=0}^{\infty} \frac{A^{n}}{n!} \sum_{j=0}^{n} \frac{n!}{j! (n - j)!} s^{j} t^{n-j} = \sum_{n=0}^{\infty} \frac{A^{n} (s + t)^{n}}{n!} = e^{A(s +t)}$$
where the second equality follows from rearranging terms and the third follows from the Binomial Theorem. 
\end{lproof}

If we set $s = 1$ and $t = -1$, we get 
$$e^{A} e^{-A} = e^{A(1 - 1)} = e^{0} = I$$
which means that regardless of what matrix $A$ is, the exponential matrix $e^{A}$ is \textbf{always} invertible, with inverse $e^{-A}$. 

An important motivation for the matrix exponential is its use in differential equations. 

\begin{theorem}
Let $A$ be a square matrix, and let $t$ be a real scalar. Let $f(t) = e^{tA}$. Then $$f'(t) = Ae^{tA}$$ 
\end{theorem}

\begin{proof}
Applying the previous lemma to the limit yields 
$$f'(t) = \lim_{h \rightarrow 0} \frac{e^{A(t+h)} - e^{At}}{h} = e^{At} \Big( \lim_{h \rightarrow 0} \frac{e^{Ah} - I}{h} \Big)$$

Expanding $e^{Ah} - I$ gives us 
$$f'(t) = e^{At} \Big( \lim_{h \rightarrow 0} \frac{1}{h} \Big[ Ah + \frac{A^{2} h^{2}}{2!} + \cdots \Big] \Big)= e^{At} \Big( \lim_{h \rightarrow 0} A + \frac{A^{2} h}{2!} + \cdots \Big) = e^{At} A = Ae^{At}$$
\end{proof}

Recall that if $A$ and $B$ are matrices, then usually $e^{A} e^{B} \neq e^{A + B}$. The following theorem provides a condition for this equality to hold. 

\begin{theorem}
Let $A, B$ be $n \times n$ matrices. If $A$ and $B$ are commuting matrices ($AB = BA$), then 
$$e^{A + B} = e^{A} e^{B}$$
\end{theorem}

\begin{proof}
If $AB = BA$, we use the definition of the matrix exponential to write $Ae^{Bt} = e^{Bt}A$, and similarly for other combinations of $A$, $B$, $A + B$, and their exponentials. 

Let $g(t) = e^{(A + B)t} e^{-Bt} e^{-At}$, where $t$ is a real scalar. By the previous theorem and the product rule for derivatives, we have 
$$
\begin{aligned}
g'(t) &= (A + B)e^{(A + B)t} e^{-Bt}e^{-At} + e^{(A + B)t} (-B) e^{-Bt} e^{-At} + e^{(A+B)t} e^{-Bt} (-A) e^{-At} \\
&= (A+B) g(t) - Bg(t) -Ag(t) \\
&= 0
\end{aligned}
$$

Note that we could only factor out $(-A)$ and $(-B)$ because we know $AB = BA$. 

Since the derivative is 0 for all $t$, $g(t)$ must be a matrix of constants, so $g(t) = C$, for some constant matrix $C$. Setting $t=0$ gives us 
$$C = g(0) = e^{(A+B)0} e^{-B0} e^{-A0} = I$$
which must also be true for all $t$ since $C$ is constant so 
$$e^{(A+B)t} e^{-Bt} e^{-At} = I$$
and by multiplying both sides by $e^{At} e^{Bt}$, we get 
$$e^{At} e^{Bt} = e^{(A+B)t}$$
\end{proof}

\end{document}