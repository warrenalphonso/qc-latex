\chapter{Bilinear and Quadratic Forms - WIP}

\section{Main Definitions}

\begin{definition}
A \textbf{bilinear form} on $\R^{n}$ is a function $L(x, y)$ of arguments $x, y \in \R^{n}$ which is linear in each argument. That is 
\begin{enumerate}
	\item $L(\alpha x_{1} + \beta x_{2}, y) = \alpha L(x_{1}, y) + \beta L(x_{2}, y)$
	\item $L(x, \alpha y_{1} + \beta y_{2}) = \alpha L(x, y_{1}) + \beta L(x, y_{2})$
\end{enumerate}

If $x = \begin{bmatrix}
x_{1} \\
\vdots \\
x_{n}
\end{bmatrix}$ and $y = \begin{bmatrix}
y_{1} \\
\vdots \\
y_{n}
\end{bmatrix}$, a bilinear form can be written as 
$$L(x, y) = \sum_{j, k = 1}^{n} a_{jk} x_{k} y_{j}$$
or in matrix form 
$$L(x, y) = (Ax, y)$$
where the matrix $A$ is uniquely determined by the bilinear form $L$.
\end{definition}

\begin{definition}
There are many definitions of a quadratic form. 

One can define a quadratic form on $\R^{n}$ as the ``diagonal" of a bilinear form $L$, that is, any quadratic form $Q$ is defined as $Q[x] = L(x, x) = (Ax, x)$. 

Another definition is to say a quadratic form is a homogeneous second degree polynomial, that is, $Q$ is a polynomial of $n$ variables $x_{1}, \cdots, x_{n}$ having only terms of degree 2 (only terms $ax_{k}^{2}$ and $cx_{j} x_{k}$ are allowed).

There are infinitely many ways to write a quadratic form as $Q[x] = (Ax, x)$, but if we require $A$ to be \textit{symmetric} ($A^{T} = A$), then $A$ will be unique. 

Any \textbf{quadratic form $Q[x]$} on $\R^{n}$ admits a unique representation $Q[x] = (Ax, x)$ where $A$ is a real symmetric matrix. 

We can extend this definition to $\C^{n}$ by taking the self-adjoint transformation, $A = A^{*}$ and defining $Q[x] = (Ax, x)$. Unless explicitly noted, all our theorems will be true in the complex case as well. 

\end{definition}

The only essential difference is that in the complex case we have no choice: a real quadratic form corresponds to a Hermitian matrix. 

\begin{theorem}
$(Ax, x)$ is real for all $x \in \C^{n}$ if and only if $A = A^{*}$.  
\end{theorem}

\begin{proof}
To prove if: 
$$(Ax, x) = (x, A^{*} x) = (x, Ax) = \overline{(Ax, x)}$$

To prove only if: Consider the expression $(A (x + zy), x + zy)$. We assume it is real for all $z \in \C$. Now we can write 
$$
\begin{aligned}
\Big( A (x + zy), x + zy \Big) &= ( Ax + Azy, x + zy) \\
&= \overline{(x, Ax)} + \overline{(zy, Ax)} + \overline{(x, Azy)} + \overline{(zy, Azy)} \\ 
&= (Ax, x) + \overline{z} (Ax, y) + z (Ay, x) + z \overline{z} (Ay, y)
\end{aligned}
$$

We know that the final sum must be real. Since only the middle two terms can contribute to an imaginary solution, we know 
$$\overline{z} (Ax, y) + z (Ay, x) \in \R$$
Because $z$ can be any complex number, we know $z \neq \overline{z}$, so for the imaginary parts to cancel, we must have
$$(Ax, y) = (Ay, x)$$
and because we know $(Ax, y) = (x, A^{*} y)$, this means that $A^{*} = A$. 
\end{proof}

\section{Diagonalization of Quadratic Forms}

Quadratic forms are common; they appear in the study of curves in $\R^{2}$ and some surfaces in $\R^{3}$, for example. If we are given a set in $\R^{n}$ defined by $Q[x] = 1$, where $Q$ is some quadratic form, we might want to understand the structure of this set. We will try our usual approach of doing so using a change of variables. 

\subsection{Orthogonal Diagonalization}

Suppose we have a quadratic form $Q[x] = (Ax, x)$ in $\F^{n}$. We define $y = S^{-1} x$, where $S$ is any invertible $n \times n$ matrix. Now, we have
$$Q[x] = Q[Sy] = (ASy, Sy) = (S^{*} AS y, y)$$
so when written in variable $y$, the quadratic form has matrix $S^{*}AS$. 

Now our goal is to find an invertible matrix $S$ such that $S^{*} AS$ is diagonal. Using diagonalization, we can write $A = UDU^{*}$ because $A$ is symmetric, which means it's unitary and for unitary matrices $U^{*} = U^{-1}$. Then, we have $D = U^{*} AU$, so in $y = U^{-1} x$, the quadratic form has diagonal matrix. 

Geometrically, the columns of $U$ form an orthonormal basis in $\F^{n}$, which we'll call $B$. The change of coordinate matrix $[I]_{S, B}$ from $B$ to the standard basis is exactly $U$. Since $y = U^{-1} x$, the coordinates $y_{1}, \cdots, y_{n}$ can be interpreted as coordinates of the vector $x$ in the new basis $B$. So, orthogonal diagonalization allows us to visualize the set $Q[x] = 1$ very well if we can visualize it for diagonal matrices. 

\textbf{Example:} Consider $Q[x, y] = 2x^{2} + 2y^{2} + 2xy$. We want to describe the set of points $(x, y) \in \R^{2} : Q(x, y) = 1$. 

The matrix $A$ of $Q$ is 
$$A = \begin{bmatrix}
2 & 1 \\
1 & 2
\end{bmatrix}$$ 
After orthogonally diagonalizing $A$, we can write 
$$A = U \begin{bmatrix}
3 & 0 \\
0 & 3
\end{bmatrix} U^{*} \qquad \text{ where } U = \frac{1}{\sqrt{2}} \begin{bmatrix}
1 & -1 \\
1 & 1
\end{bmatrix}$$
which means
$$U^{*} AU = \begin{bmatrix}
3 & 0 \\
0 & 1
\end{bmatrix} = D$$

This tells us the set $\{y : (Dy, y) = 1\}$ is the ellipse with half-axes $\frac{1}{\sqrt{3}}$ and 1. Thus, the set $\{x \in \R^{2}: (Ax, x) = 1 \}$ is the same ellipse but in the basis $\begin{bmatrix}
\frac{1}{\sqrt{2}} \\
\frac{1}{\sqrt{2}}
\end{bmatrix}, \begin{bmatrix}
-\frac{1}{\sqrt{2}} \\
\frac{1}{\sqrt{2}}
\end{bmatrix}$, or in other words, the same ellipse rotated $\frac{\pi}{4}$. 

\subsection{Non-orthogonal Diagonalization} 

Orthogonal diagonalization requires computing eigenvalues and eigenvectors which can be hard to do for large $n$ without computers. Non-orthogonal diagonalization requires only finding an invertible $S$ without requiring $S^{-1} = S^{*}$ such that $D = S^{*}AS$ is diagonal, which is much easier to do. We will cover two of the most used methods of non-orthogonal diagonalization. 

\subsubsection{Diagonalization by Completion of Squares} 

We will demonstrate this method on real quadratic forms (forms on $
R^{n}$), but it can also be used in the complex case. 

Consider again the quadratic form of two variables $Q[x] = 2x_{1}^{2} + 2x_{1} x_{2} + 2x_{2}^{2}$ (the same equation as before except we use $x_{1}, x_{2}$ instead of $x, y$). Since 
$$2 \Big( x_{1} + \frac{1}{2} x_{2} \Big)^{2} = 2 \Big( x_{1}^{2} + 2 x_{1} \frac{1}{2} x_{2} + \frac{1}{4} x_{2}^{2} \Big) = 2x_{1}^{2} + 2x_{1}x_{2} + \frac{1}{2} x_{2}^{2}$$
where the first two terms coincide with the first two terms of $Q$, we can write 
$$Q[x] = 2 \Big( x_{1} + \frac{1}{2} x_{2} \Big)^{2} + \frac{3}{2} x_{2}^{2} = 2y_{1}^{2} + \frac{3}{2} y_{2}^{2}$$
where $y_{1} = x_{1} + \frac{1}{2} x_{2}$ and $y_{2} = x_{2}$. We can use this same method for quadratic forms of more than 2 variables. Note that we can always split a product of two variables into a corresponding square using the identity, 

$$4 x_{1} x_{2} = (x_{1} + x_{2})^{2} - (x_{1} - x_{2})^{2}$$

\subsubsection{Diagonalization Using Row/Column Operations}

Our second method is to perform row operations on the matrix $A$ of the quadratic form. Unlike normal Gauss-Jordan row reduction, after each row operation, we need to perform the same column operation (because we want to ensure $S^{*} AS$ is diagonal). 

\textbf{Maybe insert an example. }

To understand why this works, realize that a row operation corresponds to left multiplying by an elementary matrix while a column operation is equivalent to right multiplying by the transpose of the same elementary matrix. Thus, performing row operations $E_{1}, \cdots, E_{n}$ along with the same column operations gives us 
$$E_{n} \cdots E_{1} A E_{1}^{*} \cdots E_{n}^{*} = EAE^{*}$$

\textbf{Prove this including the complex case. }

As for the identity matrix in the right side of the augmented matrix, we only performed row operations on it so we have 
$$E_{n} \cdots E_{1} I = EI = E$$

Now if we set $E^{*} = S$, we know $S^{*}AS$ is a diagonal matrix, and the matrix $E = S^{*}$ is the ``right half" of the transformed augmented matrix. 

A tricky operation to implement is the swapping of two rows. Consider 
$$A = \begin{bmatrix}
0 & 1 \\
1 & 0
\end{bmatrix}$$
Swapping rows 1 and 2 would diagonalize the matrix, but we would also be required to swap columns 1 and 2, which means we end up with the original matrix. A simple idea is to use row operations to get a non-zero entry on the diagonal. For example, 
$$\begin{bmatrix}[cc|cc]
0 & 1 & 1 & 0 \\
1 & 0 & 0 & 1
\end{bmatrix} \rightarrow \begin{bmatrix}[cc|cc] 
1 & 1 & 1 & \frac{1}{2} \\
1 & 0 & 0 & 1
\end{bmatrix} \rightarrow \begin{bmatrix}[cc|cc] 
1 & 0 & 1 & \frac{1}{2} \\
0 & -1 & -1 & \frac{1}{2}
\end{bmatrix}
$$
where the first step is adding half of row 2 to row 1 (and then the corresponding column operation of adding half of column 2 to column 1), and the second step is subtracting row 1 from row 2 (and then subtracting column 1 from column 2). 

\section{Silvester's Law of Inertia} 

We now know there are many different ways to diagonalize a quadratic form. For example, if we have a diagonal matrix $D$, we can take a diagonal matrix $S$ and transform $D$ to 
$$S^{*} DS = \text{diag}\{ s_{1}^{2} \lambda_{1}, \cdots, s_{n}^{2} \lambda_{n} \}$$

This transformation changes the diagonal entries of $D$, but it does not change the \textit{signs} of the entries. The following theorem says this is always the case. 

\begin{theorem}[Silvester's Law of Inertia]
\textbf{Wait but I don't even prove this so maybe don't make it a theorem}

For a Hermitian matrix $A$ (for a quadratic form $Q[x] = (Ax, x)$) and any of its diagonalizations $D = S^{*} AS$, the number of positive, negative, and zero diagonal entries of $D$ depends only on $A$, but not on a particular choice of diagonalization. 
\end{theorem}

We will need some more help to prove this. 

\begin{definition}
Given an $n \times n$ Hermitian matrix $A = A^{*}$ (a quadratic form $Q[x] = (Ax, x)$ on $\F^{n}$), we call a subspace $E \subset \F^{n}$ \textbf{positive} if 
$$(Ax, x) > 0$$ 
for all $x \in E, x \neq 0$. To emphasize the role of $A$ we will say $A$-positive. There are similar definitions for $A$-negative and $A$-neutral. 
\end{definition}

\begin{lemma}
Let $D$ be a diagonal matrix. Then the number of positive and negative diagonal entries of $D$ coincides with the maximal dimensions of a $D$-positive and $D$-negative subspace, respectively. 
\end{lemma}

\begin{lproof}
By rearranging the standard basis in $\F^{n}$, we can assume that the positive diagonal entries of $D$ are the first $r_{+}$ diagonal entries. 

Now consider the subspace $E_{+}$ spanned by these $r_{+}$ coordinate vectors $e_{1}, \cdots, e_{r_{+}}$. Clearly, $E_{+}$ is a $D$-positive subspace with dimension $r_{+}$. 

Now we'll show that for any other $D$-positive subspace $E$, we have $dim(E) \leq r_{+}$. Consider the orthogonal projection $P$ onto $E_{+}$, 
$$P[x_{1}, \cdots, x_{n}]^{T} = [x_{1}, \cdots, x_{r_{+}}, 0, \cdots, 0]^{T}$$

Now define an operator $T: E \rightarrow E_{+}$ by 
$$Tx = Px \qquad \forall x \in E$$

In other words, since $P$ is defined on the whole space, $T$ is the restriction of $P$ to domain $E$ and target space $E_{+}$. 

Now we'll find the null space of $T$ so that we can apply the Rank Theorem. For any $x$ such that $Tx = Px = 0$, we know $x_{1} = \cdots = x_{r_{+}} = 0$, so 
$$(Dx, x) = \sum_{r_{+} + 1}^{n} \lambda_{k} x_{k}^{2} \leq 0$$
since $\lambda_{k} \leq 0$ for $k > r_{+}$. Because $x$ belongs to $D$-positive subspace $E$, the inequality only holds for $x = 0$, so $Ker(T) = \{0\}$. 

We know $Rank(T) = dim(Ran(T)) \leq dim(E_{+}) = r_{+}$ since $Ran(T) \subset	E_{+}$. By the Rank Theorem, $dim(Ker(T)) + rank(T) = dim(E)$. But we just proved $Ker(T) = \{0\}$ so $dim(Ker(T)) = 0$ which means 
$$dim(E) = Rank(T) \leq r_{+}$$

To prove the lemma for negative entries, we just prove the above for $-D$. 
\end{lproof}

This result proves the positive and negative diagonal entries of $D$ coincide with maximal dimensions of $D$-positive and $D$-negative subspaces. The following proves a stronger result. 

\begin{theorem}
Let $A$ be an $n \times n$ Hermitian matrix, and let $D = S^{*} AS$ be its diagonalization by an invertible matrix $S$. Then the number of positive and negative diagonal entries of $D$ coincides with the maximal dimensions of an $A$-positive and $A$-negative subspace, respectively. 
\end{theorem}

\begin{proof}
Since $D = S^{*}AS$ is a diagonalization of $A$, we know 
$$(Dx, x) = (S^{*}ASx, x) = (ASx, Sx)$$
which means that for any $D$-positive subspace $E$, the subspace $SE$ is an $A$-positive subspace. The same identity implies that for any $A$-positive subspace $F$, the subspace $S^{-1}F$ is $D$-positive. 

Since $S$ and $S^{-1}$ are invertible transformations, $dim(E) = dim(SE)$ and $dim(F) = dim(S^{-1}F)$. Thus, for any $D$-positive subspace $E$, we can find an $A$-positive subspace ($SE$ for example) of the same dimension, and for any $A$-positive subspace $F$, we can find a $D$-positive subspace ($S^{-}F$ for example) of the same dimension. Thus, the maximal possible dimensions of an $A$-positive and a $D$-positive subspace coincide, so by the above lemma, the theorem is proved. The same reasoning supports $A$-negative subspaces. 
\end{proof}

\section{Minimax Characterization of Eigenvalues and Silvester's Criterion of Positivity}

\begin{definition}
A quadratic form is called 
\begin{itemize}
	\item \textbf{positive definite} if $Q[x] > 0 \quad \forall x \neq 0$
	\item \textbf{positive semidefinite} if $Q[x] \geq 0$
	\item \textbf{negative definite} if $Q[x] < 0 \quad \forall x \neq 0$ 
	\item \textbf{negative semidefinite} if $Q[x] \leq 0$ 
	\item \textbf{indefinite} if it takes on both positive and negative values 
\end{itemize}
\end{definition}

A Hermitian matrix $A = A^{*}$ is called positive definite if the corresponding quadratic forms $Q[x] = (Ax, x)$ is positive definite. Additionally, any Hermitian matrix $A$ is positive definite if its eigenvalues are all positive. By orthogonal diagonalization, we know there is a basis in which $A$ is diagonal, and it is clear that a diagonal matrix is positive definite if its eigenvalues are all positive.

In fact, because of Silvester's Law of Inertia, we don't need to compute eigenvalues. We can just find a non-orthogonal diagonalization and look at those diagonal entries. 

\subsection{Silvester's Criterion of Positivity}

We'll begin by looking for a simple requirement for positivity. Let's analyze the matrix 
$$\begin{bmatrix}
a & b \\
\overline{b} & c
\end{bmatrix}$$
Notice that if this matrix is positive definite, then $\lambda_{1} \lambda_{2} > 0$ and $\lambda_{1} + \lambda_{2} > 0$ which means 