\documentclass[letterpaper]{article}

\usepackage{fancyhdr}
\usepackage{amsmath}
\usepackage{amssymb}
\usepackage{caption}
\usepackage{graphicx}

\usepackage{amsthm}

\pagestyle{fancy}
\fancyhf{}
\rhead{Warren Alphonso}
\cfoot{\thepage} 

\newtheorem{theorem}{Theorem}
\theoremstyle{definition}
\newtheorem{definition}{Definition}[section]

\renewcommand\qedsymbol{$\blacksquare$}


\setlength{\parindent}{0cm}

\title{Linear Algebra} 
\author{Warren Alphonso}
\date{May 18, 2019}

\begin{document}
\maketitle
A very reductionist summary of key Linear Algebra concepts from \textit{Linear Algebra and its Applications} by Lay, Lay, and McDonald. 

\section{Systems of Linear Equations}
\begin{definition}
A \textbf{linear equation} is an equation that can be written in the form 
$$ a_{1}x_{1} + ... + a_{n}x_{n} = b$$ 
where $b$ and the coefficients $a_{k}$ are real or complex numbers. 
\end{definition}

We can record the important information of a system of linear equations in a matrix. Given the system
$$x_1 - 2x_2 + x_3 = 0$$ 
$$2x_2 - 8x_3 = 8 $$
$$5x_1 - 5x_3 = 10 $$
we place the coefficients of each variable aligned in columns
$$\begin{bmatrix}
1 && -2 && 1 \\
0 && 2 && -8 \\
5 && 0 && -5
\end{bmatrix}$$
This is called the \textbf{coefficient matrix} and 
$$\begin{bmatrix}
1 && -2 && 1 && 0 \\
0 && 2 && -8 && 8 \\
5 && 0 && -5 && 10
\end{bmatrix}$$
is called the \textbf{augmented matrix}. The size of a matrix is described as \textbf{ m x n} where $m$ denotes the number of rows and $n$ the number of columns. 

\begin{definition}
\textbf{Elementary Row Operations:}
\begin{enumerate}
	\item (Replacement) Replace one row by the sum of itself and a multiple of another row. 
	\item (Interchange) Interchange two rows. 
	\item (Scaling) Multiply all entries in a row by a nonzero constant. 
\end{enumerate}
\end{definition}

\begin{definition}
\textbf{Row Echelon form} denotes a matrix with: 
\begin{enumerate}
	\item All nonzero rows are above any rows of all zeros.
	\item Each leading entry of a row is in a column to the right of the leading entry of the row above it. 
	\item All entries in  column below a leading entry are zeros. 
\end{enumerate}
\textbf{Reduced Row Echelon form} means the leading entry in nonzero rows is 1. 
\end{definition}

\textbf{Parallelogram Rule for Addition:} If \textbf{u} and \textbf{v} in $\mathbb{R}^2$ are represented as points on the plane, then \textbf{u} + \textbf{v} corresponds to the fourth vertex of the parallelogram whose other vertices are 0, \textbf{u}, and \textbf{v}. 

\begin{definition}
\textbf{Span}\{$v_1, ..., v_p$\} denotes the set of all vectors formed by $c_1 v_1 + ... + c_p v_p$. 
\end{definition}

\begin{definition}
A set of vectors \{$v_1, ..., v_p$\} is said to be \textbf{linearly independent} if the equation 
$$c_1 v_1 + ... + c_p v_p = 0$$ 
has only the trivial solution. 
\end{definition}

\begin{theorem}
If a set contains more vectors than there are entries in each vector, then the set is linearly independent. That is, the set $\{v_1, ..., v_p\}$ in $\mathbb{R}^n$ is linearly dependent if $p > n$. 
\end{theorem}

\begin{proof}
Let $A = [v_1 \cdots v_p]$. Then $A$ is $n \times p$, and the equation $Ax = 0$ corresponds to a system of $n$ equations in $p$ unknowns. In $Ax = b$, the $x$ vector must have dimension $p$, so if $p>n$, then there are more variables than equations, so $Ax = 0$ has a nontrivial solution, and the columns of $A$ are linearly dependent. 
\end{proof}

An alternate way to conceptualize matrix multiplication: A \textbf{transformation} (or \textbf{function} or \textbf{mapping}) $T$ from $\mathbb{R}^n$ to $\mathbb{R}^m$ is a rule that assigns to each vector $x$ in $\mathbb{R}^n$ a vector $T(x)$ in $\mathbb{R}^m$. The set $\mathbb{R}^n$ is called the \textbf{domain} of $T$ and $\mathbb{R}^m$ is called the \textbf{codomain} of T. The set of all images $T(x)$ is called the \textbf{range} of $T$. 

\begin{definition} 
A transformation $T$ is \textbf{linear} if they preserve vector addition and scalar multiplication. That is:
\begin{enumerate}
	\item $T(u + v) = T(u) + T(v)$
	\item $T(cu) = cT(u)$ for all scalars $c$
\end{enumerate}
Every matrix transformation is a linear transformation. These two requirements mean that $T(0) = 0$ for linear transformations. 
\end{definition}

\begin{theorem}
Let $T: \mathbb{R}^n \rightarrow \mathbb{R}^m$ be a linear transformation. Then there exists a \textbf{unique} matrix $A$ such that 
$$ T(x) = Ax $$
In fact, $A$ is the $m \times n$ matrix whose $j$th column is the vector $T(e_j)$, where $e_j$ is the $j$th column of the identity matrix in $\mathbb{R}^n$: 
$$A = [T(e_1) \cdots T(e_n)]$$
\end{theorem}

\begin{proof}
Write $x = I_n x = [e_1 \cdots e_n]x = x_1 e_1 + \cdots + x_n e_n$, and use the linearity of $T$ to compute 
$$T(x) = T(x_1 e_1 + \cdots x_n e_n) = x_1 T(e_1) + \cdots + x_n T(e_n)$$ 

$$ \begin{bmatrix}
T(e_1) \cdots T(e_n)
\end{bmatrix} 
\begin{bmatrix}
x_1 \\
\vdots \\
x_n
\end{bmatrix} 
= Ax
$$ 
\end{proof}

\begin{definition}
A mapping $T: \mathbb{R}^n \rightarrow \mathbb{R}^m$ is \textbf{onto} $\mathbb{R}^m$ if each $b$ in $\mathbb{R}^m$ is the image of at least one $x$ in $\mathbb{R}^n$. 
\end{definition}

\begin{definition}
A mapping $T: \mathbb{R}^n \rightarrow \mathbb{R}^m$ is \textbf{one-to-one} if each $b$ in $\mathbb{R}^m$ is the image of at most one $x$ in $\mathbb{R}^n$. 
\end{definition}

\section{Matrix Algebra} 
\begin{definition}
If $A$ is an $m \times n$ matrix and $B$ is an $n \times p$ matrix, then 
$$AB = A
\begin{bmatrix}
b_1 & b_2 & \cdots & b_p
\end{bmatrix}
= \begin{bmatrix}
Ab_1 & Ab_2 & \cdots & Ab_p
\end{bmatrix}$$
\end{definition}

\begin{definition}
Let $A$ be an $m \times n$ matrix and let $B$ and $C$ have sizes for which the indicated sums and products are defined. 
\begin{enumerate}
	\item $A(BC) = (AB)C$
	\item $A(B + C) = AB + AC$
	\item $(B + C)A = BA + CA $
	\item $r(AB) = (rA)B = A(rB)$
	\item $I_m A = A = AI_n$
\end{enumerate}
Warnings: 
\begin{enumerate}
	\item In general, $AB \neq BA$ 
	\item If $AB = AC$, then it is \textbf{not true} in general that $B=C$
	\item If $AB = 0$, then it is \textbf{not true} always that $A=0$ or $B=0$
\end{enumerate}
\end{definition}

\begin{definition}
The \textbf{transpose} of $A$ is the matrix whose columns are formed from the corresponding rows of $A$, denoted by $A^T$. 
\begin{itemize}
	\item $(A^T)^T = A$
	\item $(A + B)^T = A^T + B^T$
	\item $(cA)^T = rA^T$
	\item $(AB)^T = B^T A^T$
\end{itemize}
\end{definition}

\begin{definition}
An $n \times n$ matrix $A$ is \textbf{invertible} if there is an $n \times n$ matrix $A^{-1}$ such that $A^{-1}A = I$. 
\begin{itemize}
	\item $(A^{-1})^{-1} = A$
	\item If $A$ and $B$ are $n \times n$ invertible matrices then so is $AB$. And $(AB)^{-1} = B^{-1}A^{-1}$
	\item $(A^T)^{-1} = (A^{-1})^T$
\end{itemize}
To compute the inverse, solve the equation  $AB = I$, by row-reducing the augmented matrix [A  I], until you get [I  B]. 
\end{definition}


\section{Determinants}
\begin{definition}
Let $A = \begin{bmatrix}
a && b \\
c && d
\end{bmatrix}$. If $ad - bc \neq 0$, then $A$ is invertible and 
$$A^{-1} = \frac{1}{ad - bc} \begin{bmatrix}
d && -b \\
-c && a
\end{bmatrix}$$
The quantity $ad - bc$ is the \textbf{determinant} of the matrix. If the determinant is 0, the matrix $A$ is not invertible. 
\end{definition}

\begin{definition}
To generalize, the determinant of an $n \times n$ matrix $A$ can be computed using a \textbf{cofactor expansion} across any row or down any column. The expansion across the $i$th row is 
$$det(A) = a_{i1}C_{i1} + a_{i2}C_{i2} + \cdots + a_{in}C_{in}$$ where $C_{ij} = (-1)^{i + j} det(A_{ij})$ 
\end{definition} 

\begin{theorem}
If $A$ is an upper triangular matrix, then det(A) is the product of the entries on the main diagonal. 
\end{theorem}

\begin{proof}
Cofactoring an upper triangular matrix by using the first column ultimately leads to continuously multiplying the upper left item by the determinant of the smaller matrix. For example, 
$$ A = \begin{bmatrix}
3 & 2 & 9 \\
0 & 4 & -1 \\
0 & 0 & -8
\end{bmatrix}$$
Then, 
$$det(A) = 3 \cdot det(\begin{bmatrix}
4 & -1 \\ 
0 & -8
\end{bmatrix}) = 3 \cdot -32 = -96 = 3 \cdot 4 \cdot -8$$
\end{proof}

\begin{definition}
Determinants after Row Operations
\begin{enumerate}
	\item If a multiple of a row in matrix $A$ is added to another row to produce matrix $B$, then $det(B) = det(A)$
	\item If two rows in $A$ are swapped to produce $B$, then $det(B) = -det(A)$
	\item If one row in $A$ is multiplied by $k$ to produce $B$, then $det(B) = k \cdot det(A)$
\end{enumerate}
\end{definition}

These identities can be used to easily find determinants of square matrices. Once we reduce a matrix $A$ to upper triangular form $B$, we know $det(B) = (-1)^{r} det(A)$ if $r$ is the number of row swaps we performed. If we cannot reduce to row echelon form, we know the determinant must be 0 since $A$ must not be invertible. 

\begin{theorem}
If $A$ is an $n \times n$ matrix, then $det(A^T) = det(A)$.
\end{theorem}

\begin{proof}
We proceed by induction. The theorem is trivially true for $n = 1$. Assume the theorem is true for $k \times k$ matrices. We will show it holds for $n = k +1$. The cofactor of $a_{1j}$ in $A$ equals the cofactor of $a_{j1}$ in $A^T$ because it is a $k \times k$ determinant. Thus, the cofactor of $det(A^T)$ down the first \textit{column} equals the cofactor of  $det(A)$ across the first \textit{row}, so $A$ and $A^T$ have equal determinants. Thus, the statement is true for all $n$. 
\end{proof}

\begin{theorem}
If $A$ and $B$ are $n \times n$ matrices, then $det(AB) = det(A) det(B)$. 
\end{theorem}

\begin{theorem}{Cramer's Rule: }
Let $A$ be an invertible $n \times n$ matrix. For any $b$ in $\mathbb{R}^n$, the unique solution $x$ of $Ax = b$ has entries given by 
$$x_{i} = \frac{det(A_i(b))}{det(A)} \text{ for } i = 1, 2, ..., n$$
where $A_i(b)$ denotes the matrix obtained by replacing $A$'s $i$th column with $b$. 
\end{theorem}

\begin{proof}
Denote the columns of $A$ by $a_1, ..., a_n$ and the columns of the $n \times n$ identity matrix by $e_1, ..., e_n$. If $Ax = b$, the definition of matrix multiplication tells us 
$$A \cdot I_i (x) = A \begin{bmatrix}
e_1 & \cdots & x & \cdots & e_n
\end{bmatrix} = 
\begin{bmatrix}
Ae_1 & \cdots & Ax & \cdots & Ae_n
\end{bmatrix}$$
$$ = \begin{bmatrix}
a_1 & \cdots & b & \cdots & a_n
\end{bmatrix} = A_i (b)$$

Using the multiplicative property of determinants, 
$$ (det(A)) (det(I_i(x))) = det(A_i (b))$$
Since $det(I_i (x))$ is $x$, we just divide by $det(A)$. 
\end{proof}

\section{Vector Spaces}
Some of this is from Chapter 1, but I think it makes more sense to define these concepts here. 
\begin{definition} 
A \textbf{subspace} of $\mathbb{R}^n$ is any set $H$ in $\mathbb{R}^n$ that is closed under addition and scalar multiplication. That is:
\begin{enumerate}
	\item The zero vector is in $H$ 
	\item For each $u$ and $v$ in $H$, the sum $u + v$ is in $H$
	\item For each $u$ in $H$ and each scalar $c$, the vector $cu$ is in $H$
\end{enumerate}
\end{definition}

\begin{definition}
The \textbf{column space} of an $m \times n$ matrix $A$ is the set of all linear combinations of the columns of $A$, denoted by $Col(A)$. Since the columns of $A$ are in $\mathbb{R}^m$, the columns space is in $\mathbb{R}^m$. 
\end{definition}

\begin{definition}
The \textbf{null space} of a matrix $A$ is the set of all solutions for $Ax = 0$, denoted by $Nul(A)$. When $Nul(A)$ contains nonzero vectors, the number of vectors in the nullspace equals the number of free variables in $Ax = 0$. 
\end{definition}

\begin{definition}
A \textbf{basis} for a subspace $H$ in $\mathbb{R}^n$ is a linearly independent set in $H$ that spans $H$. 
\end{definition}

Using the basis for a subspace $H$ is preferable because any vector in $H$ can only be written in one way as a linear combination of the basis vectors. 
\begin{proof}
Suppose $\mathbb{B}$ = \{ $b_1$ , ..., $b_p$\} is a basis for $H$, and suppose a vector $x$ in $H$ can be generated in two ways: 
$$x = c_1 b_1 + \cdots + c_p b_p \text{ and } x = d_1 b_1 + \cdots + d_p b_p$$
Subtracting gives us:
$$0 = (c_1 - d_1)b_1 + \cdots + (c_p - d_p)b_p$$
Since $\mathbb{B}$ is linearly independent, the weights must all be zero, so $c_j = d_j$ so the two representations are really just the same. 
\end{proof}

\begin{theorem}
The pivot columns of a matrix $A$ form a basis for $Col(A)$. 
\end{theorem}

\begin{proof}
Let $B$ be the reduced echelon form of $A$. The set of pivot columns of $B$ is linearly independent, since no vector is a linear combination of the vectors that precede it. Since $A$ is \textit{row equivalent} to $B$, the pivot columns of $A$ are linearly independent as well. Thus, the nonpivot columns of $A$ can be discarded from the spanning set of $Col(A)$. 

\textbf{Warning:} The pivot columns of $A$ are only evident when $A$ has been reduced to \textit{echelon} form. After reducing, make sure to use the \textbf{pivot columns of $A$ itself} for the basis of $Col(A)$. The columns of an echelon form of $A$ are often not in the column space of $A$. 
\end{proof}

\begin{theorem}{Unique Representation Theorem: }
Let $\mathbb{B} = \{ b_1 , ..., b_n \}$ be a basis for a vector space $V$. Then for each $x$ in $V$, there exists a unique set of scalars $c_1 ,..., c_n$ such that 
$$x = c_1 b_1 + \cdots + c_n b_n$$
\end{theorem}

\begin{proof}
Since $\mathbb{B}$ spans $V$, we know there exist scalars such that we can form $x$. Assume $x$ also has the representation 
$$ x = d_1 b_1 + \cdots + d_n b_n$$
Then, after subtracting we have
$$ 0 = (c_1 - d_1) b_1 + \cdots + (c_n - d_n) b_n$$
Since $\mathbb{B}$ is linearly independent, these weights must all be zero so $c_j = d_j$. 
\end{proof}

Because of the unique representation of each vector $x$ in a basis, we can define the coordinates of $x$ relative to the basis $\mathbb{B}$ as the weights $c_1 , ..., c_n$. 

\begin{definition}{Changing coordinates: }
$$x = \mathbb{B} [x]_{\mathbb{B}}$$
where $\mathbb{B}$ denotes the matrix whose columns are basis vectors, and $x_{\mathbb{B}}$ denotes the $x$ vector represented by basis coordinates. 

To understand this, let $b_1 = \begin{bmatrix}
2 \\
1
\end{bmatrix} , b_2 = \begin{bmatrix}
-1 \\
1
\end{bmatrix} , x = \begin{bmatrix}
4 \\
5
\end{bmatrix} , \text{ and }\mathbb{B} = \{b_1, b_2\}$. To find $[x]_{\mathbb{B}}$ of $x$ relative to $\mathbb{B}$,

$$ c_1 \begin{bmatrix}
2 \\
1
\end{bmatrix} + c_2 \begin{bmatrix}
-1 \\ 
1
\end{bmatrix} = \begin{bmatrix}
4 \\
5
\end{bmatrix}
$$

or

$$ \begin{bmatrix}
2 & -1 \\
1 & 1
\end{bmatrix} 
\begin{bmatrix}
c_1 \\
c_2
\end{bmatrix} 
= \begin{bmatrix}
4 \\
5
\end{bmatrix}
$$ 

Since we know the columns of $\mathbb{B}$ are linearly independent, it must be invertible so we can multiply $x$ by $\mathbb{B}^{-1}$ to get $[x]_{\mathbb{B}}$. 
\end{definition}

\begin{definition}{Change of basis: }
We can generalize the above further. Let $\mathbb{B} = \{ b_1 , ..., b_n \}$ and $\mathbb{C} = \{ c_1 , ..., c_n \}$ be bases of a vector space $V$. Then there is a \textit{unique} $n \times n$ matrix $\underset{\mathbb{C} \rightarrow \mathbb{B}}{P}$ such that
$$[x]_{\mathbb{C}} = \underset{\mathbb{C} \rightarrow \mathbb{B}}{P} [x]_{\mathbb{B}} $$

The columns of $\underset{\mathbb{C} \rightarrow \mathbb{B}}{P}$ are the $\mathbb{C}$-coordinate vectors of the vectors in the basis $\mathbb{B}$, that is
$$ \underset{\mathbb{C} \rightarrow \mathbb{B}}{P} = \begin{bmatrix}
[b_1 ]_{\mathbb{C}} \cdots [b_n]_{\mathbb{C}}
\end{bmatrix}$$

\end{definition}

\begin{definition}
An \textbf{isomorphism} from $V$ to $W$ is a one-to-one linear transformation. 
\end{definition}

\begin{definition}
The \textbf{dimension} of a nonzero subspace $H$ is the number of vectors in any basis for $H$. The dimension of the zero subspace $\{0\}$ is defined to be zero. 
\end{definition}

\begin{definition}
The \textbf{rank} of a matrix $A$ is the dimension of the column space of $A$. 
\end{definition}

\begin{theorem}
If a matrix $A$ has $n$ columns, then $Rank(A) + Dim(Nul(A)) = n$. 
\end{theorem}
\begin{proof}
An intuitive understanding for this can be achieved by restating the theorem as follows:
$$ \Big( \text{num of pivot columns} \Big) + \Big( \text{num of nonpivot columns} \Big) = \Big( \text{num of columns} \Big)   $$
\end{proof}

\begin{definition}
If $A$ is an $m \times n$ matrix, each row has $n$ entries and can be understood as a vector in $\mathbb{R}^n$. The set of all linear combinations of the row vectors is called the \textbf{row space} of $A$, denoted by $Row(A)$. Note that $Row(A) = Col(A^{T})$. 
\end{definition}

\section{Eigenvalues and Eigenvectors}
\begin{definition}
An \textbf{eigenvector} of an $n \times n$ matrix $A$ is a nonzero vector $x$ such that $Ax = \lambda x$. The scalar $\lambda$ is called an \textbf{eigenvalue} of $A$ if and only if the equation $$(A - \lambda I)x = 0$$ has a nontrivial solution. The set of all solutions to this equation is the null space of the matrix $A - \lambda I$; this subspace of $\mathbb{R}^n$ is called the \textbf{eigenspace} of $A$. 
\end{definition}

\begin{theorem}
The eigenvalues of a triangular matrix are the entries on its main diagonal. 
\end{theorem}

\begin{proof}
Consider the $3 \times 3$ case. If $A$ is upper triangular, then 
$$ A - \lambda I = \begin{bmatrix}
a_{11} - \lambda & a_{12} & a_{13} \\
0 & a_{22} - \lambda & a_{23} \\
0 & 0 & a_{33} - \lambda
\end{bmatrix}$$
The scalar $\lambda$ is an eigenvalue if and only if $(A - \lambda I)x = 0$ has a nontrivial solution, which means the equation must have a free variable, which would only occur if at least one of the values on the diagonal is zero. 
\end{proof}

\begin{theorem}
The eigenvectors of $A$, $v_1 , ... , v_r$, that correspond to \textit{distinct} eigenvalues, $\lambda_1 , ..., \lambda_r$, are linearly independent. 
\end{theorem}

\begin{proof}
Suppose $\{v_1, ..., v_r\}$ is linearly dependent. Let $p$ be the least index such that $v_{p+1}$ is a linear combination of the preceding linearly independent eigenvectors. Then there exist scalars such that 
$$c_1 v_1 + \cdots + c_p v_p = v_{p+1}$$
Multiply both sides by $A$, using the fact that $Av_k = \lambda_k v_k$, to get
$$c_1 \lambda_1 v_1 + \cdots + c_p \lambda_p v_p = \lambda_{p+1} v_{p+1}$$
We can also multiply both sides of our first equation by $\lambda_{p+1}$ and then subtract to get 
$$c_1 (\lambda_1 - \lambda_{p+1})v_1 + \cdots + c_p (\lambda_p - \lambda_{p+1})v_p = 0$$
Because we assumed $v_{p+1}$ was the first linearly dependent eigenvector, the set $\{v_1 , ..., v_p\}$ must be linearly independent. That means all $(\lambda_k - \lambda_{p+1})$ should be 0, but because the eigenvalues are distinct they cannot be. Thus, we arrive at a contradiction. 
\end{proof}

Remember that to find eigenvalues, we need to find scalars $\lambda$ such that 
$$(A - \lambda I)x = 0$$
has a nontrivial solution. This is equivalent to the matrix $A - \lambda I$ being not invertible, which is equivalent to $det(A - \lambda I) = 0$. Writing the determinant as a polynomial involving only $\lambda$ is called the characteristic equation of a matrix. 

\begin{theorem}{Diagonalization Theorem: }
An $n \times n$ matrix $A$ is \textbf{diagonalizable} if and only if $A$ has $n$ linearly independent eigenvectors. If this condition is met, we can write 
$$A = PDP^{-1}$$
where $P$ is a matrix whose columns are $n$ linearly independent eigenvectors of $A$, and $D$ is a diagonal matrix whose diagonal entries are corresponding eigenvalueus of $A$. 
\end{theorem}

\begin{proof}
Right multiplying both sides by $P$ gives us $AP = PD$. 
$$AP = A \begin{bmatrix}
v_1 & \cdots & v_n
\end{bmatrix} = \begin{bmatrix}
Av_1 & \cdots & Av_n
\end{bmatrix} = \begin{bmatrix}
\lambda_1 v_1 & \cdots & \lambda_n v_n
\end{bmatrix}$$

$$PD = P \begin{bmatrix}
\lambda_1 & 0 & \cdots & 0 \\
0 & \lambda_2 & \cdots & 0 \\
\vdots & \vdots &   & \vdots \\
0 & 0 & \cdots & \lambda_n
\end{bmatrix} = 
\begin{bmatrix}
\lambda_1 v_1 & \cdots & \lambda_n v_n
\end{bmatrix}$$

Since these are equal and $P$ has an inverse because its columns are linearly independent eigenvectors, $A = PDP^{-1}$.
\end{proof}

\section{Orthogonality and Least Squares}

\end{document}